\begin{Answer}{1}
		Trống
	
\end{Answer}
\begin{Answer}{2}
		Trống
	
\end{Answer}
\begin{Answer}{3}
		Trống
	
\end{Answer}
\begin{Answer}{4}
		Trống
	
\end{Answer}
\begin{Answer}{5}
		Trống
	
\end{Answer}
\begin{Answer}{6}
		Trống
	
\end{Answer}
\begin{Answer}{7}
		Đổi: $80\,cm=0,8\,m$
		
		Diện tích sàn nhà hình chữ nhật là:
		$4\,.\,5=20\,(m^2)$
		
		Diện tích của viên gạch lát hình vuông là:
		$0,8\,.\,0,8=0,64\,(m^2)$
		
		Vậy để lát hết sàn nhà đó cần số viên gạch là:
		$20:0,64=31,25\approx 32$ (viên gạch)
	
\end{Answer}
\begin{Answer}{8}
		Để tính diện tích mũi tên ta chia mũi tên thành hai phần gồm hình tam giác và hình chữ nhật (như hình vẽ).
		
		Diện tích hình chữ nhật nhỏ màu trắng là: $30\,.\,10=300\,(cm^2)$
		
		Cạnh đáy của tam giác màu trắng là: $10+10+10=30\,(cm)$
		
		Đường cao của tam giác màu trắng là: $45-30=15\,(cm)$
		
		Diện tích tam giác màu trắng là: $\dfrac{1}{2}\,.\,30\,.\,15=225\,(cm^2)$
		
		Diện tích hình mũi tên là: $300+225=525\,(cm^2)$
		
		Diện tích hình chữ nhật lớn là: $50\,.\,40=2000\,(cm^2)$
		
		Diện tích phần tô màu xanh là: $2000-525=1475\,(cm^2)$
	
\end{Answer}
\begin{Answer}{9}
		\begin{enumerate}[a), leftmargin=*]
			\i Các hình $AHGD;CFGD$là những hình thang vuông.
			\i Hiệu diện tích giữa hai hình vuông chính là diện tích của hai hình thang vuông bằng nhau $AHGD$; $CFGD$\\
			Vậy ${{S}_{AHGD}}={{S}_{CFGD}}=\dfrac{120}{2}=60\,\,(cm^2)$\\
			Ta lại có: ${{S}_{AHGD}}=\dfrac{(AD+HG)\,.\,AH}{2}=60\,(cm^2)$\\
			$\Rightarrow \dfrac{12}{2}\,.\,AH=60$ \\
			$\Rightarrow AH=10\,\,(cm)$ \\
			$AHGD;CFGD$ là hai hình thang vuông bằng nhau nên $AH=CF=10\,(cm)$
			\i Cạnh hình vuông $ABCD$là: $\dfrac{10+12}{2}=11\,(cm)$.\\
			Cạnh hình vuông $BFGH$là: $12-11=1\,(cm)$.
		\end{enumerate}
	
\end{Answer}
\begin{Answer}{10}
		Để tính diện tích lối đi ta chia hình như sau:
		
		Hình bình hành $1$ là hình có cạnh đáy $2m$ và chiều cao $20\,m$.
		
		Diện tích hình bình hành 1 là: $20.2=40\,(m^2)$
		
		Diện tích hình thang vuông 2 là: $\dfrac{(36+38).2}{2}=74\,(m^2)$
		
		Diện tích lối đi là: $40+74=114\,\,(m^2)$
		
		Chi phí để làm lối đi hết: $114.110000=12540000$ (đồng)
	
\end{Answer}
\begin{Answer}{11}
		Cần bỏ ít nhất 4 que tính để còn lại hình gồm 6 tam giác nhỏ.
		Ta có hình như sau:
		
	
\end{Answer}
\begin{Answer}{12}
		Ta chia hình như sau:
		
		Nhìn một cạnh của tam giác cần tính là đường chéo của 1 hình bình hành gồm 20 tam giác nhỏ bằng nhau.
		
		$\Rightarrow$ Mỗi hình bình hành như trên có diện tích bằng 20 (đvdt).
		
		$\Rightarrow$ Một nửa của hình bình hành được tô màu cam, xanh dương, xanh lá có diện tích bằng 10 (đvdt).
		
		Diện tích tam giác cần tính bằng diện tích 3 nửa hình bình hành được tô màu cam, màu xanh dương, màu xanh lá cộng thêm diện tích 9 tam giác nhỏ bằng nhau: $3.10+9.1=39$ (đvdt).
	
\end{Answer}
\begin{Answer}{13}
		Ta đặt tên các điểm và vẽ thêm hai đường chéo của hình chữ nhật như sau.
		Khi ấy lá cờ được chia làm $12$ tam giác có diện tích bằng nhau; trong đó có $8$ tam giác xám và $4$ tam giác trắng. Vậy tỉ lệ diện thích của phần trắng và phần xám là $1:2$
	
\end{Answer}
\begin{Answer}{14}
		Ta dễ chứng minh được ${{S}_{IJKL}}=\dfrac{{{S}_{EFGH}}}{2}.$
		
		Lại có ${{S}_{EFGH}}={{S}_{ABCD}}-4{{S}_{EBH}}$. Mà ${{S}_{EBH}}=\dfrac{2}{3}{{S}_{ABH}}=\dfrac{2}{9}{{S}_{ABC}}=\dfrac{1}{9}{{S}_{ABCD}}.$
		
		Vậy ${{S}_{EFGH}}=\dfrac{5}{9}{{S}_{ABCD}}.$ Nên ${{S}_{IJKL}}=\dfrac{5}{18}{{S}_{ABCD}}$
	
\end{Answer}
\begin{Answer}{15}
		Ta có:
		\begin{align*}
			{{P}_{ABC}}&=AB+BC+AC \\
			& =AD+DB+BE+EC+CF+FA \\
			& =(AD+FA)+(DB+BE)+(EC+CF) \\
			& ={{P}_{ADF}}-DF+{{P}_{BDE}}-DE+{{P}_{CEF}}-EF \\
			& ={{P}_{ADE}}+{{P}_{BDE}}+{{P}_{CEF}}-(DF+DE+EF) \\
			& ={{P}_{ADE}}+{{P}_{BDE}}+{{P}_{CEF}}-{{P}_{DEF}} \\
			& =12+24+24-19=41 \\
		\end{align*}
		Vậy chu vi tam giác $ABC$ là 41 (đvđd)
	
\end{Answer}

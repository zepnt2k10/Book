\begin{Answer}{90}
		Các hình có trục đối xứng là: Hình 1, Hình 2, Hình 3, Hình 4, Hình 5, Hình 6, Hình 7.
		
		Các hình có tâm đối xứng là: Hình 1, Hình 3, Hình 5, Hình 6.
	
\end{Answer}
\begin{Answer}{91}
		Các hình có tâm đối xứng là:
	
\end{Answer}
\begin{Answer}{92}
		\begin{enumerate}[a), leftmargin=*]
			\i Các vật dụng có trục đối xứng: Cổng nhà, con diều, bình  hoa, cửa sổ, \ldots
			
			\i Các vật dụng có tâm đối xứng: Mặt của thớt, ạch hoa lát nền, mặt đồng hồ, cỏ 4 lá, lá lốt \ldots
		\end{enumerate}
	
\end{Answer}
\begin{Answer}{93}
		chèn ảnh
	
\end{Answer}
\begin{Answer}{94}
		chèn ảnh
	
\end{Answer}
\begin{Answer}{95}
		\begin{enumerate}[a), leftmargin=*]
			\i Các chữ chỉ có tâm đối xứng: N, S.
			\i Các chữ chỉ có trục đối xứng: V, E, M, A, T, D.
			\i Các chữ vừa có tâm, vừa có trục đối xứng: O, H, I.
		\end{enumerate}
	
\end{Answer}
\begin{Answer}{96}
		chèn ảnh
	
\end{Answer}
\begin{Answer}{97}
		\begin{enumerate}[a), leftmargin=*]
			\i Hình có trục đối xứng là: Hình 1, Hình  4, Hình  5.
			\i Hình có tâm đối xứng là: Hình 6, Hình 8, Hình  9.
		\end{enumerate}
	
\end{Answer}
\begin{Answer}{98}
		Hình a có 1 trục đối xứng
		
		Hình b có 2 trục đối xứng
		
		Hình c có 2 trục đối xứng
		
		Hình d có 4 trục đối xứng
	
\end{Answer}
\begin{Answer}{99}
		Các em tự vẽ
	
\end{Answer}
\begin{Answer}{100}
		\begin{enumerate}[a), leftmargin=*]
			\i Ta có thể ghép thành các hình như sau:

			\i Các hình có tâm đối xứng từ 4 hình vuông đã cho như sau
			
		\end{enumerate}		
	
\end{Answer}
\begin{Answer}{101}
		Nếu $n$ là số chẵn ta sẽ chia đều hai bên và ghép theo ý tưởng hình ở đề bài
		Nếu $n$ lẻ ta xếp thep ý tưởng sau
	
\end{Answer}

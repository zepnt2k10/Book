\begin{Answer}{1}
		\begin{enumerate}[a),leftmargin=*]
			\i Khi tiến hành thống kê lớp trưởng lớp  $6A$ cần thu thập thông tin về loại nhạc cụ yêu thích nhất của các học sinh trong lớp.
			\i Đối tượng thống kê là  loại nhạc cụ: Organ, Ghita, Kèn, Trống, Sáo.\\
			Tiêu chí thống kê là số học sinh yêu thích từng loại nhạc cụ đó.
			\ Số thành viên trong câu lạc bộ theo thống kê của lớp trưởng là: $12 + 7+ 15 +25 +15 = 74$ (học sinh)\\
			Theo quy định, mỗi lớp ở bậc THCS có không quá 45 HS. Thực tế, do điều kiện khó khăn, một lớp có số học sinh nhiều hơn 45 HS nhưng không có lớp nào có 74 học sinh,  74 là giá trị không hợp lí.
		\end{enumerate}
	
\end{Answer}
\begin{Answer}{2}
		thieeus
	
\end{Answer}
\begin{Answer}{3}
		\begin{enumerate}[a),leftmargin=*]
			\i Đối tượng thống kê là  30 xe moto đi qua trạm kiểm soát giao thông
			Tiêu chí thống kê là tốc độ đi của  30 xe moto.
			\i Trong bảng trên xe đi với tốc độ lớn nhất là  $80\, km/h$
			\i Ở tốc độ $60\, km/h$ có nhiều xe đi nhất, có  5 xe đi .
			\i Nếu đoạn đường đó cho phép xe moto đi với tốc độ tối đa là  $60 \,km/h$ thì có 12  xe vi phạm luật giao thông đường bộ.
		\end{enumerate}
	
\end{Answer}
\begin{Answer}{4}
		\begin{enumerate}[a),leftmargin=*]
			\i Khi tiến hành thống kê, bạn Thanh cần thu thập thông tin về size áo phông của từng bạn trong lớp 6A1.
			\begin{enumerate}[+,leftmargin=*]
				\i Đối tượng thống kê là các size áo : S, M, L.
				\i Tiêu chí thống kê số bạn mặc vừa trong mỗi size.
			\end{enumerate}
			\i Lớp 6A1 có 30 bạn.
			\i Bạn Thanh nói không đúng, có 8 bạn mặc size S.
			\i
			\begin{tabular}{|c|c|c|c|}
				\hline
				Size áo &	S&	M&	L\\
				\hline
				Số bạn mặc&	8&	14&	8\\
				\hline
			\end{tabular}
		\end{enumerate}
	
\end{Answer}
\begin{Answer}{5}
		\begin{enumerate}[a),leftmargin=*]
			\i \begin{tabular}{|c|c|c|c|c|c|c|}
				\hline
				Số học sinh nghỉ&	0&	1	&2&	3&	4&	5\\
				\hline
				Số buổi&	5&	6&	6&	3&	4&	2\\
				\hline
			\end{tabular}
			\i Số học sinh nghỉ học nhiều nhất trong một buổi là 5 học sinh.\\
			Trung bình mỗi buổi học, số học sinh nghỉ là:
			\[\dfrac{{5.0 + 6.1 + 6.2 + 3.3 + 4.4 + 2.5}}{{26}} = 2 \text{ (học sinh)}\]
		\end{enumerate}
	
\end{Answer}
\begin{Answer}{6}
		\begin{enumerate}[a),leftmargin=*]
			\i Thông tin chưa hợp lý của bảng dữ liệu trên là K, -2, 100.
			\i Các thông không hợp lý trên vi phạm tiêu chí:
			\begin{enumerate}[-,leftmargin=*]
				\i K: dữ liệu phải là số
				\i 100: Số học sinh trong một lớp không quá 45 nên số học sinh vắng trong ngày không thể là 100
				\i -2: Số học sinh vắng phải là số tự nhiên.
			\end{enumerate}	
		\end{enumerate}
	
\end{Answer}
\begin{Answer}{7}
		\begin{enumerate}[--,leftmargin=*]
			\i Loại giày shop bán chạy nhất là loài giày có giá 250 nghìn.
			\i Loại giày có giá 450 nghìn có lượng tiêu thụ ít nhất
			\i Nếu là chủ shop giày, em sẽ nhập mẫu giày có giá 250 nghìn, 350 nghìn, 300  nghìn,  400 nghìn để bán nhiều hơn cho các tháng tiếp theo.
		\end{enumerate}
		$^*$Không nên dừng nhập mẫu giày có giá 450 nghìn đồng vì vẫn có khách hàng mua, tuy nhiên lượng nhập vào nên giảm đi và tìm mọi cách để kích cầu dòng sản phẩm này
		Ví dụ: quảng cáo thêm cho sản phẩm trên các trang mạng xã hội, tặng thêm hàng khuyến mại, giảm giá \ldots.
	
\end{Answer}
\begin{Answer}{8}
		\begin{enumerate}[a),leftmargin=*]
			\i Đối tượng thống kê là các điểm số: 3, 4, 5, 6, 7, 8, 9, 10.\\
			 Tiêu chí thống kê là số học sinh ứng với mỗi loại điểm.
			\i \begin{tabular}{|c|c|c|c|c|c|c|c|c|}
				\hline
					Điểm&	3&	4	&5&	6&	7&	8&	9&	10\\
					\hline
				Số học sinh&	1&	2&	7&	10&	7&	6	&4&	3\\
				\hline
			\end{tabular}
		
			\i Điểm trung bình của 10 bạn có điểm thấp nhất là  $\dfrac{{3.1 + 4.2 + 5.7}}{{10}} = \dfrac{{46}}{{10}} = 4,6$\\
			Điểm trung bình của 10 bạn có điểm cao nhất là  $\dfrac{{10.3 + 9.4 + 8.3}}{{10}} = \dfrac{{90}}{{10}} = 9,0$
		\end{enumerate}
	
\end{Answer}

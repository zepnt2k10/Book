\begin{Answer}{19}
		\begin{enumerate}[a),leftmargin=*]
			\i $X_1 = \{\text{ĐX; ĐĐo; XĐ; XĐo; ĐoĐ; ĐoX}\}$
			\i $X_2 = \{\text{ĐXĐo; ĐĐoX; XĐĐo; XĐoĐ; ĐoĐX; ĐoXĐ}\}$
		\end{enumerate}
	
\end{Answer}
\begin{Answer}{20}
	$\{$2 sấp; 2 ngửa; 1 sấp 1 ngửa$\}$
	
\end{Answer}
\begin{Answer}{21}
		\begin{enumerate}[a),leftmargin=*]
			\i Có 8 kết quả có thể: 10; 20; 30; 40; 50; 60; 70; 80.
			\i 10; 20; 30; 40; 50; 60; 70.
			\i Sự kiện xảy ra.
		\end{enumerate}
	
\end{Answer}
\begin{Answer}{22}
		\begin{enumerate}[a),leftmargin=*]
			\i Các sự kiện có thể xảy ra là: Nhi lấy ra được quả bóng màu xanh, Nhi lấy ra được quả bóng màu đỏ hoặc Nhi lấy ra được quả bóng màu vàng.
			\i Sự kiện “Nhi lấy được quả bóng màu xanh” không luôn xảy ra vì có thể quả bóng Nhi lấy ra có màu đỏ hoặc màu vàng.
			\i Xác suất lấy được quả bóng màu xanh là: $\dfrac{4}{{4 + 3 + 3}} = \dfrac{2}{5}$
		\end{enumerate}
	
\end{Answer}
\begin{Answer}{23}
		\begin{enumerate}[a),leftmargin=*]
			\i Sự kiện không thể xảy ra.
			\i Sự kiện không thể xảy ra..
			\i Sự kiện chắc chắn xảy ra.
			\i Sự kiện có thể xảy ra.
		\end{enumerate}
	
\end{Answer}
\begin{Answer}{24}
		\begin{enumerate}[a),leftmargin=*]
			\i Không chắc chắn được bạn nào sẽ là lớp trưởng.
			\i Bạn đó nói chưa chắc đúng vì lớp trưởng có thể là Chính (bạn nam).
			\i Kết quả có thể để sự kiện "Lớp trưởng không phải là Chính" xảy ra là: Thanh, Ly, Linh.
		\end{enumerate}
	
\end{Answer}
\begin{Answer}{25}
			a) $\dfrac{n(X)}{n} = \dfrac{10}{20} = 50\%$\quad\quad
			b) $\dfrac{n(Đ)}{n} = \dfrac{6}{20} = 30\%$\quad\quad
			c) $\dfrac{n(V)}{n} = \dfrac{4}{20} = 20\%$
	
\end{Answer}
\begin{Answer}{26}
		$\dfrac{{n(S)}}{n} = \dfrac{6}{{15}} = 40\% $
	
\end{Answer}
\begin{Answer}{27}
		\begin{enumerate}[a),leftmargin=*]
			\i $\dfrac{{n(X)}}{n} = \dfrac{{43}}{{100}} = 43\% $
			\i Tổng số lần lấy ra không phải quả bóng màu đỏ là: $43 + 18 + 17 = 78$
			\[\dfrac{m(\text{kĐ})}{n} = \dfrac{78}{100} = 78\&\]
		\end{enumerate}
	
\end{Answer}
\begin{Answer}{28}
		\begin{enumerate}[a),leftmargin=*]
			\i \begin{enumerate}[--,leftmargin=*]
				\i Sự kiện chắc chắn xảy ra: (4).
				\i Sự kiện không thể xảy ra: (2).
				\i Sự kiện có thể xảy ra: (1), (3).
			\end{enumerate}
			\i Phải lấy ra ít nhất 4 quả bóng để tổng các số trên các quả bóng chắc chắn lớn hơn 5 khi trường hợp lấy được các quả bóng được đánh số nhỏ nhất là  $0 + 1 + 2 + 3 = 6$
		\end{enumerate}
	
\end{Answer}
\begin{Answer}{29}
		Muốn xem An và Bình ai là người thắng cuộc thì ta phải tính số điểm của An và Bình rồi so sánh để tìm được người thắng cuộc.
		
		An chọn số 3, kết quả gieo của An là $2,3,6,4,3$ nên An được số điểm là:
		\[ - 5 + 10 - 5 - 5 + 10 = 5 \text{ (điểm)}\]
		Bình chọn số 4, kết quả gieo của Bình là $4,3,4,5,4$ nên Bình được số điểm là:
		\[10 - 5 + 10 - 5 + 10 = 20 \text{ (điểm)}\]
		Số điểm của Bình nhiều hơn so với điểm của An. Vậy Bình thắng cuộc.
	
\end{Answer}
\begin{Answer}{30}
		\begin{enumerate}[a),leftmargin=*]
			\i $\dfrac{{n(NTH{\rm{S}})}}{n} = \dfrac{{20 + 18 + 22 + 10 + 15}}{{100}} = \dfrac{{85}}{{100}} = 85\% $
			\i ) Số chấm xuất hiện không là số nguyên tố, cũng không là hợp số chính là sự xuất hiện của mặt 1 chấm: $\dfrac{{n(1)}}{n} = \dfrac{{15}}{{100}} = 15\%$
		\end{enumerate}
	
\end{Answer}
\begin{Answer}{31}
		\begin{enumerate}[a),leftmargin=*]
			\i 12 lần
			\i $\dfrac{3}{12} = 0,25$
		\end{enumerate}
	
\end{Answer}
\begin{Answer}{32}
		\begin{enumerate}[a),leftmargin=*]
			\i Bảng thống kê:
			\begin{center}
				\begin{tabular}{|l|c|c|c|c|c|}
					\hline
					Số tuổi & 27&28&35&41&43\\
					\hline	
					Số công nhân & 3&4&7&3&1\\
					\hline	
				\end{tabular}
			\end{center}
			\i	Công nhân ở độ tuổi  35 có số lượng nhiều nhất.
			\i	Xác suất thực nghiệm của sự kiện công nhân có tuổi trẻ nhất là: $\dfrac{3}{18} = 0,17$
		\end{enumerate}
	
\end{Answer}
\begin{Answer}{33}
		Để Bình thắng ở lượt chơi này thì Bình phải quay vào các nấc điểm là  85;  90; 95; 100.
		
		Xác suất thực nghiệm của sự kiện Bình thắng ở lượt chơi này là:  $\dfrac{4}{{20}} = 20\% $.
	
\end{Answer}
\begin{Answer}{34}
		Có 20 cuốn truyện, mỗi lần lấy ra hai cuốn truyện vậy tổng số lần có thể lấy ra là:  $\dfrac{20.19}{2}= 190$ (lần)
		\begin{enumerate}[a),leftmargin=*]
			\i Xác suất để lấy được hai cuốn truyện cổ tích là: $\dfrac{{9.8:2}}{{190}} = \dfrac{{18}}{{95}}$
			\i Xác suất để lấy được hai cuốn truyện trong đó có một cuốn truyện cổ tích và một cuốn truyện cười là:  $\dfrac{{9.6}}{{190}} = \dfrac{{27}}{{95}}$
			\i Số cách lấy ra được hai cuốn truyện tranh là:  $5.4 = 20$\\
			Số cách lấy ra được một cuốn truyện tranh và một cuốn truyện cổ tích hoặc một cuốn truyện cười là:  $5.\left( {9 + 6} \right) = 75$\\
			Xác suất để lấy được hai cuốn truyện trong đó có ít nhất một cuốn truyện tranh là:  $\dfrac{{20 + 75}}{{190}} = \dfrac{1}{2}$
		\end{enumerate}
	
\end{Answer}
\begin{Answer}{35}
		Tổng số học sinh khối 6 của trường THCS X là:\[n  =  40  +  20  + 15  + 15  +  30  +  10  +  5  +  15  +  20  =  170\]
		\begin{enumerate}[a),leftmargin=*]
			\i Xác suất thực nghiệm của sự kiện một học sinh Môn Toán đạt lọai giỏi là:
			\[\dfrac{{{\rm{40 + 20 + 15}}}}{{{\rm{170}}}}{\rm{ = }}\dfrac{{{\rm{75}}}}{{{\rm{170}}}}{\rm{ = }}\dfrac{{{\rm{15}}}}{{{\rm{34}}}}\]
			\i Xác suất thực nghiệm cůa sự kiện một học sinh đạt loại khá trở lên ở cả hai môn là:
			\[\dfrac{{40 + 20 + 15 + 30}}{{170}} = \dfrac{{105}}{{170}} = \dfrac{{21}}{{34}}\]
			\i Xác suất thực nghiệm cůa sự kiện một học sinh đạt loại trung bình ít nhất một môn là:
			\[\dfrac{{15 + 10 + 20 + 5 + 15}}{{170}} = \dfrac{{65}}{{170}} = \dfrac{{13}}{{34}}\]
		\end{enumerate}
	
\end{Answer}
\begin{Answer}{36}
		\begin{enumerate}[a),leftmargin=*]
			\i	Số điểm mà Nam có được sau lần ném thứ 19 là:
			\[5.5 + 9.3 + 1.\left( { - 2} \right) + 5.\left( { - 1} \right) = 45\]
			\i	Để đạt được 50 điểm, Nam cần thêm $50-45 =5$ nữa. Do đó Nam vẫn còn cơ hội đạt được 50 điểm. Muốn vậy Nam cần phải ném bi vào ô 5 điểm ở lần cuối cùng.
		\end{enumerate}
	
\end{Answer}

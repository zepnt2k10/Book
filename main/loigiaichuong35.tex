\begin{Answer}{74}
		\begin{enumerate}[a), leftmargin=*]
			\i Hình 2, 3.
			\i Hình 1, 2, 3.
			\i Hình 1.
			\i Hình 1, 3.
		\end{enumerate}
	
\end{Answer}
\begin{Answer}{75}
		\begin{enumerate}[a), leftmargin=*]
			\i Hình vuông.
			\i Đường chéo.
			\i  $60^\circ$.
			\i Bốn.
			\i Hình thang cân.
			\i Song song và bằng nhau
			\i Ba.
			\i  $120^\circ$.
		\end{enumerate}
	
\end{Answer}
\begin{Answer}{76}
		\begin{enumerate}[--, leftmargin=*]
			\i Hình chữ nhât: Cửa sổ, bàn học, khung ảnh, ti vi, quyển sách, \ldots
			\i Hình bình hành: mái nhà, cầu thang, \ldots
			\i Hình thoi: Họa tiết gạch, câu đối, hoa văn chiếu trúc, móc treo đồ, \ldots
		\end{enumerate}
	
\end{Answer}
\begin{Answer}{77}
		\begin{enumerate}[Bước 1:, leftmargin=*]
			\i Vẽ đoạn thẳng $AB = 6\, cm$.
			\i Vẽ đường thẳng vuông góc với $AB$ tại  $A$. Trên đường thẳng đó lấy điểm $D$ sao cho $AD = 4\, cm$.
			\i Vẽ đường thẳng qua $D$ và song song với $AB$. Vẽ đường thẳng qua $B$ và song song với $AD$.
			\i Hai đường thẳng này cắt nhau tại $C$. Ta được hình chữ nhật $ABCD$ cần vẽ.
		\end{enumerate}
	
\end{Answer}
\begin{Answer}{78}
		\begin{enumerate}[Bước 1:, leftmargin=*]
			\i Vẽ đoạn thẳng $AB = 5\,cm$.
			\i Vẽ đường thẳng đi qua $B$. Lấy điểm $C$  trên đường thẳng đó sao cho $BC = 5\,cm$.
			\i Vẽ đường thẳng đi qua $C$ và song song với cạnh $AB$. Vẽ đường thẳng đi qua $A$ và song song với cạnh $BC$ ta được hình thoi $ABCD$ có cạnh $5\,cm$.
		\end{enumerate}
	
\end{Answer}
\begin{Answer}{79}
		\begin{enumerate}[Bước 1:, leftmargin=*]
			\i Vẽ đoạn thẳng $AB = 3\,cm$
			\i Vẽ đường thẳng đi qua $B$. Trên đường thẳng đó lấy $C$ sao cho $BC= 2\,cm$.
			\i Vẽ đường thẳng đi qua $A$ và song song với  $BC$, đường thẳng qua $C$  và song song với $AB$.
			\i Hai đường thẳng này cắt nhau tại  $D$ ta được hình bình hành $ABCD$.
		\end{enumerate}
	
\end{Answer}
\begin{Answer}{80}
		\begin{enumerate}[--, leftmargin=*]
			\i Có 2 hình chữ nhật.
			\i Có 4 hình thang cân.
		\end{enumerate}
	
\end{Answer}
\begin{Answer}{81}
		\begin{enumerate}[a), leftmargin=*]
			\i \begin{enumerate}[--, leftmargin=*]
				\i Có 21 hình bình hành
				\i Có 14 hình thoi
				\i Có 14 hình thang cân
				\end{enumerate}
			\i \begin{enumerate}[--, leftmargin=*]
				\i Có 16 hình bình hành
				\i Có 5 hình thoi
				\i Có 14 hình thang cân
			\end{enumerate}
		\end{enumerate}
	
\end{Answer}
\begin{Answer}{82}
		\begin{enumerate}[a), leftmargin=*]
			\i \begin{enumerate}[Bước 1:, leftmargin=*]
				\i Gọi $M$ là trung điểm của  $CD$.
				\i Trên $CD$ xác định điểm $E$ sao cho $ME = 1,5\,cm$. Trên $CD$ xác định điểm $G$  sao cho $MG = 1,5\, cm$.  Ta được hình thang cân $ABEG$.
				\end{enumerate}
			\i \begin{enumerate}[Bước 1:, leftmargin=*]
				\i Trên đường thẳng chứa cạnh $CD$ xác định điểm $G$ nằm ngoài đoạn $CD$ sao cho $DG = 1\,cm$
				\i Trên đường thẳng chứa cạnh $CD$ xác định điểm $E$ nằm ngoài đoạn $CD$ sao cho $CE =1\, cm$. Ta được hình thang cân  $ABEG$.
			\end{enumerate}
		\end{enumerate}
	
\end{Answer}
\begin{Answer}{83}
		Ta có 3 hình tam giác đều giống nhau:
		
		
		Ghép 3 tam giác đều lại với nhau ta được hình thang cân như sau:
		
	
\end{Answer}
\begin{Answer}{84}
		Ghép 4 tam giác vuông như nhau ta được các hình như sau:
		\begin{enumerate}[a), leftmargin=*]
			\i Hình chữ nhật:
			
			\i Hình bình hành:
			
			\i Hình thang cân
			
			\i Hình thoi:
		\end{enumerate}
		
	
\end{Answer}
\begin{Answer}{85}
		Ta chia hình thoi theo đường lưới như hình vẽ. Sau đó cắt các cạnh của hình thoi nhỏ được đánh dấu, gập các miếng có cạnh bị cắt để được hình theo yêu cầu
	
\end{Answer}
\begin{Answer}{86}
		Tiến sĩ John có thể mở được khóa này bằng cách đặt 17 quân domino như sau:
	
\end{Answer}
\begin{Answer}{87}
		chèn ảnh
	
\end{Answer}
\begin{Answer}{88}
		chèn ảnh
	
\end{Answer}
\begin{Answer}{89}
		\begin{enumerate}[a), leftmargin=*]
			\i Ghép như hình bên:
			
			\i Vì mỗi họa tiết hình thang cân chia được 3 tam giác đều. Tam giác đều cạnh 10cm chia được 100 tam giác đều. 100 chia 3 dư 1 nên không thể ghép các hình thang cân lại thành tam giác đều cạnh 10cm được.
		\end{enumerate}
	
\end{Answer}

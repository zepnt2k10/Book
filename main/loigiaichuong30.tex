\begin{Answer}{1}
		\begin{enumerate}[a),leftmargin=*]
			\i Các góc trong hình vẽ:
			\begin{enumerate}[--,leftmargin=*]
				\i $\widehat{xOy}$: đỉnh $O$; cạnh $Ox,Oy$.
				\i $\widehat{yOz}$: đỉnh $O$; cạnh $Oy,Oz$.
				\i $\widehat{xOz}$: đỉnh $O$; cạnh $Ox,Oz$.
			\end{enumerate}
			\i Các điểm nằm trong góc $\widehat{xOy}:\,\,N,P$.
			\begin{enumerate}[--,leftmargin=*]
				\i Các điểm nằm ngoài góc $\widehat{xOy}:\,\,M,K$.
				\i Các điểm nằm trong góc $\widehat{yOz}:\,\,M$.
				\i Các điểm nằm ngoài góc $\widehat{yOz}:\,\,N,P,K$.
				\i Các điểm nằm trong góc $\widehat{xOz}:\,\,M,N,P$.
				\i Các điểm nằm ngoài góc $\widehat{xOz}:\,\,K$.
			\end{enumerate}
		\end{enumerate}
	
\end{Answer}
\begin{Answer}{2}
		Các góc có đỉnh $A:\,\,\widehat{BAH},\widehat{BAM},\widehat{BAC},\widehat{HAM},\widehat{HAC},\widehat{MAC}$.
		
		Các góc có đỉnh $M:\,\,\widehat{AMC},\widehat{AMH},\widehat{AMB}$.
		
		Các góc có đỉnh .
	
\end{Answer}
\begin{Answer}{3}
		\begin{enumerate}[a),leftmargin=*]
			\i Góc nhọn: $\widehat{MNP}$; góc tù: $\widehat{mDn}$; góc vuông: $\widehat{ABC},\widehat{xGk},\widehat{yOz}$; góc bẹt: $\widehat{FEQ}$.
			\i Dùng êke kiểm tra lại kết quả.
			\i Dùng thước đo góc: $\widehat{MNP}={{30}^\circ},\widehat{yOz}={{90}^\circ},\widehat{ABC}={{90}^\circ},\widehat{mDn}={{135}^\circ},\widehat{FEQ}={{180}^\circ},\widehat{xGk}={{90}^\circ}$
		\end{enumerate}
	
\end{Answer}
\begin{Answer}{4}
		Góc tạo bởi kim giờ và kim phút là:
		\begin{enumerate}[a),leftmargin=*]
			\i góc nhọn: lúc 2 giờ, 11 giờ.
			\i góc tù: lúc 4 giờ, 5 giờ.
			\i góc bẹt: lúc 6 giờ.
			\i góc vuông: lúc 3 giờ, 9 giờ.
		\end{enumerate}
	
\end{Answer}
\begin{Answer}{5}
		Hình ảnh thực tế về góc:
		\begin{enumerate}[+,leftmargin=*]
			\i Góc tạo bởi kim phút và kim giây của đồng hồ.
			\i Góc tạo bởi cái bóng và cây cột giữa trời nắng và mặt đất.
		\end{enumerate}
	
\end{Answer}
\begin{Answer}{6}
		\begin{enumerate}[a),leftmargin=*]
			\i Có 12 góc.
			\i $\widehat{ABC}={{55}^\circ};\widehat{BAC}={{80}^\circ};\widehat{ACB}={{45}^\circ};\widehat{ADB}={{110}^\circ};\widehat{BDC}={{130}^\circ};\widehat{ADC}={{120}^\circ}$
			$\widehat{CBD}={{25}^\circ},\widehat{DBA}={{20}^\circ},\widehat{BCD}={{20}^\circ},\widehat{DCA}={{25}^\circ},\widehat{BAD}={{40}^\circ},\widehat{CAD}={{40}^\circ}$
		\end{enumerate}
	
\end{Answer}
\begin{Answer}{7}
		\begin{enumerate}[a),leftmargin=*]
			\i Tam giác $ABC$ có:
			\begin{enumerate}[--,leftmargin=*]
				\i $\widehat{BAC}={{30}^\circ}$
				\i $\widehat{ABC}={{70}^\circ}$
				\i $\widehat{ACB}={{80}^\circ}$
				\i $\widehat{BAC}+\widehat{ABC}+\widehat{ACB}={{180}^\circ}$
			\end{enumerate}
			\i Tam giác đều $DEG$ có:
			\begin{enumerate}[--,leftmargin=*]
				\i $\widehat{D}={{60}^\circ}$
				\i $\widehat{G}={{60}^\circ}$
				\i $\widehat{E}={{60}^\circ}$
			\end{enumerate}
		\end{enumerate}
	
\end{Answer}
\begin{Answer}{8}
		Góc tạo bởi kim giờ và kim phút của đồng hồ:
		\begin{enumerate}[--,leftmargin=*]
			\i lúc 4 giờ: ${{120}^\circ}$
			\i lúc 9 giờ: ${{90}^\circ}$
			\i lúc 11 giờ: ${{30}^\circ}$
			\i lúc 6 giờ: ${{180}^\circ}$
		\end{enumerate}
	
\end{Answer}
\begin{Answer}{9}
		Những vạch số nằm trong góc tạo bởi
		\begin{enumerate}[a),leftmargin=*]
			\i kim giây và kim phút: 7; 6; 5; 4; 3
			\i kim giờ và kim phút: 11; 12; 1; 1
		\end{enumerate}
	
\end{Answer}
\begin{Answer}{10}
		\begin{enumerate}[a),leftmargin=*]
			\i Các góc có trong hình vẽ: $\widehat{mOz},\widehat{mOy},\widehat{mOx},\widehat{zOy},\widehat{zOx},\widehat{yOx}$
			\i Góc nhọn: $\widehat{mOz},\widehat{zOy},\widehat{yOx}$; góc vuông: $\widehat{mOn},\widehat{xOz}$; góc tù: $\widehat{xOm}$.
		\end{enumerate}
	
\end{Answer}
\begin{Answer}{11}
		A. Phòng bếp.
	
\end{Answer}
\begin{Answer}{12}
		Số góc tạo bởi hai trong năm tia là:
		\begin{enumerate}[--,leftmargin=*]
			\i Số góc tạo bởi tia $Ox$ với 1 trong 4 tia còn lại là: 4 góc.
			\i Số góc tạo bởi tia $Om$ với 1 trong 3 tia còn lại (không kể $Ox$) là: 3 góc.
			\i Số góc tạo bởi tia $Oy$ với 1 trong 2 tia còn lại (không kể $Ox,Oy$) là: 2 góc.
			\i Số góc tạo bởi tia $On$ với tia $Ot$ còn lại là: 1 góc.
		\end{enumerate}
		Vậy số góc tạo thành là: $4+3+2+1=10$ góc.
	
\end{Answer}
\begin{Answer}{13}
		Xét góc tạo bởi tia thứ nhất với 1 trong 3 tia còn lại có 3 góc.\\
		Xét góc tạo bởi tia thứ 2 với 1 trong 2 tia còn lại có 2 góc.\\
		Xét góc tạo bởi tia thứ 3 với 1 tia còn lại có 1 góc.\\
		Do đó, số góc tạo thành: $3+2+1=6$ góc.\\
		Vì $Oy,On$ là hai tia đối nên tạo thành 1 góc bẹt.\\
		Vậy số góc tạo bởi 2 trong 4 tia không kể góc bẹt là: $6-1=5$ góc.
	
\end{Answer}
\begin{Answer}{14}
		Số góc tạo bởi tia thứ nhất với 1 trong $n-1$ tia còn lại là $n-1$ góc.\\
		Số góc tạo bởi tia thứ hai với 1 trong $n-2$ tia còn lại là $n-2$ góc.\\
		\ldots \\
		Số góc tạo bởi tia thứ $n-1$ với tia thứ $n$ là 1 góc.\\
		Do đó, tổng số góc là: $1+2+...+n-2+n-1=n\cdot\left( n-1 \right):2$\\
		Ta có: $n\cdot\left( n-1 \right):2=21\Rightarrow n\cdot\left( n-1 \right)=42$\\
		mà $42=7\cdot6$ nên $n=7$
	
\end{Answer}
\begin{Answer}{15}
		Ta có:\\
		Xóa 1 tia gốc $O$ thì số góc giảm đi 10\\
		Khi chưa xóa tia, số góc tạo bởi tia đó với $n-1$ tia còn lại là $n-1$ góc.\\
		Số góc giảm đi 10 khi xóa 1 tia nên: $n-1=10 \Rightarrow n=11$.
	
\end{Answer}
\begin{Answer}{16}
		Ta có: $\widehat{MAN}$ là góc bẹt nên $\widehat{MAN}={{180}^\circ}$.\\
		$\widehat{MAT}+\widehat{NAT}=\widehat{MAN}\Rightarrow \widehat{MAT}+\widehat{NAT}={{180}^\circ}$\\
		mà  $\widehat{MAT}-\widehat{NAT}={{8}^\circ}$  nên:
		$\widehat{MAT}=\left({{180}^\circ}+{{8}^\circ}\right):2={{94}^\circ}$ \\
		$\widehat{NAT}=\left({{180}^\circ}-{{8}^\circ}\right):2={{86}^\circ}$ \\
	
\end{Answer}

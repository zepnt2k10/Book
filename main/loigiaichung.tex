\begin{Answer}{B.}
		\begin{enumerate}[a),leftmargin=*]
			\i Đối tượng thống kê ở đây là:
			\begin{enumerate}[+,leftmargin=*]
				\i Các thành viên trong đội của Bác Hùng
				\i Tiêu chí thống kê là số tuổi của mỗi người trong đội.
			\end{enumerate}
			\i Có 9 người từ  45 tuổi trở xuống
			
			Và có 18 người từ 45 tuổi trở lên.
			
			Vậy thông báo của bác Hùng là sai
		\end{enumerate}
	
\end{Answer}
\begin{Answer}{1}
		\begin{enumerate}[a),leftmargin=*]
			\i Tổ của Minh có 16  học sinh.
			\begin{enumerate}[--,leftmargin=*]
				\i Đối tượng thống kê ở đây là: Các thành viên trong tổ của Minh.
				\i Tiêu chí thống kê là số điểm kiểm tra giữa kì môn Toán của các thành viên trong tổ.
			\end{enumerate}
			\i Lập bảng thống kê:
			\begin{center}
				\begin{tabular}{|l|c|c|c|c|c|c|c|}
					\hline
					Số điểm & 10 & 9&8&7&6&5&3\\
					\hline
					Số người& 3&3&3&4&1&1&1\\
					\hline
				\end{tabular}
			\end{center}	
			Từ bảng thống kê trên ta thấy:
			\begin{enumerate}[+,leftmargin=*]
				\i Tổ của Minh có 14  bạn đạt điểm trên trung bình( trên 5 điểm)
				\i Có 13 bạn đạt điểm khá giỏi
			\end{enumerate}
			Trong tổ của Minh có điểm 10 điểm là điểm cao nhất nên Minh được 8 điểm chưa phải là điểm cao nhất.
		\end{enumerate}
	
\end{Answer}
\begin{Answer}{2}
		\begin{enumerate}[a),leftmargin=*]
			\i Trong 4 loại virus trên, virus gây ra tỉ lệ tử vong cao nhất là Marburg, virus đó xuất hiện năm 1976.
			\i Thế giới phát hiện ra loại virus SARS vào năm 2002
			\begin{enumerate}[+,leftmargin=*]
				\i Tỉ lệ tử vong của SARS là  $9,6\%$
				\i Tỉ lệ tử vong của Sars--CoV--2 là $2,3 \%$
			\end{enumerate}
			Vậy tỉ lệ tử vong của SARS lớn hơn Sars--CoV--2.
			\i Chủng virus SARS--CoV--2 thuộc cùng một họ với SARS--CoV, được gọi là virus Corona, xuất hiện lần đầu vào tháng 12 năm 2019 tại thành phố Vũ Hán của Trung Quốc. COVID--19 do virus SARS--CoV--2 gây ra có tỉ lệ tử vong ước tính khoảng $2,3\%$ (tính đến tháng 3.2020). Các triệu chứng thường gặp bao gồm sốt, ho khan và khó thở và có thể tiến triển thành viêm phổi.
		\end{enumerate}
	
\end{Answer}
\begin{Answer}{3}
		Tính đến 0 giờ ngày 31-12-2020
		\begin{enumerate}[a),leftmargin=*]
			\i Khu vực Châu Âu có số ca mắc Covid--19 cao nhất thế giới (23.166.545 triệu người)
			\i Tổng số ca tử vong trên thế giới là:
			\[533444 + 505233 + 335174 + 359316 + 64166 + 1058 = 1.798.391 \text{ (triệu người)}\]
			\i Tỉ lệ tử vong so với số ca mắc của Châu Á là:  $\frac{{335174}}{{20532735}} \approx 1,63\%$
		\end{enumerate}
	
\end{Answer}
\begin{Answer}{4}
		\begin{enumerate}[--,leftmargin=*]
			\i Đối tượng thống kê người tải TikTok tại Việt Nam trong các tháng từ tháng 1/2020 đến tháng 02/2021
			\i Tiêu chí thống kê là số lượt tải TikTok trên một tháng được thống kê từ tháng 01/2020 đến tháng 02/2021 tại Việt Nam
			\i Vào 04/2020 số người tải TikTok cao nhất, có 1,94 triệu lượt
			\i Vào khoảng từ tháng 01/2020 đến tháng 08/2020 có số lượt tải rất cao, từ 1,25  đến 1,94 triệu lượt
			Từ tháng 09/2020 đến tháng 02/2021 số lượt tải giảm đi rõ rệt, chỉ từ 0,78 đến 1,2 triệu lượt
			\i Tháng 12/ 2020 có số lượt tait TikTok là thấp nhất, có  lượt \ldots.
		\end{enumerate}
	
\end{Answer}
\begin{Answer}{5}
		\begin{enumerate}[a),leftmargin=*]
			\i Lần thứ 2 lấy được bi vàng, lần thứ 4 lấy được bi xanh.
			\i Có 3 kết quả khác nhau khi lấy bi là lấy được bi xanh, bi đỏ và bi vàng.
		\end{enumerate}
	
\end{Answer}
\begin{Answer}{6}
		\begin{enumerate}[a),leftmargin=*]
			\i Sự kiện A có khả năng xảy ra cao hơn. Vì từ 1 đến 10 có năm số là số chẵn đó là: 2, 4, 6, 8, 10 và có bốn số là số nguyên tố đó là: 2, 3, 5, 7.
			\i Xác suất thực nghiệm của sự kiện A là: $\frac{9}{{20}} = 0,45$
			
			Xác suất thực nghiệm của sự kiện B là:  $\frac{8}{{20}} = 0,4$
		\end{enumerate}
	
\end{Answer}
\begin{Answer}{7}
		\begin{enumerate}[a),leftmargin=*]
			\i Có 5 khách hàng không hài lòng trong số 20 khách hàng nên xác suất thực nghiệm là: $\frac{5}{{20}} = 0,25$
			\i Tổng số khách hàng không hài lòng sau hai tháng là: $5 + 3 = 8$ trong số 40 khách hàng nên xác suất thực nghiệm là:   $\frac{8}{{40}} = 0,2$
			
			Vì $0,2 < 0,25$ nên độ không hài lòng của khách giảm đi, tức độ hài lòng của khách tăng lên.
		\end{enumerate}
	
\end{Answer}

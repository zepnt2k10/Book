\begin{Answer}{102}
		\begin{enumerate}[a), leftmargin=*]
			\i Chu vi của hình bình hành là:  $2.(4+5)=18 \,(cm).$
			\i Diện tích hình thoi ABCD là:  $\frac{1}{2}.6.8=24\, (cm^2)$
			\i Chu vi hình chữ nhật ABCD là:  $2.(4+6)=5 \,(cm).$\\
			Diện tích hình chữ nhật ABCD là:  $4.6=24\,(cm^2).$
			\i Cạnh hình vuông IKMN là:  $16:4=4\,(dm).$\\
			Diện tích hình vuông IKMN là:  $4.4=16\,(dm^2).$
			\i Chu vi hình thoi MNPQ là:  $4.7=28\,(cm).$
		\end{enumerate}
	
\end{Answer}
\begin{Answer}{103}
		Đổi  $3m=300\,cm,9m=900\,cm.$
		
		Diện tích nền của căn phòng là:  $300\cdot900=270000\,(cm^2).$
		
		Diện tích viên gạch lát nền là:  $30\cdot30=900\,(cm^2)$
		
		Số viên gạch dùng để lát nền của că phòng hình chữ nhật là:  $270000:900=300$  (viên)
	
\end{Answer}
\begin{Answer}{104}
		Độ rộng của cổng vào là:  $12\cdot\frac{1}{4}=3 \,(cm).$
		
		Chu vi mảnh vườn hình chữ nhật là:  $2\cdot(12+8)=10 (m).$
		
		Hàng rào của khu vườn đó dài:  $10-3=7 \,(m).$
	
\end{Answer}
\begin{Answer}{105}
		Chiều dài của ngôi nhà là:  $8+6=14\,(m)$
		
		Chiều rộng của ngôi nhà là:  $6+2=8\,(m)$
		
		Diện tích mặt cắt ngôi nhà là:  $14\cdot8=112 \,(m^2).$
	
\end{Answer}
\begin{Answer}{106}
		Đổi  $3\, m=300 \,cm.$
		
		Cạnh của mảnh vườn hình vuông để trồng hoa là:  $300-2\cdot20=260\, (cm)$
		
		Diện tích mảnh vườn hình vuông để trồng hoa là:  $260\cdot260=67600 \,(cm^2).$
	
\end{Answer}
\begin{Answer}{107}
		Diện tích khu đất tam giác $ABC$ là:  $\frac{1}{2}.25.75=937,5\,(m^2).$
		
		Diện tích khu đất hình thang $ACDE$ là:  $\frac{1}{2}(75+25)\cdot37=1850 \,(m^2)$
		
		Diện tích cả khu đất $ABCDE$ là:  $937,5+1850=2787,5 \,(m^2)$
	
\end{Answer}
\begin{Answer}{108}
		\begin{enumerate}[a), leftmargin=*]
			\i Diện tích hình vuông sẽ tăng lên 4 lần.
			\i Diện tích hình chữ nhật sẽ tăng lên 6 lần.
			\i Diện tích hình chữ nhật sẽ tăng lên 2 lần.
		\end{enumerate}
	
\end{Answer}
\begin{Answer}{109}
		\begin{enumerate}[a), leftmargin=*]
			\i Cạnh hình vuông sẽ giảm đi 2 lần.
			\i Chiều dài hình chữ nhật tăng lên 4 lần.
			\i Chiều rộng của hình chữ nhật tăng lên 4 lần.
		\end{enumerate}
	
\end{Answer}
\begin{Answer}{110}
		Ta thấy tam giác đều $ABC$ có thể chia thành 9 hình tam giác nhỏ bằng nhau
		
		Diện tích của mỗi hình tam giác nhỏ là:  $36:9=4\,(m^2).$
		
		Diện tích của phần màu trắng là:  $4\cdot3=12\,(m^2).$
	
\end{Answer}
\begin{Answer}{111}
		Diện tích của hình vuông to là:  $4\cdot4=16 \,(cm^2).$
		
		Mỗi cạnh của hình vuông nhỏ là:  $4:4=1\,(cm).$
		
		Ta thấy phần được tô màu hồng chính là phần còn lại khi lấy diện tích cả hình vuông trừ 4 hình tam giác vuông màu trắng bằng nhau.
		
		Diện tích các tam giác vuông:  $\frac{1}{2}\cdot3\cdot1=\frac{3}{2}\, (cm^2).$
		
		Diện tích của phần được tô màu hồng là:  $16-\frac{3}{2}\cdot4=10\, (cm^2).$
	
\end{Answer}
\begin{Answer}{112}
		Do diện tích tăng thêm là  $37,5 \,cm^2$ nên ta suy ra diện tích $ACD$ là  $37,5 cm^2$. Gọi $AH$ là đường cao của tam giác $ACD$
		
		Độ dài $AH$ là:  $37,5\cdot2:5=15\,(cm).$
		
		Do tam giác $ABC$ và tam giác $ACD$ có chung đường cao $AH$ nên suy ra:
		
		Độ dài $BC$ là:  $150\cdot2:15=20 \,(cm).$
	
\end{Answer}
\begin{Answer}{113}
		Nối thêm các đường như hình vẽ
		
		Khi đó ta có:  ${{S}_{ABE}}={{S}_{HCD}}$  và  ${{S}_{AED}}={{S}_{BCF}}$
		
		Mà  ${{S}_{ABCD}}={{S}_{ABE}}+{{S}_{BEC}}+{{S}_{DEC}}+{{S}_{AED}}$
		Hay  ${{S}_{ABCD}}={{S}_{DCH}}+{{S}_{BEC}}+{{S}_{DEC}}+{{S}_{BCF}}=S$ phần màu xám.
		Vậy  ${{S}_{ABCD}}=1$.
	
\end{Answer}
\begin{Answer}{114}
		Chiều dài của khu vườn hình chữ nhật là:  $135:5=27\,(m)$
		
		Chiều rộng khu vườn hình chữ nhật là:  $27:3=9 \,(m)$
		
		Chu vi ban đầu của khu vườn đó là:  $2\cdot(9+27)=72\,(m)$
		
		Người ta dùng số cọc là:  $72:3=24$ (cọc)
	
\end{Answer}
\begin{Answer}{115}
		Chu vi hình vuông thứ nhất là:  $4\cdot{{A}_{1}}{{A}_{2}}$
		
		Chu vi hình vuông thứ hai là:  $4\cdot{{A}_{3}}{{A}_{4}}$
		
		Chu vi hình vuông thứ ba là:  $4\cdot{{A}_{5}}{{A}_{6}}$
		
		Chu vi hình vuông thứ tư là:  $4\cdot{{A}_{7}}{{A}_{8}}$
		
		Chu vi hinh vuông thứ năm là:  $4\cdot{{A}_{9}}{{A}_{10}}$
		
		Vậy tổng chu vi của tất cả các hình vuông trên là:
		$4\cdot({{A}_{1}}{{A}_{2}}+{{A}_{3}}{{A}_{4}}+{{A}_{5}}{{A}_{6}}+{{A}_{7}}{{A}_{8}}+{{A}_{9}}{{A}_{10}})=4\cdot AB=4\cdot 100=400 \,(cm)$
	
\end{Answer}
\begin{Answer}{116}
		\begin{enumerate}[a), leftmargin=*]
			\i Diện tích khu vực 16m50 là:
			\[165\cdot400=66000 \,(dm^2)=660 \,(m^2)\]
			Số tiền dự kiến để trồng lại cỏ là:
			\[660\cdot35000=231000000 \,\text{(đồng)}\]
			\ Chu vi của sân bóng là:
			\[2\cdot(1000+700)=3400 \,(dm)\]
			Để kẻ xong đường viền sân bóng cần ít nhất:
			\[3400:1000=3,4\approx 4\,  \text{(thùng sơn)}\]
		\end{enumerate}
	
\end{Answer}
\begin{Answer}{117}
		Diện tích phần tăng thêm là diện tích của tam giác có cạnh đáy  $15m$ và chiều cao là chiều cao của hình thang ban đầu.
		
		Chiều cao của hình thang ban đầu là:  $45\cdot2:15=6 \,(m).$
		
		Diện tích hình thang ban đầu là:  $\frac{1}{2}(20+40)\cdot6=180\, (m^2)$
	
\end{Answer}
\begin{Answer}{118}
		Xét tam giác $EBC$ và tam giác $ABC$ có chung đường cao kẻ từ đỉnh $B$, do $AE$ gấp $2$ lần $AC$ nên  $EC=\frac{1}{3}AC.$
		
		Vậy  ${{S}_{EBC}}=\frac{1}{3}{{S}_{ABC}}$
		
		Xét tam giác $BDC$ và tam giác $ABC$ có chung đường cao kẻ từ đỉnh $C$, do $AD$ gấp $2$ lần $BD$ nên  $BD=\frac{1}{3}AB.$
		
		Vậy  ${{S}_{BDC}}=\frac{1}{3}{{S}_{ABC}}$
		$\Rightarrow {{S}_{EBC}}={{S}_{BDC}}$.
		
		Mà  ${{S}_{EBC}}={{S}_{BGC}}+{{S}_{GEC}}$ và  ${{S}_{BCD}}={{S}_{BGC}}+{{S}_{GBD}}$
		
		Suy ra  ${{S}_{GEC}}={{S}_{GBD}}$  nên  $\frac{{{S}_{GBD}}}{{{S}_{GEC}}}=1$.
	
\end{Answer}
\begin{Answer}{119}
		Nối thêm các hình như hình vẽ ta được hình vẽ:
		
		
		Do $P,Q,R,S$ lần lượt là trung điểm của các cạnh $AB,BC,CD,AD$ nên  ${{S}_{1}}={{S}_{2}}={{S}_{3}}={{S}_{4}}$
		
		Mà  ${{S}_{1}}+{{S}_{2}}+{{S}_{3}}+{{S}_{4}}=\frac{{{S}_{DROS}}+{{S}_{RCQO}}+{{S}_{OQPB}}+{{S}_{SOPA}}}{2}=\frac{{{S}_{ABCD}}}{2}$
		Suy ra  ${{S}_{RQPS}}=\frac{{{S}_{ABCD}}}{2}$.
		
		Lại có $T$ là trung điểm của cạnh $RS$ nên  $ST=TR=\frac{SR}{2}$
		
		Nên  ${{S}_{STP}}=\frac{{{S}_{SPR}}}{2}=\frac{{{S}_{RQPS}}}{4}$  và  ${{S}_{QRT}}=\frac{{{S}_{SRQ}}}{2}=\frac{{{S}_{RQPS}}}{4}$.
		
		Vậy  ${{S}_{PTQ}}={{S}_{RQPS}}-\frac{{{S}_{RQPS}}}{4}-\frac{{{S}_{RQPS}}}{4}=\frac{{{S}_{RQPS}}}{2}=\frac{{{S}_{ABCD}}}{4}$.
	
\end{Answer}
\begin{Answer}{120}
		Mỗi chỗ để đỗ xe ô tô là một hình bình hành có chiều cao là một nửa chiều rộng của hình chữ nhật.
		
		Chiều cao của hình bình hành là:
		\[12:2=6 \,(m)\]
		Diện tích của một chỗ đỗ xe ô tô là:
		\[3\cdot6=18 \,(m^2)\]
		Diện tích 5 chỗ đỗ xe ô tô là:
		\[18\cdot5=90 \,m^2\]
	
\end{Answer}
\begin{Answer}{121}
		Xét tam giác $BAM$ và tam giác $BMO$ có chung đường cao từ đỉnh $B$ và $M$ nằm chính giữa $OA$ nên  $AM=MO$  $\Rightarrow {{S}_{BAM}}={{S}_{BMO}}$.
		
		Xét tam giác $BCN$ và tam giác $BNO$ có chung đường cao từ đỉnh $B$ và $N$ là điểm chính giữa $OC$ nên  $ON=NC\Rightarrow {{S}_{BNO}}={{S}_{BCN}}$.
		
		Tương tự ta cũng có:  ${{S}_{DAM}}={{S}_{DMO}},{{S}_{DCN}}={{S}_{DNO}}$.
		
		Mà  ${{S}_{BAM}}+{{S}_{BMO}}+{{S}_{BNO}}+{{S}_{BCN}}+{{S}_{DAM}}+{{S}_{DMO}}+{{S}_{DCN}}+{{S}_{DNO}}={{S}_{ABCD}}$
		
		$\Rightarrow 2({{S}_{BMO}}+{{S}_{BNO}}+{{S}_{DMO}}+{{S}_{DNO}})={{S}_{ABCD}}$\\
		$\Rightarrow 2{{S}_{BMDN}}={{S}_{ABCD}}$ \\
		$\Rightarrow {{S}_{BMDN}}=\frac{{{S}_{ABCD}}}{2}$\\
		
		Do $ABCD$ là hình vuông nên nó cũng là hình thoi
		
		$\Rightarrow {{S}_{ABCD}}=\frac{1}{2}\cdot20\cdot20=200\,(dm^2)$
		
		Vậy  ${{S}_{BMDN}}=\frac{1}{2}{{S}_{ABCD}}=100\,(dm^2)$
	
\end{Answer}

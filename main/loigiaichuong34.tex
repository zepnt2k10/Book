\begin{Answer}{58}
		\begin{enumerate}[a), leftmargin=*]
			\i Các hình là tam giác đều là: Hình 1, Hình 3.
			\begin{enumerate}[--, leftmargin=*]
				\i Dùng thước kẻ đo độ dài 3 cạnh của tam giác.
				\i Hình 1: Ba cạnh bằng nhau và bằng 4 cm.
				\i Hình 3: Ba cạnh bằng nhau và bằng 4,5 cm.
			\end{enumerate}
			\i Hình vuông là: Hình 1.
			\begin{enumerate}[--, leftmargin=*]
				\i Bốn cạnh bằng nhau.
				\i Bốn góc bằng nhau và bằng $90^\circ$.
				\i Hai đường chéo bằng nhau.
			\end{enumerate}
			\i Hình lục giác đều là: Hình 2.
			\begin{enumerate}[--, leftmargin=*]
				\i Sáu cạnh bằng nhau.
				\i Ba đường chéo chính bằng nhau.
			\end{enumerate}
		\end{enumerate}
	
\end{Answer}
\begin{Answer}{59}
		\textit{Cách 1.}
		\begin{enumerate}[Bước 1:, leftmargin=*]
			\i Vẽ đoạn thẳng $MN =4$ cm.
			\i Dùng eke có góc $60^\circ$ vẽ $NMx = 60^\circ$ và góc  $MNy = 90^\circ$.
			\i Tia $Mx$ và tia $Ny$ cắt nhau tại $O$.
			Ta được tam giác đều $OMN$ có.
			\begin{enumerate}[+, leftmargin=*]
				\i Các cạnh:  $OM = ON = MN = 4$ cm.
				\i Các góc:  $ONM, OMN, MON$.
				\i Các đỉnh: $M, N, O$.
			\end{enumerate}
		\end{enumerate}
		\textit{Cách 2.}
		\begin{enumerate}[Bước 1:, leftmargin=*]
			\i Vẽ đoạn thẳng  $MN = 4$ cm.
			\i Lấy $M$ làm tâm. Vẽ một phần đường tròn bán kính  $4$ cm.\\
			 Lấy $N$ làm tâm dùng compa vẽ 1 phần đường tròn $MN$.
			\i Gọi $O$ là giao điểm giữa 2 phần đường tròn vừa vẽ.
			\i Nối các đoạn $OM. ON$ ta được tam giác đều $OMN$.
		\end{enumerate}
	
\end{Answer}
\begin{Answer}{60}
		\begin{enumerate}[Bước 1:, leftmargin=*]
			\i Vẽ đoạn thẳng $EF = 5$ cm
			\i Vẽ đường thẳng vuông góc với $EF$  tại $E$. Xác định điểm $P$  trên đường thẳng đó sao cho $EP = 5$ cm.
			\i Vẽ đường thẳng vuông góc với $EF$  tại $F$.  Xác định điểm $P$  trên đường thẳng đó sao cho $FO = 3$ cm.
			\i Nối $O,P$  ta được hình vuông  $EFOP$ có:
			\begin{enumerate}[--, leftmargin=*]
				\i Các cạnh: $EF, FO, PO, EP$.
				\i Các góc: $EFO, FOP, OPE, PEF$.
				\i Các đỉnh: $E,F,O,P$.
			\end{enumerate}
		\end{enumerate}
	
\end{Answer}
\begin{Answer}{61}
		\textit{Cách 1:}
		\begin{enumerate}[Bước 1:, leftmargin=*]
			\i Vẽ đường tròn tâm $I$  bán kính $3$ cm.
			\i Trên đường tròn lấy điểm  bất kì. Vẽ đường tròn tâm $O$ bán kính $3$ cm cắt đường tròn tại $2$ điểm $P, F$.
			\i Đặt thước đi qua $P,I$ cắt đường tròn ban đầu tại $E$.\\
			Đặt thước đi qua$F,I$  cắt đường tròn ban đầu tại $N$.\\
			Đặt thước đi qua $O,I$ cắt đường tròn ban đầu tại $Q$.
		\end{enumerate}
		\textit{Cách 2.}
		
		Vẽ lần lượt các tam giác đều: $IOP, IPN, INQ, IQE, IEF, IFO$, có các cạnh bằng 3 cm trên cùng 1 hình vẽ. Nối $O,F$ ta được lục giác đều $OPNQEF$.
		
		Lục giác đều  có:
		\begin{enumerate}[--, leftmargin=*]
			\i Các đường chéo phụ: $ON, NE, EO, FP, PQ, QF$.
			\i Các đường chéo chính: $OQ, PE, FN$.
		\end{enumerate}
	
\end{Answer}
\begin{Answer}{62}
		Hình a) có 15 tam giác đều
		Hình b) có 6 tam giác đều.
	
\end{Answer}
\begin{Answer}{63}
		Hình vẽ có 2 lục giác đều và có 8 tam giác đều.
	
\end{Answer}
\begin{Answer}{64}
		\begin{enumerate}[a), leftmargin=*]
			\i Từ 6 tam giác bằng nhau ghé được 1 lục giác đều.
			
			\i Từ 4 tam giác đều ghép thành 1 tam giác đều.

		\end{enumerate}
	
\end{Answer}
\begin{Answer}{65}
		\textit{Các hình vuông cần tìm sẽ có những kích thước: $1\times 1, 2\times 2,\ldots, 5\times5$. Ta sẽ tính số hình vuông mỗi kích thước trên.}
		\begin{enumerate}[--, leftmargin=*]
			\i Hình vuông kích thước  $1\times 1$: Ta cho các hình vuông kích thước $1\times 1$  này di chuyển trong hình vuông lớn, đến các vị trí sao cho đỉnh của hình vuông nằm trên giao điểm của các đường lưới. Mỗi vị trí tương ứng với một hình vuông kich thước $ 1\times 1$.
			\i Ta thấy hình vuông di chuyển theo chiều ngang sẽ có 5 vị trí, theo chiều dọc cũng có 5 vị tri do đó số hình vuông kích thước $1\times 1$  là $5 \times 5 = 25$ (hình).
			\i Số hình vuông kích thước  $2 \times 2$: Tương tự như trên, hình vuông kích thước $2 \times 2$  di chuyển theo chiều ngang sẽ có 4 vị trí, chiều dọc cũng có 4 vị trí nên số hình vuông kích thước $2 \times 2$  là $4 \times 4 = 16$  (hình).
			\i Tương tự ta có:
			\begin{enumerate}[+, leftmargin=*]
				\i Số hình vuông có kích thước $3 \times 3$ là: $3\times 3 = 9$.
				\i Số hình vuông có kích thước $4 \times 4$ là: $2 \times 2 = 4$.
				\i Sô hình vuông có kích thước $5 \times 5$ là: $1 \times 1 = 1$.
			\end{enumerate}
		\end{enumerate}
		Vậy có tất cả số hình vuông là: $5^2 + 4^2 + 3^2 +2^2 + 1^2 = 55$ (hình vuông).
	
\end{Answer}
\begin{Answer}{66}
		\begin{enumerate}[a), leftmargin=*]
			\i Dùng 12 que diêm để xếp một hình vuông:
			
			\i Dùng 12 que diêm vẽ tam giác đều.
			
			\i Dùng 12 que diêm vẽ lục giác đều.
		\end{enumerate}
	
\end{Answer}
\begin{Answer}{67}
		Trong hình có 20 tam giác đều.
	
\end{Answer}
\begin{Answer}{68}
		Trong hình có 10 hình vuông.
	
\end{Answer}
\begin{Answer}{69}
		chèn ảnh
	
\end{Answer}
\begin{Answer}{70}
		chèn ảnh
	
\end{Answer}
\begin{Answer}{71}
		Quãng đường nhanh nhất đến kho báu mất:
		\[3 \times 3 + 3\times 2 = 13 \text{ (giây).}\]
	
\end{Answer}
\begin{Answer}{72}
		\begin{enumerate}[--, leftmargin=*]
			\i Số hình vuông kích thước $1\times 1$  là: $4 \times 5 = 20$.
			\i Số hình vuông kích thước  $2\times 2$: hình vuông kích thước $2\times 2$  di chuyển theo chiều ngang sẽ có $4$ vị trí, chiều dọc có $3$ vị trí nên số hình vuông kích thước $2\times 2$  là $3\times 4 = 12$  (hình).
			\i Số hình vuông có kích thước  $3\times 3$: hình vuông kích thước $3\times 3$  di chuyển theo chiều ngang sẽ có 3 vị trí, chiều dọc có 2 vị trí nên số hình vuông kích thước  $3\times 3$ là $2\times 3 = 6$ (hình)
			\i Số hình vuông có kích thước $4 \times 4$ là: $1\times 2 = 2$.
		\end{enumerate}
		Vậy có tất cả số hình vuông là: 40 (hình vuông)
	
\end{Answer}
\begin{Answer}{73}
		Số hình vuông được tạo thành là (không kể phần màu xám)
		\[{9^2} + {8^2} + {7^2} + {6^2} + {5^2} + {4^2} + {3^2} + {2^2} + {1^2} = 285 \text{ (hình).}\]
		Bây giờ ta đếm số hình vuông có một phần cạnh bị bỏ đi (tức một phần cạnh nằm trong phần màu xám), gọi chung là ``hình vuông xấu".
		\begin{enumerate}[--, leftmargin=*]
			\i Có $3\times3$  hình vuông $1\times 1$  ``xấu".
			\i Có $4\times4$  hình vuông $2\times 2$  ``xấu".
			\i Số hình vuông $3\times 3$  xấu là: $5\times 5 - 1\times 1 = 24$. (các đỉnh trên cùng bên phải của chúng được đánh dấu trong hình dưới bởi các chấm nhỏ màu đen trong hình).
		\end{enumerate}
		Tương tự ta có:
		\begin{enumerate}[--, leftmargin=*]
			\i Số hình vuông $4\times 4$  ``xấu" là: $5 \times 6 - 2 \times 2 = 26$.
			\i Số hình vuông $5 \times 5$  ``xấu" là: $5 \times 5 - 3\times 3 = 16$.
			\i Có 4 hình vuông $6 \times 6$  ``xấu"
			
			\i Các hình vuông $7\times7; 8\times 8; 9\times 9$ thì không có phần cạnh nào nằm ở phần màu xám nên số hình vuông ``xấu" là 0.
		\end{enumerate}
		Vì vậy số hình vuông được tạo thành là: $285 - \left( {9 + 16 + 24 + 26 + 16 + 4} \right) = 190$  (hình vuông).
	
\end{Answer}

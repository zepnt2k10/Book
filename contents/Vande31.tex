\def\i{\item}
\graphicspath{{../pictures/vande31/}}
\chapter{Chương 5}
\section{Ôn tập chương}
\subsection{Kiến thức cần nhớ}
\begin{tabular}{|c|p{2.2cm}|p{0.5\linewidth}|p{0.25\linewidth}|}
	\hline
	STT	&Nội dung& 	Kiến thức cần nhớ& 	Các dạng bài tập thường gặp\\
	\hline
	1&	Điểm Đường thẳng Tia& \begin{enumerate}[--,leftmargin=*]
		\i Điểm là dấu chấm nhỏ , kí hiệu bằng chữ in hoa ($A$, $B$, $C$, ...)
		\i Đường thẳng không bị giới hạn về hai phía, thường kí hiệu bằng chữ thường ($a$, $b$, $c$, ...)
		
		$M\in d;\,N\notin d$
		\i Tia là hình gồm điểm $O$ và nửa đường thẳng được chia bởi điểm $O$
		
		Tia $Ox,\,Oy$
		\i Hai tia chung gốc tạo thành một đường thẳng là hai tia đối nhau
		$Ox,\,Oy$là hai tia đối nhau
	\end{enumerate} &
	\begin{enumerate}[--,leftmargin=*]
		\i Vẽ hình theo yêu cầu 
		\i Xác định số đường thẳng, đoạn thẳng, tia , ...
	\end{enumerate}\\
	\hline
	2	&Ba điểm thẳng hàng& \begin{enumerate}[--,leftmargin=*]
		\i Ba điểm thẳng hàng là 3 điểm cùng thuộc 1 đường thẳng
		\i Trong ba điểm thẳng hàng có một và chỉ một điểm nằm giữa 2 điểm còn lại 
		\begin{enumerate}[+,leftmargin=*]
			\i Điểm $C$ nằm giữa $A$ và $B$
			\i $A$và $B$ nằm khác phía so với $C$ 
			\i $A$ và $C$ nằm cùng phía với $B$;
			\i $C$ và $B$ nằm cùng phía với $A$.
		\end{enumerate}
	\end{enumerate}&
	\begin{enumerate}[--,leftmargin=*]
		\i Nhận xét ba điểm thẳng hàng 
		\i Vẽ hình theo yêu cầu 
		\i Bài toán trồng cây
	\end{enumerate}\\
	\hline
	3&	Vị trí tương đối giữa hai đường thẳng&
	\begin{enumerate}[--,leftmargin=*]
		\i Hai đường thẳng $a,\,b$ có các trường hợp 
		\begin{enumerate}[+,leftmargin=*]
			\i $a \parallel b$: $a$ và $b$ không có điểm chung
			\i $a$ cắt $b$: $a$ và $b$  có 1 điểm chung
			\i $a$$\underset{\scriptscriptstyle-}{=}$$b$: $a$ và $b$ có vô số điểm chung.
		\end{enumerate}
		Chú ý: hai đường thẳng phân biệt có thể song song hoặc cắt nhau.
	\end{enumerate}&
	\begin{enumerate}[--,leftmargin=*]
		\i Nhận biết đường thẳng song song, cắt nhau, trùng nhau
		\i Vẽ hình theo yêu cầu
	\end{enumerate}\\
	4	&Độ dài đoạn thẳng 
	Trung điểm của đoạn thẳng&
	\begin{enumerate}[--,leftmargin=*]
		\i Đoạn thẳng $AB$ là hình giữa hai điểm $A,\,B$ và các điểm nằm giữa $A$, $B$, $A$ và $B$ là hai đầu mút 
		\i Nếu $M$ nằm giữa $A$ và $B$ thì $AM+MB=AB$
		\i Nếu $I$ nằm giữa $A$ và $B$ , và $IA=IB$ thì $I$ là trung điểm của $AB$
	\end{enumerate}&
	\begin{enumerate}[--,leftmargin=*]
		\i Nhận biết đoạn thẳng 
		\i Tính, so sánh độ dài đoạn thẳng 
		\i Chứng minh trung điểm đoạn thẳng
	\end{enumerate}\\
	 \hline
	5&	Góc - Số đo góc	&
	\begin{enumerate}[--,leftmargin=*]
		\i Góc là hình giữa hai tia chung gốc.
		Gốc chung giữa 2 tia là đỉnh của góc Hai tia là hai cạnh của góc. 
		Kí hiệu : $\widehat{xOy}$
		\i Quy tắc cộng góc: Nếu điểm $M$ nằm trong góc $\widehat{xOy}$ thì $\widehat{xOM}+\widehat{MOy}=\widehat{xOy}$
		\i Góc vuông  
		
		\i Góc nhọn
		
		\i Góc tù
		
		\i Góc bẹt  
	\end{enumerate} & 
	\begin{enumerate}[--,leftmargin=*]
		\i Đọc tên góc, viết kí hiệu 
		\i Đo góc 
		\i Vẽ góc
		\i Tính số đo góc
	\end{enumerate}
\end{tabular}

%B. Bài tập tự luyện 
%Bài 1. Em hày hoàn thành 10 câu trắc nghiệm sau 
%1) Trong các câu sau , câu nào đúng 
%A. Hai tia chung gốc thì đối nhau
%B. Hai tia chung gốc cùng nằm trên 1 đường thẳng thì đối nhau 
%C. Hai tia chung gốc tạo thành 1 đường thẳng thì đối nhau
%D. Hai tia chung gốc tạo thành một nửa đường thẳng thì đối nhau
%2) Ba đường thẳng A, B, C thẳng hàng khi 
%A. A, B, C thuộc ba đường thẳng phân biệt
%B. A, B, C thuộc ba đưởng thẳng song song 
%C. A, B, C thuộc cùng một đường thẳng bất kì 
%D. Cả 3 đáp án trên đều đúng
%3) Khẳng định nào sai 
%A. Góc nhọn nhỏ hơn góc vuông 
%B. Góc tù nhỏ hơn góc nhọn 
%C. Góc bẹt nhỏ hơn góc vuông
%D. Góc vuông nhỏ hơn góc tù 
%4) Đo góc $\widehat{xOy}$
%
%A. $45{}^\circ $	B. $30{}^\circ $	C. $50{}^\circ $	D. $40{}^\circ $
%
%5) Có bao nhiêu bộ ba điểm thẳng hàng trong hình sau 
%
%A. 10	B. 11	C. 12	D. 13
%6) Trong hình vẽ dưới đây , số đường thẳng đi qua D và không đi qua E là
%
%A. 4	B. 3	C. 2	D. 1
%7) Có bao nhiêu cặp đường thẳng song song trong hình vẽ
%
%A. 6	B. 5	C. 4	D. 7
%8) Có bao nhiêu bộ ba điểm thẳng hàng trong hình vẽ 
%
%A. 2	B. 4	C. 5	D. 3
%9) Nếu điểm B nằm trong góc $\widehat{xOy}$ thì
%A. $\widehat{xOB}+\widehat{xOy}=\widehat{yOB}$		B. $\widehat{xOB}-\widehat{yOB}=\widehat{xOy}$
%C. $\widehat{xOB}+\widehat{yOB}=\widehat{xOy}$		D. $\widehat{yOB}+\widehat{xOy}=\widehat{xOB}$
%10) Ta có thể xem kim giờ và kim phút của đồng hồ là hai tia chung gốc ( gốc trùng với trục quay của hai kim ). Tại mỗi thởi điểm, hai kim tạo thành một góc. Quan sát các đồng hồ sau và sắp xếp các hình đồng hồ theo thứ tự giảm dần số đo của góc tạo bởi kim giờ và kim phút.
%Hình 1		Hình 2	Hình 3	Hình 4
%A. Hình 1, hình 2, hình 3, hình4
%B. Hình 1, hình 2, hình 4, hình 3
%C. Hình 3, hình 2, hình 4, hình 1  
%D. Hình 3, hình 4, hình 1, hình 2
%Bài 2. Ghép mỗi ý ở cột Hình hình học với hình vẽ tương ứng ở cột Hình vẽ
%Hình vẽ
%A)    
%B)        
%C)       
%D) 
%E)
%F) 
%G) 
%H)    
%L)      
%M)      
%N) 
%
%
%Hình hình học
%(1) Điểm $A$
%(2) Đường thẳng đi qua hai điểm $A$và $B$
%(3) Đoạn thẳng $MN$
%(4) Tia $At$
%(5) Điểm nằm trên đường thẳng  
%(6) Hai đường thẳng cắt nhau
%(7) Điểm nằm ngoài đường thẳng 
%(8) Hai đường thẳng song song
%(9) Ba điểm không thẳng hàng 
%(10) Ba điểm thẳng hàng
%(11) Đoạn thẳng $AB$ có độ dài bằng $3cm$
%(12) Điểm nằm $M$giữa hai điểm $C$và $D$
%
%
%
%
%
%
%
%
%Bài 3. Cho hình vẽ 
%
%a) Kể tên các tia đối nhau 
%b) Kể tên các cặp đường thẳng cắt nhau (các cặp đường thẳng trùng nhau chỉ tính 1 lần)
%c) Kể tên các bộ ba điểm thẳng hàng
%Bài 4. Cho hình chữ nhật \[ABCD\] và các điểm $M,N,H$ như hình vẽ. 
%a) Kể tên các góc đỉnh $M$
%b) Đo các góc $\widehat{ANM};\,\widehat{NMD};\,\widehat{DMC};\,\widehat{ADM}$
%c) Kể tên các góc vuông trong hình vẽ 
%d) Kể tên các góc nhọn trong hình vẽ.
%Bài 5. Em hãy tìm ít nhất 3 biển báo giao thông (ghi rõ tên biển báo) có hình ảnh 2 đường thẳng song song, 3 biển báo có hình ảnh 2 đường thẳng cắt nhau.
%Bài 6. Vẽ hình theo diễn đạt (mỗi hình 1 ý)
%a) Vẽ hai tia $Ox,\,Oy$ phân biệt và không đối nhau
%b) Vẽ đường thẳng $d$// ${d}'$, đường thẳng $c$cắt $d$ và  ${d}'$ lần lượt tại M và N
%c) Vẽ góc $\widehat{mAt}$$=60{}^\circ $, lấy điểm P, Q nằm trong góc $\widehat{mAt}$ sao cho $A,\,P,\,Q$ không thẳng hàng. Đo góc $\widehat{PAm};\,\widehat{tAQ}$
%Bài 7. Trên tia $Ox$ lấy hai điểm $A$ và $B$ sao cho $OA=3cm$, $OB=6cm$
%a) So sánh $OA$ và $OB$
%b) Tính độ dài $AB$ 
%c) Điểm $A$ có là trung điểm của $OB$ không? Vì sao? 
%Bài 8. Trên tia $Ox$ lấy $A$ và $B$ sao cho $OA=8cm$, $OB=4cm$ 
%a) Tính độ dài $AB$
%b) Trên tia đối của tia $Ox$ lấy điểm $C$ sao cho $OC=4cm$. Tính $AC,\,BC$
%c) $O$ có là trung điểm của $BC$ không? Vì sao?
%Bài 9. Vẽ đoạn thẳng $AB=10cm$. Gọi $I$là trung điểm của $AB$. Trên đoạn thẳng $AB$lấy $M,\,N$sao cho $AM=BN=2cm$. Chứng minh $I$ là trung điểm của $MN$
%
%Mức độ nâng cao 
%Bài 10. Vẽ đoạn thẳng $MN=6cm$. Trên tia $MN$lấy điểm $O$ sao cho$NO=2cm$. Tính $OM$
%Bài 11. Cho đoạn thẳng $AB$ và $M$ là trung điểm của nó. Gọi $C$là điểm nằm giữa $M$ và $B$. Hãy chứng tỏ rằng $CM=\frac{CA-CB}{2}$
%Bài 12: Vòng quay mặt trời trong khu vui chơi có đường kính là $66m,$ chiều cao của trục vòng quay so với mặt đất là $43m.$Hỏi điểm cao nhất và điểm thấp nhất của vòng quay nằm ở độ cao nào so với mặt đất?
%
%Bài 13: Cho $2022$ đường thẳng cắt nhau từng đôi một. Hỏi có nhiều nhất bao nhiêu giao điểm được tạo thành?
%Bài 14: Vẽ góc $\widehat{xOy}={{60}^{0}}.$ Vẽ tia \[Oz\] sao cho $\widehat{xOz}={{30}^{0}}.$ Tính góc $\widehat{y\text{O}z}?$
%Bài 15: Cho $1998$ tia gốc $O.$ Sau khi vẽ thêm hai tia đi qua gốc $O.$ Số đo tăng  thêm tại đỉnh $O$ là bao nhiêu?
%Bài 16: Cho $n$ điểm không có ba điểm nào thẳng hàng. Nếu ta vẽ thêm hai điểm (không tạo ra ba điểm thẳng hàng) thì số đường thẳng nối hai trong số các điểm đó tăng lên $13$ đường thẳng. Tìm $n.$
%Bài 17: Cho điểm $M$nằm giữa hai điểm $A$ và $B.$ Điểm $I$là trung điểm của đoạn thẳng $AB$ và $5AB=8AM.$ Biết $MI=2cm.$ Tính $AB.$ 
%C. Hướng dẫn giải và đáp số
%Bài 1. 
%1) C	2) C	3) B	4) A	5) A
%6) D	7) A	8) D	9) C	10) D
%Bài 2. 
%$\left( 1 \right)-B$	$\left( 4 \right)-E$	$\left( 7 \right)-$trống	$\left( 10 \right)-H$
%$\left( 2 \right)-C$	$\left( 5 \right)-G$	$\left( 8 \right)-F$	$\left( 11 \right)-M$
%$\left( 3 \right)-A$	$\left( 6 \right)-D$	$\left( 9 \right)-N$	$\left( 12 \right)-L$
%Bài 3. 
%a) Các tia đối nhau là: $Ax$và $Ay$; $Cx$ và $Cy$; $Dx$ và $Dy$
%b) Các đường thẳng cắt nhau là: $AO$và $BA$; $BE$và $BA$
%c) Các bộ ba điểm thẳng hàng là: $A,C,D$; $B,F,A$
%Bài 4. 
%a) Các góc có đỉnh $M$ là: $\widehat{BMC};\widehat{NMH};\widehat{BMN};\widehat{HMC}$
%b) $\widehat{ANM}=120{}^\circ ;\widehat{DMC}=35{}^\circ ;\widehat{NMD}=120{}^\circ ;\widehat{ADM}=35{}^\circ $
%c) Các góc vuông trong hình vẽ là: $\widehat{BA\text{D}};\widehat{A\text{D}C};\widehat{ABC};\widehat{BC\text{D}}$
%d) Các góc nhọn trong hình vẽ là: $\widehat{BMN};\widehat{HMC};\widehat{MCH};\widehat{HCD};\widehat{CDH};\widehat{DHC};\widehat{ADM}$
%Bài 5. 
%- Các biển báo có 2 đường thẳng song song là: Nhường đường cho xe ngược chiều qua đường hẹp; Đường hai chiều; Giao nhau với đường hai chiều
%- Các biển báo có 2 đường thẳng cắt nhau là: Cấm rẽ trái; cấm rẽ phải; cấm đỗ xe ngang lề
%
%Bài 6. 
%a)                                                 b)                                               c)
%
%$\widehat{PAm}=40{}^\circ ;\widehat{QAt}=35{}^\circ $
%Bài 7. 
%
%Vì $A$ nằm giữa $O,B$ nên ta có: $OA<OB$
%Lại có: 
%$OA+AB=OB\Rightarrow AB=OB-OA=6-3=3\left( cm \right)$
%Vì $A$ nằm giữa $O,B$ và $OA=AB=\frac{OB}{2}=3\left( cm \right)$ nên $A$ là trung điểm của $AB$
%Bài 8. 
%
%a) Vì $B$ nằm giữa $O,A$ nên ta có: $OB+AB=OA$
%$\Rightarrow AB=OA-OB=8-4=4\left( cm \right)$
%b) Vì $O$ nằm giữa $A,C$ nên ta có: $AC=OA+OC=8+4=12\left( cm \right)$
%Vì $O$ nằm giữa $B,C$ nên ta có: $BC=OB+OC=4+4=8\left( cm \right)$
%Vì $O$ nằm giữa $B,C$ và $OB=OC=\frac{BC}{2}=4\left( cm \right)$ nên $O$ là trung điểm của $BC$
%Bài 9. 
%
%Vì $I$ là trung điểm của $AB$ nên ta có: $AI=IB=\frac{AB}{2}=\frac{10}{2}=5\left( cm \right)$
%Vì $M$ nằm giữa $A,I$ nên ta có: $AI=AM+IM$
%$\Rightarrow IM=AI-AM=5-2=3\left( cm \right)$
%Vì $N$ nằm giữa $B,I$ nên ta có: $BI=BN+IN$
%$\Rightarrow IN=BI-BN=5-2=3\left( cm \right)$
%Vì $I$ nằm giữa $M,N$và $IM=IN=3\left( cm \right)$ nên $I$ là trung điểm của $MN$.
%Bài 10. 
%TH1: $O$ nằm giữa $M,N$
%
%Ta có: $OM+ON=MN\Rightarrow OM=MN-ON=6-2=4\left( cm \right)$
%TH2: $N$ nằm giữa $M,O$
%
%Ta có: $OM=MN+NO=6+2=8\left( cm \right)$
%Bài 11. 
%
%Vì $M$ là trung điểm của $AB$ nên ta có: $MA=MB=\frac{AB}{2}$
%Mặt khác, $C$ nằm giữa $AB$ nên $AB=CA+CB$
%$\Rightarrow MB=MA=\frac{AB}{2}=\frac{CA+CB}{2}$
%Vì $C$ nằm giữa $M,B$ nên ta có: $MC+CB=MB$
%$\Rightarrow MC=MB-CB$
%Hay $MC=\frac{CA+CB}{2}-CB=\frac{CA+CB}{2}-\frac{2CB}{2}=\frac{CA-CB}{2}$
%Bài 12. 
%Gọi điểm cao nhất trục  và điểm thấp nhất và điểm vuông góc kẻ từ trục đến mặt đất lần lượt là $A,B,C,D$($A,B,C,D$ không thẳng hàng) như hình vẽ. 
%
%Khi đó: $AC=66m,\,BD=43m.$
%- Vì $B$ là trục quay của vòng quay mặt trời nên $B$ là trung điểm của $AC.$
%$\Rightarrow AB=BC=\frac{AC}{2}=\frac{66}{2}=33m.$
%- Vì $C$nằm giữa $A$ và $B$ nên ta có: 
%$\begin{align}
%	Bài 13.   & BC+CD=BD \\ 
%	Bài 14.  & 33+\,CD=43 \\ 
%	Bài 15.  & CD=43-33 \\ 
%	Bài 16.  & CD=10m. \\ 
%	Bài 17. \end{align}$
%- Vì $C$nằm giữa $A$ và $D$ nên ta có:
%$\begin{align}
%	Bài 18.   & AC+CD=AD \\ 
%	Bài 19.  & 66+\,10=AD \\ 
%	Bài 20.  & AD=76m. \\ 
%	Bài 21.  & CD=10m. \\ 
%	Bài 22. \end{align}$
%Vậy khoảng cách từ điểm cao nhất của vòng quay so với mặt đất là $76m.$
%Khoảng cách từ điểm tháp nhất của vòng quay so với mặt đất là $10m.$
%Bài 23. 
%Ta có $1$ đường thẳng bất kì tạo với $2021$đường còn lại $2021$ giao điểm.
%Có $2022$đường như vậy nên ta có: $2021.2022$ giao điểm.
%Nhưng mỗi giao điểm dược tính hai lần nên thực tế số giao điểm là:$\frac{2021.2022}{2}=2023011.$
%Bài 24. 
%
%$T{{H}_{1}}:$ $Oz$ nằm giữa \[Ox\] và $Oy.$
%Ta có hình vẽ:
%
%Vì $Ot$ nằm giữa \[Ox\]và $Oy$ nên
%$\begin{align}
%	& \widehat{xOz}+\widehat{zOy}=\widehat{xOy} \\ 
%	& {{30}^{0}}+\widehat{zOy}={{60}^{0}} \\ 
%	& \widehat{zOy}={{60}^{0}}-{{30}^{0}} \\ 
%	& \widehat{zOy}={{30}^{0}} \\ 
%\end{align}$
%$T{{H}_{2}}:$ $Oz$ không nằm giữa \[Ox\] và $Oy.$
%
%
%Vì $Ox$ nằm giữa \[Oz\]và $Oy$ nên
%$\begin{align}
%	& \widehat{xOz}+\widehat{xOy}=\widehat{yOz} \\ 
%	& {{30}^{0}}+{{60}^{0}}=\widehat{yOz} \\ 
%	& \widehat{yOz}={{90}^{0}} \\ 
%\end{align}$
%Bài 25. 
%Cứ $1$ tia gốc $O$ bất kì tạo với $1997$ tia gốc $O$ còn lại $1997$góc.
%Có $1998$ tia gốc $O$ như thế nên ta có $1998.1997$ góc tạo thành.
%Nhưng mỗi góc được tính $2$ lần nên thực tế số góc tạo thành là: $\frac{1998.1997}{2}=1995003$(góc)
%Nếu thêm $2$ tia gốc $O$ thì khi đó số góc tạo thành là: $\frac{2000.1999}{2}=1999000$
%Vậy số góc tăng thêm là: $1999000-1995003=3997$(góc).
%Bài 26. 
%Cứ $1$ điểm bất kì ta nói với tất cả các điểm còn lại, khi đó số đường thẳng được nối là: $n.\left( n-1 \right)$
%Nhưng mỗi đường thẳng được nối hai lần nên thực tế số đường thẳng được tạo thành là:$\frac{n.\left( n-1 \right)}{2}$
%Sau khi vẽ thêm hai điểm (không tạo ra ba điểm nào thẳng hàng) thì số đường thẳng tạo thành là: $\frac{\left( n+2 \right)\left( n+1 \right)}{2}$
%Vì số đường thẳng tạo thành sau khi thêm hai điểm tăng lên 13 đường thẳng nên ta có:
%$\begin{align}
%	Bài 27.   & \frac{\left( n+2 \right)\left( n+1 \right)}{2}-\frac{n.\left( n-1 \right)}{2}=13 \\ 
%	Bài 28.  & \frac{{{n}^{2}}+n+2n+2-{{n}^{2}}+n}{2}=13 \\ 
%	Bài 29.  & 4n+2=26 \\ 
%	Bài 30.  & 4n=24 \\ 
%	Bài 31.  & n=6. \\ 
%	Bài 32. \end{align}$
%Vậy $n=6.$
%Bài 33. 
%
%Ta có: $I$ là trung điểm của $AB\Rightarrow BI=\frac{1}{2}AB$
%Mà $5.AB=8.BM\Rightarrow BM=\frac{5}{8}AB$
%Vì $\frac{1}{2}<\frac{5}{8}$ nên $BI<BM$ nên $I$ nằm giữa $B$ và $M.$
%Do đó: $BI+MI=MB$
%hay $\frac{1}{2}AB+2=\frac{5}{8}AB$
%$\begin{align}
%	Bài 34.   & \frac{5}{8}AB-\frac{1}{2}AB=2 \\ 
%	Bài 35.  & \left( \frac{5}{8}-\frac{1}{2} \right)AB=2 \\ 
%	Bài 36.  & \frac{1}{8}AB=2 \\ 
%	Bài 37.  & AB=16cm. \\ 
%	Bài 38. \end{align}$

\def\i{\item}
\graphicspath{{../pictures/vande38/}}
\chapter{HÌNH HỌC TRỰC QUAN}
\section{Ôn tập chương 3} 
\subsection{KIẾN THỨC CẦN NHỚ}
\subsubsection{Chu vi, diện tích, tích chất một số hình}
\begin{tabular}{|p{0.3\textwidth}|p{0.65\textwidth}|}
	\hline
	&Tính chất\\
	\hline
	Tam giác đều&	\begin{enumerate}[--, leftmargin=*]
		\i Ba cạnh bằng nhau
		\i Ba góc bằng nhau
	\end{enumerate}\\
	\hline
	Lục giác đều&	\begin{enumerate}[--, leftmargin=*]
		\i Sáu cạnh bằng nhau
		\i Sáu góc bằng nhau, mỗi góc bằng $120^\circ$
		\i Ba đường chéo chính bằng nhau
	\end{enumerate}\\
	\hline
	Hình vuông
	Chu vi: $4a$; diện tích: $a^2$	&\begin{enumerate}[--, leftmargin=*]
		\i Bốn cạnh bằng nhau
		\i Bốn góc bằng nhau, mỗi góc bằng ${{90}^{0}}$
		\i Hai đường chéo bằng nhau
	\end{enumerate}\\
	\hline
	Hình chữ nhật
	Chu vi: $\left( a+b \right)\times 2$; diện tích: $a.b$& \begin{enumerate}[--, leftmargin=*]
		\i Bốn góc bằng nhau và bằng ${{90}^{0}}$
		\i Các cạnh đối bằng nhau
		\i Hai đường chéo bằng nhau
	\end{enumerate}\\	
	\hline
	Hình thoi
	Chu vi: $4a$; diện tích: $\dfrac{1}{2}{{d}_{1}}.{{d}_{2}}$&\begin{enumerate}[--, leftmargin=*]
		\i Bốn cạnh bằng nhau
		\i Hai đường chéo vuông góc với nhau
		\i Các cạnh đối song song với nhau
		\i Các góc đối bằng nhau.
	\end{enumerate}\\	
	\hline
	Hình bình hành
	Chu vi: $\left( a+b \right)\times 2$; diện tích: $a.h$&	
	\begin{enumerate}[--, leftmargin=*]
		\i Các cạnh đối bằng nhau
		\i Hai đường chéo cắt nhau tại trung điểm của mỗi đường
		\i Các cạnh đối song song với nhau
		\i Các góc đối bằng nhau
	\end{enumerate}\\
	\hline
	Hình thang cân
	Diện tích: $\dfrac{1}{2}\left( a+c \right).h$&
	\begin{enumerate}[--, leftmargin=*]
		\i Hai cạnh bên bằng nhau
		\i Hai đường chéo bằng nhau
		\i Hai cạnh đáy song song với nhau
		\i Hai góc kề một đáy bằng nhau.
	\end{enumerate}	\\
	\hline
\end{tabular}
\subsubsection{Hình có trục đối xứng, tâm đối xứng}

\begin{tabular}{|p{0.22\textwidth}|p{0.35\textwidth}|p{0.35\textwidth}|}
	\hline
	&Đối xứng trục&Đối xứng tâm\\
	\hline
	Hình có tính đối xứng&	 
	Gấp hình theo trục $d$ thì hai phần của hình trùng nhau.&	 
	Quay hình nửa vòng quanh tâm $O$ thì hai phần của hình trùng nhau.\\
	\hline
	Các hình đã học có tính đối xứng& \begin{enumerate}[--, leftmargin=*]
		\i Hình thang cân
		\i Hình chữ nhật
		\i Hình thoi
		\i Hình vuông
		\i Tam giác đều
		\i Lục giác đều
		\i Đường tròn
	\end{enumerate}&
	\begin{enumerate}[--, leftmargin=*]
		\i Hình chữ nhật
		\i Hình bình hành
		\i Hình thoi
		\i Hình vuông
		\i Lục giác đều
		\i Đường tròn
	\end{enumerate}\\
	\hline
\end{tabular}
\subsection{BÀI TẬP TỰ LUYỆN}
\Opensolutionfile{loigiaichung}[loigiaichuong38]
\subsubsection*{Mức độ cơ bản}
\begin{bt}
	Trong các hình dưới đây, hình nào là hình tam giác đều, hình lục giác đều, hình vuông?
	\begin{loigiaichuong38}
		Trống
	\end{loigiaichuong38}
\end{bt}
\begin{bt}
	Hình nào là hình chữ nhật, hình bình hành, hình thoi, hình thang cân trong các hình dưới đây? 
	\begin{loigiaichuong38}
		Trống
	\end{loigiaichuong38}
\end{bt} 
\begin{bt}
	Trong các hình dưới đây, hình nào có trục đối xứng? Hình nào có tâm đối xứng?
	\begin{loigiaichuong38}
		Trống
	\end{loigiaichuong38}
\end{bt}
\begin{bt}
	Trong các hình tam giác đều, hình tròn, hình lục giác đều, hình vuông, hình chữ nhật, hình thoi, hình bình hành, hình thang cân, kể tên các hình:
	\begin{enumerate}[a), leftmargin=*]
		\i Chỉ có tâm đối xứng.
		\i Chỉ có trục đối xứng.
		\i Có cả trục đối xứng và tâm đối xứng.
	\end{enumerate}
	Em hãy vẽ cả trục đối xứng và tâm đối xứng của hình có ở câu c).
	\begin{loigiaichuong38}
		Trống
	\end{loigiaichuong38}
\end{bt}
\begin{bt}
	Cho hình vẽ, em hãy:
	
	\begin{enumerate}[a), leftmargin=*]
		\i Tính diện tích của các hình từ (1) đến (9)
		\i Tính chu vi của $ABCD$ biết mỗi ô là một hình vuông cạnh $1\, cm$.
	\end{enumerate}
	\begin{loigiaichuong38}
		Trống
	\end{loigiaichuong38}
\end{bt}
\begin{bt}
	Một hình vuông có chu vi $60\, m$ được cắt thành ba hình chữ nhật như nhau. Tính chu vi mỗi hình chữ nhật đó.
	\begin{loigiaichuong38}
		Trống
	\end{loigiaichuong38}
\end{bt}
\begin{bt}
	Người ta muốn lát gạch sân nhà hình chữ nhật của căn phòng có chiều dài là $5\, m$; chiều rộng là $4\, m$. Viên gạch lát có dạng hình vuông cạnh $80\, cm$. Hỏi cần bao nhiêu viên gạch để lát hết sàn nhà đó? (Mạch vữa không đáng kể).
	\begin{loigiaichuong38}
		Đổi: $80\,cm=0,8\,m$
		
		Diện tích sàn nhà hình chữ nhật là: 
		$4\,.\,5=20\,(m^2)$
		
		Diện tích của viên gạch lát hình vuông là: 
		$0,8\,.\,0,8=0,64\,(m^2)$
		
		Vậy để lát hết sàn nhà đó cần số viên gạch là: 
		$20:0,64=31,25\approx 32$ (viên gạch)
	\end{loigiaichuong38}
\end{bt}
\begin{bt}
	Biển báo ``Đường một chiều" được người ta cho kích thước như hình vẽ. Em hãy tính diện tích hình mũi tên trắng và phần tô màu xanh.
	\begin{loigiaichuong38}
		Để tính diện tích mũi tên ta chia mũi tên thành hai phần gồm hình tam giác và hình chữ nhật (như hình vẽ).
		
		Diện tích hình chữ nhật nhỏ màu trắng là: $30\,.\,10=300\,(cm^2)$
		
		Cạnh đáy của tam giác màu trắng là: $10+10+10=30\,(cm)$
		
		Đường cao của tam giác màu trắng là: $45-30=15\,(cm)$
		
		Diện tích tam giác màu trắng là: $\dfrac{1}{2}\,.\,30\,.\,15=225\,(cm^2)$
		
		Diện tích hình mũi tên là: $300+225=525\,(cm^2)$
		
		Diện tích hình chữ nhật lớn là: $50\,.\,40=2000\,(cm^2)$
		
		Diện tích phần tô màu xanh là: $2000-525=1475\,(cm^2)$
	\end{loigiaichuong38}
\end{bt}
\begin{bt}
	Cho hai hình vuông $ABCD$ và $BFGH$. Biết hiệu diện tích hai hình vuông bằng $120\,m^2$ và tổng hai cạnh của hai hình vuông $12\, m$. 
	\begin{enumerate}[a), leftmargin=*]
		\i Các hình $AHGD$; $CFGD$ là hình gì?
		\i Tính $AH,CF$.
		\i Tính cạnh mỗi hình vuông.
	\end{enumerate}
	\begin{loigiaichuong38}
		\begin{enumerate}[a), leftmargin=*]
			\i Các hình $AHGD;CFGD$là những hình thang vuông.
			\i Hiệu diện tích giữa hai hình vuông chính là diện tích của hai hình thang vuông bằng nhau $AHGD$; $CFGD$\\
			Vậy ${{S}_{AHGD}}={{S}_{CFGD}}=\dfrac{120}{2}=60\,\,(cm^2)$\\
			Ta lại có: ${{S}_{AHGD}}=\dfrac{(AD+HG)\,.\,AH}{2}=60\,(cm^2)$\\
			$\Rightarrow \dfrac{12}{2}\,.\,AH=60$ \\ 
			$\Rightarrow AH=10\,\,(cm)$ \\ 
			$AHGD;CFGD$ là hai hình thang vuông bằng nhau nên $AH=CF=10\,(cm)$
			\i Cạnh hình vuông $ABCD$là: $\dfrac{10+12}{2}=11\,(cm)$.\\
			Cạnh hình vuông $BFGH$là: $12-11=1\,(cm)$.
		\end{enumerate}
	\end{loigiaichuong38}
\end{bt}
\begin{bt}
	Trong một khu vườn hình chữ nhật, người ta làm hai lối đi lát sỏi để tiện chăm sóc theo kích thước như hình vẽ. Chi phí mỗi mét vuông làm lối đi hết 110 nghìn đồng. Hỏi chi phí để làm hết lối đi là bao nhiêu?
	\begin{loigiaichuong38}
		Để tính diện tích lối đi ta chia hình như sau: 
		
		Hình bình hành $1$ là hình có cạnh đáy $2m$ và chiều cao $20\,m$.
		
		Diện tích hình bình hành 1 là: $20.2=40\,(m^2)$
		
		Diện tích hình thang vuông 2 là: $\dfrac{(36+38).2}{2}=74\,(m^2)$
		
		Diện tích lối đi là: $40+74=114\,\,(m^2)$
		
		Chi phí để làm lối đi hết: $114.110000=12540000$ (đồng)
	\end{loigiaichuong38}
\end{bt}
\subsubsection*{Mức độ nâng cao} 
\begin{bt}
	An xếp một lưới tam giác từ các que tính bằng nhau như hình dưới đây. Cần gỡ ít nhất bao nhiêu que tính để còn lại hình chỉ gồm 6 tam giác nhỏ?
	\begin{loigiaichuong38}
		Cần bỏ ít nhất 4 que tính để còn lại hình gồm 6 tam giác nhỏ.
		Ta có hình như sau:
		
	\end{loigiaichuong38}
\end{bt}
\begin{bt}
	Trong hình bên dưới, một hình lục giác được tạo thành từ 96 hình tam giác đều bằng nhau với diện tích mỗi hình bằng 1 (đvdt). Hãy tính diện tích hình tam giác được tô đậm.
	\begin{loigiaichuong38}
		Ta chia hình như sau: 
		
		Nhìn một cạnh của tam giác cần tính là đường chéo của 1 hình bình hành gồm 20 tam giác nhỏ bằng nhau.
		
		$\Rightarrow$ Mỗi hình bình hành như trên có diện tích bằng 20 (đvdt).
		
		$\Rightarrow$ Một nửa của hình bình hành được tô màu cam, xanh dương, xanh lá có diện tích bằng 10 (đvdt).
		
		Diện tích tam giác cần tính bằng diện tích 3 nửa hình bình hành được tô màu cam, màu xanh dương, màu xanh lá cộng thêm diện tích 9 tam giác nhỏ bằng nhau: $3.10+9.1=39$ (đvdt).
	\end{loigiaichuong38}
\end{bt} 
\begin{bt}
	Hình dưới đây là một lá cờ đi biển. Mỗi cạnh của lá cờ được chia làm ba đoạn bằng nhau.  Hỏi tỉ lệ diện tích của phần trắng và phầm đen là bao nhiêu?
	\begin{loigiaichuong38}
		Ta đặt tên các điểm và vẽ thêm hai đường chéo của hình chữ nhật như sau. 
		Khi ấy lá cờ được chia làm $12$ tam giác có diện tích bằng nhau; trong đó có $8$ tam giác xám và $4$ tam giác trắng. Vậy tỉ lệ diện thích của phần trắng và phần xám là $1:2$
	\end{loigiaichuong38}
\end{bt}
\begin{bt}
	Cho $3$ hình vuông $ABCD$, $EFGH$và $IJKL$như hình vẽ sao cho $EH=FD=CG=BH=\dfrac{AB}{3}.$ Các điểm $I,J,K,L$ lần lượt là trung điểm các cạnh $EH,EF,FG,GH.$ Hỏi tỉ lệ $\dfrac{{{S}_{IJKL}}}{{{S}_{ABCD}}}$bằng bao nhiêu?
	\begin{loigiaichuong38}
		Ta dễ chứng minh được ${{S}_{IJKL}}=\dfrac{{{S}_{EFGH}}}{2}.$
		
		Lại có ${{S}_{EFGH}}={{S}_{ABCD}}-4{{S}_{EBH}}$. Mà ${{S}_{EBH}}=\dfrac{2}{3}{{S}_{ABH}}=\dfrac{2}{9}{{S}_{ABC}}=\dfrac{1}{9}{{S}_{ABCD}}.$
		
		Vậy ${{S}_{EFGH}}=\dfrac{5}{9}{{S}_{ABCD}}.$ Nên ${{S}_{IJKL}}=\dfrac{5}{18}{{S}_{ABCD}}$
	\end{loigiaichuong38}
\end{bt}
\begin{bt}
	Chu vi tam giác $ADF,BDE,DEF,FEC$lần lượt là 12, 24, 29, 24 như hình vẽ. Hỏi chu vi tam giác $ABC$ là bao nhiêu?
	\begin{loigiaichuong38}
		Ta có:
		\begin{align*}
			{{P}_{ABC}}&=AB+BC+AC \\ 
			& =AD+DB+BE+EC+CF+FA \\ 
			& =(AD+FA)+(DB+BE)+(EC+CF) \\ 
			& ={{P}_{ADF}}-DF+{{P}_{BDE}}-DE+{{P}_{CEF}}-EF \\ 
			& ={{P}_{ADE}}+{{P}_{BDE}}+{{P}_{CEF}}-(DF+DE+EF) \\ 
			& ={{P}_{ADE}}+{{P}_{BDE}}+{{P}_{CEF}}-{{P}_{DEF}} \\ 
			& =12+24+24-19=41 \\ 
		\end{align*}
		Vậy chu vi tam giác $ABC$ là 41 (đvđd)
	\end{loigiaichuong38}
\end{bt}
\Closesolutionfile{loigiaichung}



\def\i{\item}
\graphicspath{{../pictures/vande35/}}
\chapter{HÌNH HỌC TRỰC QUAN}
\section{Một số hình phẳng có dạng đặc biệt} 
\section*{Phần 2: Hình chữ nhật, hình thoi, hình bình hành, hình thang cân}
\subsection{Kiến thức cần nhớ}
\subsubsection{Hình chữ nhật}
\subsubsection*{a) Tính chất}
Trong hình chữ nhật $ABCD$: 
\begin{enumerate}[--, leftmargin=*]
	\i Bốn góc bằng nhau và bằng $90^\circ$. 
	\i Các cạnh đối bằng nhau.
	\i Hai đường chéo bằng nhau.
\end{enumerate}
\subsubsection*{b) Cách vẽ }
\textbf{Ví dụ:} Vẽ hình chữ nhật $ABCD$ có một cạnh bằng $5\, cm$, một cạnh bằng $3 \,cm$ 
\begin{enumerate}[Bước 1:, leftmargin=*]
	\i Vẽ đoạn thẳng $AB = 5\, cm$. 
	\i Vẽ đường thẳng vuông góc với $AB$ tại  $A$. Trên đường thẳng đó lấy điểm $D$ sao cho $AD = 3\, cm$. 
	\i Vẽ đường thẳng vuông góc với $AB$ tại $B$. Trên đường thẳng đó lấy điểm $C$ sao cho $BC = 3\, cm$. 
	\i Nối $D$ với $C$ ta được hình chữ nhật $ABCD$.
\end{enumerate}
\subsubsection{Hình thoi}
\subsubsection*{a) Tính chất}
Trong một hình thoi:
\begin{enumerate}[--, leftmargin=*]
	\i Bốn cạnh bằng nhau. 
	\i Hai đường chéo vuông góc với nhau.
	\i Các cạnh đối song song với nhau.
	\i Các góc đối bằng nhau
\end{enumerate}
\subsubsection*{b) Cách vẽ}
\textbf{Ví dụ:} Vẽ hình thoi $ABCD$ có một cạnh bằng $3\, cm$. 
\begin{enumerate}[Bước 1:, leftmargin=*]
	\i Vẽ đoạn thẳng $AB = 3\, cm$. 
	\i Vẽ đường thẳng đi qua $B$. Lấy điểm $C$ trên đường thẳng đó sao cho $BC = 3\, cm$. 
	\i Vẽ đường thẳng đi qua $C$  và song song với cạnh $AB$. Vẽ đường thẳng đi qua $A$ và song song với cạnh $BC$. 
	\i Hai đường thẳng này cắt nhau tại  $D$, ta được hình thoi $ABCD$.
\end{enumerate} 
\subsubsection{3. Hình bình hành.}
\subsubsection*{a) Tính chất}
Trong hình bình hành:
\begin{enumerate}[--, leftmargin=*]
	\i Các cạnh đối bằng nhau.
	\i Hai đường chéo cắt nhau tại trung điểm của mỗi đường. 
	\i Các cạnh đối song song với nhau.
	\i Các góc đối bằng nhau.
\end{enumerate}
\subsubsection*{b) Cách vẽ}
\textbf{Ví dụ:} Vẽ hình bình hành $ABCD$ có $AB = 3\, cm$, $BC = 3\, cm$.
\begin{enumerate}[Bước 1:, leftmargin=*]
	\i Vẽ đoạn thẳng $AB = 5\, cm$. 
	\i Vẽ đường thẳng đi qua  $B$. Trên đường thẳng đó lấy $C$  sao cho $BC = 3 \,cm$.
	\i Vẽ đường thẳng đi qua $A$ và song song với  $BC$, đường thẳng qua $C$ và song song với $AB$.
	\i Hai đường thẳng này cắt nhau tại $D$ ta được hình bình hành $ABCD$.
\end{enumerate}
\subsubsection{Hình thang cân}
\subsubsection*{a) Tính chất}
Trong hình thang cân:
\begin{enumerate}[--, leftmargin=*]
	\i Hai cạnh bên bằng nhau. 
	\i Hai đường chéo bằng nhau.
	\i Hai cạnh đáy song song với nhau. 
	\i Hai góc kề một đáy bằng nhau.
\end{enumerate}
\subsection{Thực hiện giải toán.}
\begin{vd}
	Trong các hình sau hình nào là hình chữ nhật, hình thoi, hình bình hành, hình thang cân.
	\loigiai{
		\begin{enumerate}[--, leftmargin=*]
			\i Hình chữ nhật: Hình 3, 7, 9.
			\i Hình thoi: 1, 7. 
			\i Hình bình hành: 3, 7, 9, 10. 
			\i Hình thang cân: 3, 4, 7, 8, 9.
		\end{enumerate} 
		\textbf{Nhận xét:} Hình chữ nhật cũng là một hình thang cân, một hình bình hành.
		Hình vuông cũng là một hình chữ nhật.
	}
\end{vd}
\begin{vd}
	Em hãy vẽ
	\begin{enumerate}[a), leftmargin=*]
		\i Hình chữ nhật $ABCD$ có một cạnh bằng 4 cm  và một cạnh bằng 3 cm. 
		\i Hình thoi $EFGH$ có cạnh  5 cm.
		\i Hình bình hành $MNPQ$ có độ dài 2 cạnh lần lượt là 3 cm, 4 cm. 
	\end{enumerate}
	\loigiai{
		\begin{enumerate}[a), leftmargin=*]
			\i \begin{enumerate}[Bước 1:, leftmargin=*]
				\i Vẽ đoạn thẳng $AB = 4\,cm$. 
				\i Vẽ đường thẳng vuông góc với $AB$ tại  $A$. Trên đường thẳng đó lấy điểm $D$ sao cho $AD = 3\,cm$. 
				\i Vẽ đường thẳng qua $D$ song song với $AB$. Vẽ đường thẳng qua $B$ song song với $AD$. Hai đường thẳng cắt nhau tại $C$.	
				\end{enumerate}
			\i \begin{enumerate}[Bước 1:, leftmargin=*]
				\i Vẽ đoạn thẳng  $AB = 5\,cm$.
				\i Vẽ đường thẳng đi qua $B$. Lấy điểm $C$  trên đường thẳng đó sao cho $BC = 5\,cm$. 
				\i Vẽ đường thẳng đi qua $C$ và song song với cạnh $AB$. Vẽ đường thẳng đi qua $A$ và song song với cạnh $BC$ ta được hình thoi $ABCD$ có cạnh $5\, cm$.
				\end{enumerate}
			\i \begin{enumerate}[Bước 1:, leftmargin=*]
				\i Vẽ đoạn thẳng $MN = 3\,cm$.
				\i Vẽ đường thẳng đi qua $N$. Trên đường thẳng đó lấy $P$  sao cho $NP = 4\,cm$. 
				\i Vẽ đường thẳng đi qua $M$ và song song với  $NP$, đường thẳng qua $P$  và song song với $MN$. Hai đường thẳng này cắt nhau tại $Q$ ta được hình bình hành $MNPQ$.
			\end{enumerate}
		\end{enumerate}
	}
\end{vd}
\begin{vd}
	Cho miếng bìa hình chữ nhật kích thước $4 \times 9$. 
	
	Hãy: 
	\begin{enumerate}[a), leftmargin=*]
		\i Cắt miếng bìa đó thành 3 phần rồi ghép thành một hình vuông
		\i Cắt miếng bìa đó thành 2 phần rồi ghép thành một hình vuông
	\end{enumerate}
	\loigiai{
		Hình vuông sau khi ghép sẽ có kích thước  
		\begin{enumerate}[a), leftmargin=*]
			\i Cắt theo đường kẻ đậm\\
			
			Ghép lại như sau:\\
			
			\i Cắt \\
			
			Ghép\\

		\end{enumerate}	
	}
\end{vd}
\subsection{Mở rộng kiến thức}
\subsection{Bài tập tự luyện}
\Opensolutionfile{loigiaichung}[loigiaichuong35]
\subsubsection{Mức độ cơ bản}
\begin{bt}
	\begin{enumerate}[a), leftmargin=*]
		\i Trong các hình sau hình nào là hình chữ nhật.
		
		\i Trong các hình sau hình nào là hình bình hành.
		
		\i Trong các hình sau hình nào là hình thoi.
		
		\i Trong các hình sau hình nào là hình thang cân.
	\end{enumerate}
	\begin{loigiaichuong35}
		\begin{enumerate}[a), leftmargin=*]
			\i Hình 2, 3.
			\i Hình 1, 2, 3.
			\i Hình 1.
			\i Hình 1, 3.
		\end{enumerate}
	\end{loigiaichuong35}
\end{bt}
\begin{bt}
	Điền các cụm từ sau vào chỗ trống để được khẳng định đúng.
	\begin{center}
		$60^\circ$ \quad\quad Bằng nhau \quad\quad		Hình vuông	\quad\quad	Ba 	\quad\quad	Đường chéo
	
	Song song và bằng nhau	\quad\quad	Hình thang cân	\quad\quad	Bốn \quad\quad $120^\circ$ 
	\end{center}		 
	\begin{enumerate}[a), leftmargin=*]
		\i \ldots\ldots\ldots là hình có 4 cạnh bằng nhau,  góc bằng nhau và bằng  
		\i Hình thoi có \ldots\ldots\ldots bằng nhau.
		\i Trong tam giác đều có ba cạnh bằng nhau và ba góc bằng nhau và mỗi góc bằng \ldots\ldots\ldots
		\i Hình chữ nhật có \ldots\ldots\ldots góc vuông.
		\i Hình thang có hai đường chéo bằng nhau là \ldots\ldots\ldots
		\i Hình bình hành có các cạnh đối \ldots\ldots\ldots
		\i Hình lục giác đều có \ldots\ldots\ldots đường chéo bằng nhau.
		\i Hình lục giác đều có 6 cạnh bằng nhau, 6 góc bằng nhau và mỗi góc bằng \ldots\ldots\ldots
	\end{enumerate}
	\begin{loigiaichuong35}
		\begin{enumerate}[a), leftmargin=*]
			\i Hình vuông.
			\i Đường chéo.
			\i  $60^\circ$.
			\i Bốn.
			\i Hình thang cân.
			\i Song song và bằng nhau
			\i Ba.
			\i  $120^\circ$.
		\end{enumerate}
	\end{loigiaichuong35}
\end{bt}
\begin{bt}
	Tìm các vật dụng trong đời sống có hình dạng là hình chữ nhật, hình bình hành, hình thoi.
	\begin{loigiaichuong35}
		\begin{enumerate}[--, leftmargin=*]
			\i Hình chữ nhât: Cửa sổ, bàn học, khung ảnh, ti vi, quyển sách, \ldots
			\i Hình bình hành: mái nhà, cầu thang, \ldots
			\i Hình thoi: Họa tiết gạch, câu đối, hoa văn chiếu trúc, móc treo đồ, \ldots
		\end{enumerate}
	\end{loigiaichuong35}
\end{bt}
\begin{bt}
	Vẽ hình chữ nhật $ABCD$ có chiều dài, chiều rộng lần lượt bằng 6 $cm$  và 4 $cm$
	\begin{loigiaichuong35}
		\begin{enumerate}[Bước 1:, leftmargin=*]
			\i Vẽ đoạn thẳng $AB = 6\, cm$. 
			\i Vẽ đường thẳng vuông góc với $AB$ tại  $A$. Trên đường thẳng đó lấy điểm $D$ sao cho $AD = 4\, cm$. 
			\i Vẽ đường thẳng qua $D$ và song song với $AB$. Vẽ đường thẳng qua $B$ và song song với $AD$. 
			\i Hai đường thẳng này cắt nhau tại $C$. Ta được hình chữ nhật $ABCD$ cần vẽ.
		\end{enumerate}
	\end{loigiaichuong35}
\end{bt}
\begin{bt}
	Vẽ hình thoi $ABCD$ có cạnh là $5\, cm$
	\begin{loigiaichuong35}
		\begin{enumerate}[Bước 1:, leftmargin=*]
			\i Vẽ đoạn thẳng $AB = 5\,cm$.
			\i Vẽ đường thẳng đi qua $B$. Lấy điểm $C$  trên đường thẳng đó sao cho $BC = 5\,cm$. 
			\i Vẽ đường thẳng đi qua $C$ và song song với cạnh $AB$. Vẽ đường thẳng đi qua $A$ và song song với cạnh $BC$ ta được hình thoi $ABCD$ có cạnh $5\,cm$.
		\end{enumerate}
	\end{loigiaichuong35}
\end{bt}
\begin{bt}
	Vẽ hình bình hành $ABCD$ có $2$ cạnh lần lượt là $3\, cm$, $2\, cm$.
	\begin{loigiaichuong35}
		\begin{enumerate}[Bước 1:, leftmargin=*]
			\i Vẽ đoạn thẳng $AB = 3\,cm$ 
			\i Vẽ đường thẳng đi qua $B$. Trên đường thẳng đó lấy $C$ sao cho $BC= 2\,cm$. 
			\i Vẽ đường thẳng đi qua $A$ và song song với  $BC$, đường thẳng qua $C$  và song song với $AB$.    
			\i Hai đường thẳng này cắt nhau tại  $D$ ta được hình bình hành $ABCD$. 
		\end{enumerate}
	\end{loigiaichuong35}
\end{bt} 
\begin{bt}
	Trong hình vẽ sau có bao nhiêu hình chữ nhật, hình thang cân
	\begin{loigiaichuong35}
		\begin{enumerate}[--, leftmargin=*]
			\i Có 2 hình chữ nhật.
			\i Có 4 hình thang cân.
		\end{enumerate}
	\end{loigiaichuong35}
\end{bt}
\begin{bt}
	Trong hình vẽ sau dây có bao nhiêu hình bình hành, bao nhiêu hình thoi, bao nhiêu hình thang cân
	\begin{loigiaichuong35}
		\begin{enumerate}[a), leftmargin=*]
			\i \begin{enumerate}[--, leftmargin=*]
				\i Có 21 hình bình hành
				\i Có 14 hình thoi
				\i Có 14 hình thang cân
				\end{enumerate}
			\i \begin{enumerate}[--, leftmargin=*]
				\i Có 16 hình bình hành
				\i Có 5 hình thoi
				\i Có 14 hình thang cân
			\end{enumerate}
		\end{enumerate}
	\end{loigiaichuong35}
\end{bt}
\begin{bt}
	Cho hình chữ nhật $ABCD$ có $AB = 5 \,cm$, $AD = 3\, cm$. Trên $CD$ lấy các điểm $G, E$ để được.
	\begin{enumerate}[a), leftmargin=*]
		\i Hình thang cân có đáy $GE = 3\, cm$.
		\i Hình thang cân có đáy $GE = 7\, cm$. 
	\end{enumerate}
	Em hãy vẽ và nêu cách vẽ của các hình trên.
	\begin{loigiaichuong35}
		\begin{enumerate}[a), leftmargin=*]
			\i \begin{enumerate}[Bước 1:, leftmargin=*]
				\i Gọi $M$ là trung điểm của  $CD$. 
				\i Trên $CD$ xác định điểm $E$ sao cho $ME = 1,5\,cm$. Trên $CD$ xác định điểm $G$  sao cho $MG = 1,5\, cm$.  Ta được hình thang cân $ABEG$.
				\end{enumerate}
			\i \begin{enumerate}[Bước 1:, leftmargin=*]
				\i Trên đường thẳng chứa cạnh $CD$ xác định điểm $G$ nằm ngoài đoạn $CD$ sao cho $DG = 1\,cm$
				\i Trên đường thẳng chứa cạnh $CD$ xác định điểm $E$ nằm ngoài đoạn $CD$ sao cho $CE =1\, cm$. Ta được hình thang cân  $ABEG$.
			\end{enumerate}
		\end{enumerate}
	\end{loigiaichuong35}
\end{bt}
\begin{bt}
	Từ ba hình tam giác đều rồi ghép lại để được 1 hình thang cân.
	\begin{loigiaichuong35}
		Ta có 3 hình tam giác đều giống nhau: 
		
		
		Ghép 3 tam giác đều lại với nhau ta được hình thang cân như sau: 
		
	\end{loigiaichuong35}
\end{bt}
\begin{bt}
	Cho tam giác vuông như hình vẽ. Với 4 tam giác vuông như trên, em hãy tìm cách ghép:
	\begin{enumerate}[a), leftmargin=*]
		\i Một hình chữ nhật.
		\i Một hình bình hành (không phải hình chữ nhật).
		\i Một hình thang cân.
		\i Một hình thoi.
	\end{enumerate}
	\begin{loigiaichuong35}
		Ghép 4 tam giác vuông như nhau ta được các hình như sau: 
		\begin{enumerate}[a), leftmargin=*]
			\i Hình chữ nhật: 
			
			\i Hình bình hành: 
			
			\i Hình thang cân 
			
			\i Hình thoi: 
		\end{enumerate}
		
	\end{loigiaichuong35}
\end{bt}
\subsubsection*{Mức độ nâng cao}
\begin{bt}
	Cho một hình thoi lớn có cạnh bằng 6 $cm$. Hãy dùng thước kẻ các đường chia hình thoi này thành 9 hình thoi nhỏ bằng nhau (chia theo dạng đường lưới $3\times 3$). Sau  khi chia em hãy nêu cách cắt một số đoạn thẳng để được hình vẽ dưới đây.
	\begin{loigiaichuong35}
		Ta chia hình thoi theo đường lưới như hình vẽ. Sau đó cắt các cạnh của hình thoi nhỏ được đánh dấu, gập các miếng có cạnh bị cắt để được hình theo yêu cầu 
	\end{loigiaichuong35}
\end{bt}
\begin{bt}
	Tiến sĩ John đã sáng chế ra 3 loại chip $A$, $B$, $C$. Ông cất giữ chúng trong 3 cái hộp và mã hóa khác nhau bằng các bảng ô vuông bỏ đi một số ô. Nếu dùng các quân domino như hình vẽ dưới đây để lắp vừa các ô còn lại thì hộp sẽ được mở.
	
	Domino
	
	Hộp thứ hai chứa con chíp $B$ được mã hóa bởi bảng ô vuông cỡ $6 \times 6$ bị bỏ đi 2 ô được tô đen như hình vẽ.
	
	Tiễn sĩ John muốn dùng 17 quân domino để mở khóa. Hỏi ông có thể mở được khóa không? Nếu có hãy chỉ ra một cách mở.
	\begin{loigiaichuong35}
		Tiến sĩ John có thể mở được khóa này bằng cách đặt 17 quân domino như sau: 
	\end{loigiaichuong35}
\end{bt}
\begin{bt}
	Em hãy điền các loại hình: hình chữ nhât, hình thoi, hình thnag cân, hình bình hành vào các ô trống sao cho mỗi hàng ngang, dọc không có hình nào bị lặp lại.
	\begin{loigiaichuong35}
		chèn ảnh
	\end{loigiaichuong35}
\end{bt}
\begin{bt}
	An có một tấm bìa gồm nhiều hình vuông có in các hình vẽ như minh họa dưới đây.
	Từ tấm bìa đó An có thể gấp được thành khối hình lập phương nào dưới đây?
	\begin{loigiaichuong35}
		chèn ảnh
	\end{loigiaichuong35}
\end{bt}
\begin{bt}
	Em hãy dùng các họa tiết hình thang cân với độ dài các cạnh là $1\, cm$, $1\, cm$, $1\,cm$, $2\, cm$  (hình này được ghép từ ba tam giác đều như hình vẽ) để trang trí chiếc khăn hình tam giác đều
	\begin{enumerate}[a), leftmargin=*]
		\i có cạnh bằng $9\,cm$
		\i có cạnh bằng $10\,cm$  
	\end{enumerate} 
	\begin{loigiaichuong35}
		\begin{enumerate}[a), leftmargin=*]
			\i Ghép như hình bên:  
			
			\i Vì mỗi họa tiết hình thang cân chia được 3 tam giác đều. Tam giác đều cạnh 10cm chia được 100 tam giác đều. 100 chia 3 dư 1 nên không thể ghép các hình thang cân lại thành tam giác đều cạnh 10cm được.
		\end{enumerate}
	\end{loigiaichuong35}
\end{bt}                                        
\Closesolutionfile{loigiaichung}
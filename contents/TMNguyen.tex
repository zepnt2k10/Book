
\chapter{Đồ thị và các dạng toán liên quan }

\section[Hoành độ giao điểm của đường thẳng với Parabol]{Các bài toán liên quan đến hoành độ giao điểm của đường thẳng với Parabol}


%1
\begin{vd}[Đề thi tuyển sinh vào lớp 10, Hà Nội, năm học 2016 - 2017]
	Trên mặt phẳng tọa độ $\left( Oxy \right)$ cho đường thẳng $\left( d \right)$:$y=3x+{{m}^{2}}-1$ và parabol $(P)\colon \ y=x^2$.
	
	a) Chứng minh $\left( d \right)$ luôn cắt $\left( P \right)$ tại hai điểm phân biệt với mọi $m$.
	
	b) Gọi $x_1; x_2$ là hoành độ của $\left( d \right)$ và $\left( P \right)$.\\
	Tìm $m$ để $\left( x_1+1 \right)\left( x_2+1 \right)=1$.
	\loigiai
	{
		a) Hoành độ giao điểm của $\left( d \right)$ và $\left( P \right)$ thỏa mãn phương trình:
		\[\begin{aligned}
			&x^2=3x+{{m}^{2}}-1\Leftrightarrow x^2-3x-{{m}^{2}}+1=0\\
			\Leftrightarrow& x^2-3x-{{m}^{2}}+1=0\\
			\Rightarrow& \Delta ={{3}^{2}}-4\left( -{{m}^{2}}+1 \right)=9+4{{m}^{2}}-4=4{{m}^{2}}+5>0, \forall m.
		\end{aligned}
		\]
		Do $a=1\ne 0\Rightarrow \left( d \right)$ luôn cắt $\left( P \right)$ tại hai điểm phân biệt $\forall m$.
		\\
		b) Từ $\begin{aligned}[t]%\tag{*}
			\left( x_1+1 \right)\left( x_2+1 \right)=1
			\Leftrightarrow & x_1x_2+x_1+x_2+1=1 
			\Leftrightarrow  x_1x_2+x_1+x_2=0.\qquad (*)
		\end{aligned}$
		\\
		Theo định lý Vi-et $\left\{ \begin{aligned}
			& x_1+x_2=3 \\ 
			& x_1x_2=-{{m}^{2}}+1. \\ 
		\end{aligned} \right.$\\
		Thay vào (*) $\Rightarrow 3-{{m}^{2}}+1=0\Leftrightarrow {{m}^{2}}=4\Rightarrow m=\pm 2$.\\ 
		Vậy $m=\pm 2$ thỏa mãn điều kiện bài toán.
		
	}
\end{vd}
%2
\begin{vd}[Đề thi tuyển sinh vào lớp 10, Hà Nội, năm học 2013 - 2014]
	Cho parabol $(P)\colon \ y=\dfrac{1}{2}x^2$ và đường thẳng $(d)\colon \ y=mx-\dfrac{1}{2}{{m}^{2}}+m+1$.\\
	a) Với $m=1$, xác định tọa độ giao điểm của $\left( d \right)$ và $\left( P \right)$.\\
	b) Tìm các giá trị của $m$ để $\left( d \right)$ cắt $\left( P \right)$  tại hai điểm phân biệt  có hoành độ $x_1; x_2$ sao cho $\left| x_1-x_2 \right|=2$.
	\loigiai{
	a) Với $m=1\Rightarrow \left( d \right)\colon y=x+\dfrac{3}{2}$.\\ 
	Tọa độ giao điểm của $\left( d \right)$ và $\left( P \right)$ thỏa mãn hệ
	$\left\{ \begin{aligned}
		& y=\dfrac{x^2}{2} \\ 
		& y=x+\dfrac{3}{2}. \\ 
	\end{aligned} \right.$\\
	$\begin{aligned}
		&\Rightarrow \dfrac{x^2}{2}=x+\dfrac{3}{2}
		\Leftrightarrow x^2-2x-3=0
		\Leftrightarrow x^2-3x+x-3=0\\
		&\Leftrightarrow x\left( x-3 \right)+\left( x-3 \right)=0
		\Leftrightarrow \left( x-3 \right)\left( x+1 \right)=0\\
		&\Leftrightarrow \left[ \begin{aligned}
			& x-3=0 \\ 
			& x+1=0 \\ 
		\end{aligned} \right.\Leftrightarrow \left[ \begin{aligned}
			& x=3\Rightarrow y=\dfrac{9}{2} \\ 
			& x=-1\Rightarrow y=\dfrac{1}{2} \\ 
		\end{aligned} \right. 
	\Rightarrow A\left( 3; \dfrac{9}{2} \right); B\left( -1; \dfrac{1}{2} \right).
	\end{aligned}
	$ 
	\\
	b) Hoành độ giao điểm của $\left( d \right)$ và $\left( P \right)$ thỏa mãn phương trình:
	$$\begin{aligned}
		&\dfrac{x^2}{2}=mx-\dfrac{{{m}^{2}}}{2}+m+1 \Leftrightarrow x^2-2mx+{{m}^{2}}-2m-2=0\\
		&\Delta '={{m}^{2}}-\left( {{m}^{2}}-2m-2 \right)=2m+2>0 \Rightarrow m>-1.
	\end{aligned}  $$
	Để $\left( d \right)$ cắt $\left( P \right)$ tại hai điểm phân biệt có hoành độ $x_1; x_2$ thỏa mãn: $\left| x_1-x_2 \right|=2$.
	$$\begin{aligned}
		&\Rightarrow \Delta '=2m+2>0\Rightarrow m>-1.\\
		&x_1=\dfrac{-b+\sqrt{\Delta '}}{a}\text{ và } x_2=\dfrac{-b-\sqrt{\Delta '}}{a} \\
		&\Rightarrow x_1-x_2=\dfrac{2\sqrt{\Delta '}}{a}	
		\Rightarrow \left| x_1-x_2 \right|=\left| \dfrac{2\sqrt{\Delta '}}{a} \right|=2\\
		&\Rightarrow \left| \dfrac{2\sqrt{2m+2}}{1} \right|=2\Rightarrow \sqrt{2m+2}=1\Rightarrow 2m+2=1
		\Rightarrow 2m=-1\Rightarrow m=-\dfrac{1}{2}.
	\end{aligned}
	$$
	Vậy $m=\dfrac{-1}{2}$ thỏa mãn điều kiện đề bài.
	}
\end{vd}
\begin{nx}
	Nhiều bạn sẽ dùng định lý Vi-et để giải bài toán trên bằng cách ${{\left( x_1-x_2 \right)}^{2}}={{\left( x_1+x_2 \right)}^{2}}-4x_1x_2$ tuy nhiên nếu dùng công thức nghiệm ta có ngay $\left| x_1-x_2 \right|=\left| \dfrac{2\sqrt{\Delta '}}{a} \right|=\left| \dfrac{\sqrt{\Delta }}{a} \right|$.
\end{nx}
%3
\begin{vd}
	[Đề thi tuyển sinh vào lớp 10 tỉnh Thanh Hóa năm học 2016 - 2017]
	Cho mặt phẳng tọa độ $Oxy$ cho đường thẳng $(d)\colon \ y=2x-m+3$ và parabol $\left( P \right)\colon y=x^2$.\\
	a) Tìm $m$ để đường thẳng $\left( d \right)$ đi qua điểm $A\left( 1; 0 \right)$.\\
	b) Tìm $m$ để đường thẳng $\left( d \right)$ cắt parabol $\left( P \right)$ tại hai điểm phân biệt có hoành độ lần lượt là $x_1,\ x_2$ thỏa mãn $x_{1}^{2}-2x_2+x_1x_2=-12$.
	\loigiai{
		a) Để đường thẳng $\left( d \right)$ đi qua điểm $A\left( 1; 0 \right)$ thì 
		$$0=2.1-m+3\Leftrightarrow -m+5=0\Rightarrow m=5.$$
		Vậy $m=5$ thì đường thẳng $\left( d \right)$ đi qua điểm $A\left( 1; 0 \right)$.
		\\
		b) Hoành độ giao điểm của $\left( d \right)$ và $\left( P \right)$ thỏa mãn phương trình:
		$$x^2=2x-m+3\Leftrightarrow x^2-2x+m-3=0\qquad (*)$$
		$\Delta '=1-\left( m-3 \right)=4-m>0\Rightarrow m<4$.\\
		(*) $\Leftrightarrow x_{1}^{2}=2x_1-m+3$.\\
		Từ $x_{1}^{2}-2x_2+x_1x_2=-12\Leftrightarrow 2x_1-m+3-2x_2+x_1x_2=-12$ 	(**)\\
		Theo định lý Vi-et $\heva{&x_1+x_2=2    \\ &x_1x_2=m-3 }$,
		ta có $\begin{aligned}[t]
			(**)  &\Leftrightarrow 2x_1-m+3-2x_2+m-3=-12\\
			&\Leftrightarrow 2\left( x_1-x_2 \right)=-12\Rightarrow x_1-x_2=-6.
		\end{aligned}  $\\
		Ta có $\left\{ \begin{aligned}
			& x_1-x_2=-6 \\ 
			& x_1+x_2=2 \\ 
		\end{aligned} \right. \Rightarrow  \heva{&x_1=-2 \\ &x_2=4}$ mà 		$x_1x_2=m-3\Leftrightarrow -8=m-3\Rightarrow m=-5$.\\
		Vậy với $m=-5$ đường thẳng $\left( d \right)$ cắt parabol $\left( P \right)$ tại hai điểm phân biệt thỏa mãn yêu cầu đề bài.
	}
\end{vd}
%4
\begin{vd}
	Cho Parabol $\left( P \right)\colon y=x^2$ và đường thẳng $\left( d \right)\colon y=\left( m-2 \right)x+2$ ($m$ là tham số). Gọi $x_1;x_2$ là hoành độ các giao điểm của $\left( P \right)$ và $\left( d \right)$.\\
	Tìm $m$ để $\sqrt{x_{1}^{2}+2022}-2x_1=\sqrt{x_{2}^{2}+2022}+2x_2$. 
	\loigiai{
		Xét phương trình hoành độ $x^2=\left( m+2 \right)x+2\Leftrightarrow x^2-\left( m-2 \right)x-2=0.$\\
		$\Delta ={{\left( m-2 \right)}^{2}}-4.\left( -2 \right)={{\left( m-2 \right)}^{2}}+8>0$.\\
		$\Rightarrow $ Phương trình đã cho luôn có hai nghiệm phân biệt $x_1; x_2$ với mọi $m$.\\
		$\begin{aligned}
			&\sqrt{x_{1}^{2}+2022}-2x_1=\sqrt{x_{2}^{2}+2022}+2x_2\\
			\Leftrightarrow &\sqrt{x_{1}^{2}+2022}-\sqrt{x_{2}^{2}+2022}-2\left( x_1+x_2 \right)=0\\
			\Leftrightarrow &\dfrac{\left( \sqrt{x_{1}^{2}+2022}-\sqrt{x_{2}^{2}+2022} \right)\left( \sqrt{x_{1}^{2}+2022}+\sqrt{x_{2}^{2}+2022} \right)}{\left( \sqrt{x_{1}^{2}+2022}+\sqrt{x_{2}^{2}+2022} \right)}-2\left( x_1+x_2 \right)=0\\
			\Leftrightarrow &\dfrac{x_{1}^{2}+2022-\left( x_{2}^{2}+2022 \right)}{\sqrt{x_{1}^{2}+2022}+\sqrt{x_{2}^{2}+2022}}-2\left( x_1+x_2 \right)=0\\
			\Leftrightarrow &\dfrac{x_{1}^{2}-x_{2}^{2}}{\sqrt{x_{1}^{2}+2022}+\sqrt{x_{2}^{2}+2022}}-2\left( x_1+x_2 \right)=0\\
			\Leftrightarrow &\left( x_1+x_2 \right)\left( \dfrac{x_1-x_2}{\sqrt{x_{1}^{2}+2022}+\sqrt{x_{2}^{2}+2022}}-2 \right)=0
		\end{aligned}$\\ 
		$\Leftrightarrow \left[ \begin{aligned}
			& x_1+x_2=0 \\ 
			& \dfrac{x_1-x_2}{\sqrt{x_{1}^{2}+2022}+\sqrt{x_{2}^{2}+2022}}-2=0. \\ 
		\end{aligned} \right.$ \\
		Nếu $x_1+x_2=0\Leftrightarrow -\dfrac{b}{a}=0\Rightarrow m-2=0\Rightarrow m=2$.\\
		Nếu $\dfrac{x_1-x_2}{\sqrt{x_{1}^{2}+2022}+\sqrt{x_{2}^{2}+2022}}-2=0 \Rightarrow \dfrac{x_1-x_2}{\sqrt{x_{1}^{2}+2022}+\sqrt{x_{2}^{2}+2022}}=2$ \\
		$\Rightarrow x_1-x_2=2\left( \sqrt{x_{1}^{2}+2022}+\sqrt{x_{2}^{2}+2022} \right)$\\
		Mà $\sqrt{x_{1}^{2}+2022}+\sqrt{x_{2}^{2}+2022}>\sqrt{x_{1}^{2}}+\sqrt{x_{2}^{2}}=\left| x_1 \right|+\left| x_2 \right|\ge x_1-x_2$\\
		$\Rightarrow x_1-x_2<2\left( \sqrt{x_{1}^{2}+2022}+\sqrt{x_{2}^{2}+2022} \right)$ hay phương trình (*) vô nghiệm.\\
		Kết luận: $m=2$.\\
		Bài đồ thị này khó vì ta cần khai thác giả thiết:
		$\sqrt{x_{1}^{2}+2022}-x_1=\sqrt{x_{2}^{2}+2022}+x_2.$
	}
\end{vd}

\newpage
\section{Các bài toán liên quan đến tung độ giao điểm của đường thẳng và Parabol}
% Ví dụ 5
\begin{vd}[Đề thi tuyển sinh vào lớp 10, tỉnh Thanh Hóa năm học 2017 - 2018]
	Trong mặt phẳng $Oxy$, đường thẳng $( d )\colon y=mx+1$ và parabol $(P)\colon y=2x^2$.\\
	a) Tìm $m$ để đường thẳng $\left( d \right)$ đi qua điểm $A\left( 1; 3 \right)$.\\
	b) Chứng minh rằng đường thẳng $\left( d \right)$ luôn cắt parabol $\left( P \right)$ tại hai điểm phân biệt $A\left( x_1; y_1 \right); B\left( x_2; y_2 \right)$. Hãy tính giá trị biểu thức $T=x_1x_2+y_1y_2$.
	\loigiai{
		a) Để đường thẳng $\left( d \right)$ đi qua điểm $A\left( 1; 3 \right)$ thì: $3=m.1+1\Rightarrow m=2$.
		Vậy $m=2$ thì đường thẳng $\left( d \right)$ đi qua điểm $A\left( 1; 3 \right)$.\\
		b) Hoành độ giao điểm của $\left( d \right)$ và $\left( P \right)$ thỏa mãn phương trình:\\
		$2x^2=mx+1\Leftrightarrow {{2}^{2}}-mx-1=0$\\
		$\Delta ={{m}^{2}}-4.\left( -1 \right).2={{m}^{2}}+8>0, \forall m$.\\
		Tung độ $y$ dùng một trong 2 công thức $\left\{ \begin{aligned}
			& y=2x^2 \\ 
			& y=mx+1. \\ 
		\end{aligned} \right.$\\
		Thay vào $y=mx+1 \Rightarrow y_1y_2=\left( mx_1+1 \right) \left( mx_2+1 \right)={{m}^{2}}x_1x_2+m\left( x_1+x_2 \right)+1$ \\
		Từ $\begin{aligned}[t]
			T&=x_1x_2+y_1y_2=x_1x_2+{{m}^{2}}x_1x_2+m\left( x_1+x_2 \right)+1\\
			&=\left( {{m}^{2}}+1 \right)x_1x_2+m\left( x_1+x_2 \right)+1.
		\end{aligned}$\\
		Theo định lý Viet $\left\{ \begin{aligned}
			& x_1+x_2=\dfrac{-1}{2}=\dfrac{c}{a} \\ 
			& x_1x_2=\dfrac{m}{2}=-\dfrac{b}{a} \\ 
		\end{aligned} \right.$ \\
		$\Rightarrow T=\left( {{m}^{2}}+1 \right)\left( -\dfrac{1}{2} \right)+m.\dfrac{m}{2}+1=-\dfrac{{{m}^{2}}}{2}-\dfrac{1}{2}+\dfrac{{{m}^{2}}}{2}+1=\dfrac{1}{2}$. 
		Vậy $T=\dfrac{1}{2}, \forall m$.
	}	
\end{vd}
% Ví dụ 6
\begin{vd}
	Trong mặt phẳng tọa độ $Oxy$ cho đường thẳng $(d)\colon y=\left( m+2 \right)x+3$ và Parabol $( P )\colon y=2x^2$.\\
	a) Chứng minh đường thẳng $\left( d \right)$ luôn cắt Parabol $\left( P \right)$ tại hai điểm phân biệt.\\
	b) Gọi $A\left( x_1;y_1 \right)$, $B\left( x_2;y_2 \right)$ là hai giao điểm của $\left( P \right)$ và $\left( d \right)$.\\ Tìm $m$ để $2\sqrt{y_1y_2}=x_{1}^{2}+x_{2}^{2}$.
	\loigiai{
		a) Xét phương trình hoành độ $2x^2-\left( m+2 \right)x-3=0$ có $a.c=2.\left( -3 \right)=-6<0$.\\
		Vậy phương trình trên luôn có hai nghiệm phân biệt hay đường thẳng $\left( d \right)$luôn cắt Parabol $\left( P \right)$ tại hai điểm phân biệt.\\
		b) Ta có $x_{1}^{2}+x_{2}^{2}={{\left( x_1+x_2 \right)}^{2}}-2x_1x_2$.\\
		Vì $y_1=2x_{1}^{2}>0$; $y_2=2x_{2}^{2}>0$ nên
		$\begin{aligned}[t]
			&2\sqrt{y_1y_2}=x_{1}^{2}+x_{2}^{2} \\ 
			& \Leftrightarrow 4x_1x_2={{\left( x_1+x_2 \right)}^{2}}-2x_1x_2 \\ 
			& \Leftrightarrow {{\left( x_1+x_2 \right)}^{2}}-6x_1x_2=0\quad (*) \\ 
		\end{aligned}$\\
		Áp dụng hệ thức Viet có $x_1+x_2=\dfrac{m+2}{2};$ $x_1x_2=\dfrac{-3}{2}$.\\
		Thay vào phương trình (*) được
		$\dfrac{{{\left( m+2 \right)}^{2}}}{4}-\dfrac{54}{4}=0$.\\
		Giải phương trình tìm được $\left[ \begin{aligned}
			& {{m}_{1}}=\sqrt{54}-2 \\ 
			& {{m}_{2}}=-\sqrt{54}-2. \\ 
		\end{aligned} \right.$
		
	}
\end{vd}
\newpage
\section{Các bài toán liên quan đến hình học}
%7
\begin{vd}[Đề thi khảo sát chất lượng Quận Bắc Từ Liêm, năm học 2016 - 2017]
	\label{vd7}
	Trong mặt phẳng tọa độ $Oxy$ cho Parabol $(P)\colon y=x^2$ và đường thẳng $(d)\colon y=2x+m$ ($m$ là tham số).\\
	a) Xác định $m$ để đường thẳng $\left( d \right)$ tiếp xúc với Parabol $\left( P \right)$. Tìm hoành độ tiếp điểm.\\
	b) Tìm giá trị của $m$ để đường thẳng $\left( d \right)$ cắt Parabol $\left( P \right)$ tại 2 điểm $A,B$ nằm hai phía của trục tung sao cho diện tích tam giác $AOM$ gấp hai lần diện tích tam giác $BOM$ (M là giao điểm của $\left( d \right)$ và trục tung).
	\loigiai{
		a) Đường thẳng $\left( d \right)$ tiếp xúc với Parabol $\left( P \right)$ khi và chỉ khi phương trình hoành độ $x^2=2x+m\Leftrightarrow x^2-2x-m=0$ có nghiệm kép.\\
		Ta tìm được $m=-1$ và $x=1$.
		\immini{
			b) Đường thẳng $\left( d \right)$ cắt Parabol $\left( P \right)$ tại 2 điểm $A,B$ nằm hai phía của trục tung tức là phương trình $x^2-2x-m=0$ có hai nghiệm trái dấu hay $m>0$.\\
			+ $M\left( 0;m \right)$; $A\left( {{x}_{A}};{{y}_{A}} \right)$ ; $B\left( x_B;y_B \right)$.\\
			Các hoành độ ${{x}_{A}};x_B$ là nghiệm của phương trình $$x^2-2x-m=0$$
			Giải ra ta được $\left( {{x}_{A}};x_B \right)=\left( 1+\sqrt{1+m};1-\sqrt{1+m} \right)$ \\
			hoặc $\left( {{x}_{A}};x_B \right)=\left( 1-\sqrt{1+m};1+\sqrt{1+m} \right)$.\\
			+ Nhìn trên hình vẽ, ta thấy hai tam giác $OAM$ và $OBM$ có chung cạnh đáy $OM$. Để ${{S}_{AOM}}=2{{\text{S}}_{BOM}}$ thì chiều cao $AH$ phải gấp đôi chiều cao $BK$.
		}
		{
			\includegraphics[width=3cm]{1.Hinh_anh/CD_1/Vd7.pdf}
		}
		 $\Leftrightarrow \left| {{x}_{A}} \right|=2\left| x_B \right|$. Vì $\left| {{x}_{A}} \right|>\left| x_B \right|$ nên chọn $\left( {{x}_{A}};x_B \right)=\left( 1+\sqrt{1+m};1-\sqrt{1+m} \right)$.\\
		Giải $\left| 1+\sqrt{1+m} \right|=2\left| 1-\sqrt{1+m} \right|$ được $m=8;m=\dfrac{-8}{9}$ . \\
			Kết hợp điều kiện $m>0$ chọn $m=8.$ 
%		\immini{
%			
%		}{
%			Hinh ve
%		}
	
		
	}
\end{vd}

\begin{cy}
	\begin{itemize}
		\item Trên đây là bài toán có liên quan đến tỉ lệ diện tích của hai tam giác $OAM$ và $OBM$. Vì hai tam giác này chung cạnh đáy OM nên $\dfrac{{{S}_{OAM}}}{{{S}_{OBM}}}=\dfrac{OH}{BK}=2$.
		\item Mặt khác hai tam giác trên lại có chung đường cao kẻ từ O nên $\dfrac{{{S}_{OAM}}}{{{S}_{OBM}}}=\dfrac{AM}{BM}=2$.
	\end{itemize}
	Nếu bài toán hỏi tìm $m$ để $AM=2BM$ thì bạn đọc cũng làm tương tự như ví dụ trên.	
\end{cy}
%8
\begin{vd}
	Trên mặt phẳng tọa độ $Oxy$, cho Parabol $(P)\colon y={{x}^{2}}$ và đường thẳng $( d)\colon y=3x-m$. Tìm giá trị của $m$ để đường thẳng $\left( d \right)$ cắt Parabol $\left( P \right)$ tại 2 điểm $A,B$ cùng nằm về bên phải của trục tung sao cho $B$ là trung điểm của $AM$  ($M$ là giao điểm của $\left( d \right)$ và trục tung).
\end{vd}
\begin{dapan}
	Đường thẳng $\left( d \right)$ cắt Parabol $\left( P \right)$ tại 2 điểm $A,B$ cùng nằm về bên phải của trục tung tức là phương trình ${{x}^{2}}-3x+m=0$ có hai nghiệm dương hay $0<m<\dfrac{9}{4}$.\\
	+$M\left( 0;m \right)$; $A\left( {{x}_{A}};{{y}_{A}} \right)$; $B\left( {{x}_{B}};{{y}_{B}} \right)$.\\
	Các hoành độ ${{x}_{A}};{{x}_{B}}$ là nghiệm của phương trình${{x}^{2}}-3x+m=0$.\\
	Giải ra ta được $\left( {{x}_{A}};{{x}_{B}} \right)=\left( \dfrac{3+\sqrt{9-4m}}{4};\dfrac{3-\sqrt{9-4m}}{4} \right)$ \\
	hoặc $\left( {{x}_{A}};{{x}_{B}} \right)=\left( \dfrac{3-\sqrt{9-4m}}{4};\dfrac{3+\sqrt{9-4m}}{4} \right)$.
	\vspace{-0.5cm}
	\chenhinh{
		+ Nhìn trên hình vẽ, ta thấy $\dfrac{{{S}_{OAM}}}{{{S}_{OBM}}}=\dfrac{AM}{BM}=2$ (hai tam giác chung nhau đường cao kẻ từ $O$, $M$ là trung điểm của $AB$).\\
		Mặt khác $\dfrac{{{S}_{OAM}}}{{{S}_{OBM}}}=\dfrac{AH}{BK}$ (Hai tam giác chung cạnh đáy $OM$).\\
		Suy ra $\dfrac{AH}{BK}=2$
		$\Leftrightarrow \left| {{x}_{A}} \right|=2\left| {{x}_{B}} \right|$.\\
		Vì $\left| {{x}_{A}} \right|>\left| {{x}_{B}} \right|$ nên chọn $$\left( {{x}_{A}};{{x}_{B}} \right)=\left( \dfrac{3+\sqrt{9-4m}}{4};\dfrac{3-\sqrt{9-4m}}{4} \right).$$
	}
	{
		\includegraphics[width=5cm]{1.Hinh_anh/CD_1/Vd8.pdf}
	}
	Giải $\left| \dfrac{3+\sqrt{9-4m}}{4} \right|=2\left| \dfrac{3-\sqrt{9-4m}}{4} \right|$ được $m=2$ (thỏa mãn điều kiện).
\end{dapan}

\begin{cy}
	Bạn đọc cũng có thể giải thích $\dfrac{MB}{MA}=\dfrac{BK}{AH}$ bằng định lý Ta-let.
\end{cy}
%9
\begin{vd}[Trích đề thi thử THCS Ngô Sĩ Liên năm học 2017 - 2018]
	Trong mặt phẳng tọa độ $Oxy$, cho Parabol $(P)\colon y={{x}^{2}}$ và đường thẳng $(d)\colon y=2x+3$. Gọi $A,B$ là giao điểm của $\left( d \right)$ và $\left( P \right).$\\
	a) Tìm tọa độ $A,B$.\\
	b) Tính diện tích tam giác $OAB.$ \\
	c) Lấy $C$ thuộc Parabol $\left( P \right)$ có hoành độ bằng 2. Tính diện tích tam giác $ABC.$
	\loigiai{
		a) Tọa độ hai giao điểm của $\left( d \right)$ và $\left( P \right)$ là $A\left( -1;1 \right)$ và $B\left( 3;9 \right)$.
		\immini{
			b) Nhìn vào hình vẽ ta thấy,\\
			 ${{S}_{AOB}}={{S}_{AHGB}}-{{S}_{AHO}}-{{S}_{GBO}}$.\\
			${{S}_{AHGB}}=\dfrac{\left( 1+9 \right).4}{2}=20$ (đvdt).\\
			${{S}_{AHO}}=\dfrac{1}{2}.1.1=0{,}5$ (đvdt).\\
			${{S}_{BGO}}=\dfrac{1}{2}.3.9=13{,}5$ (đvdt).\\
			${{S}_{AOB}}=20-0{,}5-13{,}5=6$ (đvdt).\\
			Cách 2. Gọi I là giao điểm của $\left( d \right)$ và trục $Oy$. $${{S}_{AOB}}={{S}_{AOI}}+{{S}_{BOI}}$$
		}
		{
			\includegraphics[width=4cm]{1.Hinh_anh/CD_1/Vd9_1.pdf}
		}
		Bạn đọc tự làm. Thông thường, khi bài yêu cầu tính diện tích tam giác $AOB$, ta nên lựa chọn cách 2.
		\immini{
		c) Ta có $C\left( 2;4 \right)$. \\
		Ta thấy ${{S}_{ABC}}={{S}_{AHGB}}-{{S}_{AHEC}}-{{S}_{CEGB}}$.\\
		${{S}_{AHGB}}=20$ (đvdt).\\
		${{S}_{AHEC}}=\dfrac{\left( 1+4 \right).3}{2}=7{,}5$ (đvdt).\\
		${{S}_{CEGB}}=\dfrac{4+9}{2}=6{,}5$ (đvdt).\\
		Vậy ${{S}_{ABC}}=20-7{,}5-6{,}5=6$ (đvdt).
		}
		{
			\includegraphics[width=4cm]{1.Hinh_anh/CD_1/Vd9_2.pdf}
		}
	}
\end{vd}

\begin{nx}
	Đối với các bài toán tính diện tích tam giác trên mặt phẳng tọa độ, nếu tam giác đó không là tam giác vuông thì ta thường tính diện tích gián tiếp thông qua tổng hoặc hiệu các diện tích các hình khác.
\end{nx}
%10
\begin{vd}
	Cho Parabol $( P )\colon y=0,5{{x}^{2}}$ và đường thẳng $(d)\colon y=mx+1$ ($m$ là tham số). \\
	a) Tìm $m$ để $\left( d \right)$ cắt $\left( P \right)$ tại hai điểm phân biệt A, B sao cho diện tích của tam giác OAB bằng 2 (đvdt).\\
	b) Tìm m để diện tích tam giác OAB bé nhất.
	\loigiai{
		a) Xét phương trình hoành độ $0{,}5{{x}^{2}}=mx+1\Leftrightarrow 0{,}5{{x}^{2}}-mx-1=0$ (*).\\
		Dễ thấy phương trình (*) có $ac=-0,5<0$, luôn có hai nghiệm phân biệt.\\ Vậy $\left( d \right)$ luôn cắt $\left( P \right)$ tại hai điểm phân biệt với mọi $m$.\\
		$A\left( {{x}_{A}};{{y}_{A}} \right);B\left( {{x}_{B}};{{y}_{B}} \right)$. Các hoành độ ${{x}_{A}},{{x}_{B}}$ là nghiệm của (*).		
		\immini{
		
		Gọi I là giao điểm của $\left( d \right)$ với trục $Oy$. $I\left( 0;1 \right)$.\\
		Tương tự Ví dụ 1, %\ref{vd7}, 
		ta có\\
		$\begin{aligned}
			{{S}_{OAI}}&=\dfrac{1}{2}.\left| {{x}_{A}} \right|.1=\dfrac{1}{2}\left| {{x}_{A}} \right|.\\
			{{S}_{OBI}}&=\dfrac{1}{2}\left| {{x}_{B}} \right|.1=\dfrac{1}{2}\left| {{x}_{B}} \right|
		\end{aligned}$ 
		\\ 
		$
			{{S}_{OAB}}={{S}_{OAI}}+{{S}_{OBI}}  =\dfrac{1}{2}\left( \left| {{x}_{A}} \right|+\left| {{x}_{B}} \right| \right). 
		$\\
		Áp dụng Viet từ phương trình (*), ta có\\
		$\begin{aligned}
			&{{\left( \left| {{x}_{A}} \right|+\left| {{x}_{B}} \right| \right)}^{2}}\\
			&={{\left( {{x}_{A}}+{{x}_{B}} \right)}^{2}}-2{{x}_{A}}.{{x}_{B}}
			+2\left| {{x}_{A}}.{{x}_{B}} \right|\\
			&=4{{m}^{2}}+4+4=4\left( {{m}^{2}}+2 \right)\\
			&\Rightarrow \left| {{x}_{A}} \right|+\left| {{x}_{B}} \right|=2\sqrt{{{m}^{2}}+2}\Rightarrow {{S}_{AOB}}=\sqrt{{{m}^{2}}+2}
		\end{aligned}$
		}
		{
			\includegraphics[width=5.5cm]{1.Hinh_anh/CD_1/Vd10.pdf}
		}\\		
		+ Giải $\sqrt{{{m}^{2}}+2}=2\Leftrightarrow {{m}^{2}}=2\Leftrightarrow \left[ \begin{aligned}
			& m=\sqrt{2} \\ 
			& m=-\sqrt{2}.\\ 
		\end{aligned} \right.$\\
		Vậy với $m=\pm \sqrt{2}$ thì diện tích tam giác $OAB$ bằng $2$ (đvdt).\\
		b) Theo câu a) ${{S}_{OAB}}=\sqrt{{{m}^{2}}+2}\ge \sqrt{2}$ (đvdt) với mọi $m.$ 
		Dấu bằng xảy ra khi và chỉ khi $m=0.$
		Vậy diện tích tam giác OAB bé nhất bằng $\sqrt{2}$ (đvdt) khi $m=0.$ 
		
	}
\end{vd}
%11
\begin{vd}
	Cho Parabol $(P)\colon y={{x}^{2}}$ và đường thẳng $(d)\colon y=x+2$ ($m$ là tham số). 
	
	a) Tìm tọa độ giao điểm $A$, $B$ của $\left( d \right)$ và $\left( P \right)$.
	
	b) Tìm tọa $C$ trên trục $Oy$  để diện tích tam giác $ABC$ bằng $4{,}5$  (đvdt).
	
	c) Tìm tọa độ M thuộc cung $AB$ của $\left( P \right)$ sao cho tam giác MAB có diện tích lớn nhất.
	\loigiai{
		a) $A\left( -1;1 \right);B\left( 2;4 \right).$ \\
		b) Gọi I là giao điểm của $\left( d \right)$ với trục $Oy$, tìm được $\left( 0;2 \right)$.\\
		Ta có ${{S}_{ABC}}={{S}_{ACI}}+{{S}_{BCI}}=\dfrac{1}{2}\left( AH+BK \right).IC$.
		\\
		Có $AH=1$, $BK=2$, ${{S}_{ABC}}=4,5$ (đvdt).\\ 
		Thay vào được $IC=3$ hay $\left| {{y}_{C}}-2 \right|=3\Leftrightarrow \left[ \begin{aligned}
			& {{y}_{C}}=5 \\ 
			& {{y}_{C}}=-1. \\ 
		\end{aligned} \right.$ \\
		Vậy $C\left( 0,-5 \right)$ hoặc $C\left( 0,-1 \right)$.
		\immini{
			c) $A$ và $B$ là hai điểm cố định vậy độ dài $AB$ không đổi.
			Diện tích tam giác $AMB$ lớn nhất khi và chỉ khi chiều cao kẻ từ $M$ đến $AB$ lớn nhất hay khoảng cách từ $M$ đến $\left( d \right)$ lớn nhất.\\
			Khi đó $M$ là tiếp điểm của đường thẳng $\left( {{d}_{1}} \right)$ song song với $\left( d \right)$  và $\left( {{d}_{1}} \right)$ tiếp xúc với $\left( P \right)$.	
			\\
			Gọi phương trình $( {{d}_{1}})\colon y=ax+b$.\\
			Do ${{d}_{1}}\parallel d$ nên $\left\{ \begin{aligned}
				& a=1 \\ 
				& b\ne 3 \\ 
			\end{aligned} \right.\Rightarrow \left( {{d}_{1}} \right):y=x+b.$	
		}
		{
			\includegraphics[width=4cm]{1.Hinh_anh/CD_1/Vd11.pdf}
		}
		Xét phương trình hoành độ ${{x}^{2}}=x+b\Leftrightarrow {{x}^{2}}+x-b=0\text{  (*)}.$ \\
		$\left( {{d}_{1}} \right)$ tiếp xúc với $\left( P \right)$ khi phương trình (*) có nghiệm kép hay $$\Delta =1+4b=0\Leftrightarrow b=-\dfrac{1}{4}.$$ 
		Vậy với $b=-\dfrac{1}{4}$ thì diện tích tam giác $MAB$ đạt giá trị lớn nhất.		
	}
\end{vd}

%%%===============
\newpage
\baitapvenha
\hienthidapan
%1
\begin{bt}[Đề thi HK.II Quận Thanh Xuân năm học 2016 - 2017]
	Trong mặt phẳng tọa độ $Oxy$ cho đường thẳng $(d)\colon y=mx-1$ $(m\ne 0)$ và parabol $(P)\colon y=-{{x}^{2}}$ .
	\\
	a) Chứng minh $\left( d \right)$ luôn cắt $\left( P \right)$ tại hai điểm phân biệt.
	\\
	b) Tìm ${{x}_{1}};{{x}_{2}}$ là hoành độ giảo điểm của $\left( d \right)$ và $\left( P \right)$. Tìm $m$ sao cho $x_{1}^{2}+x_{2}^{2}=6.$
	\loigiai{
		a) Xét phương trình hoành độ $-{{x}^{2}}=mx-1\Leftrightarrow {{x}^{2}}+mx-1=0$ (*) .\\
		Có $\Delta ={{m}^{2}}+4>0$ với mọi $m$.\\
		$\Rightarrow $ Phương trình (*) luôn có  hai nghiệm phân biệt.\\
		$\Rightarrow \left( d \right)$ luôn cắt $\left( P \right)$ tại hai điểm phân biệt.
		\\
		b) Từ (*) và ý (a) suy ra $\heva{&{{x}_{1}}+{{x}_{2}}=-m\\&{{x}_{1}}.{{x}_{2}}=-1}$.\\
		Có  $x_{1}^{2}+x_{2}^{2}=6.\Leftrightarrow {{\left( {{x}_{1}}+{{x}_{2}} \right)}^{2}}-2{{x}_{1}}.{{x}_{2}}=6$.
		$\Rightarrow {{m}^{2}}+2=6\Leftrightarrow \left[ \begin{aligned}
			& m=-2 \\ 
			& m=2 \\ 
		\end{aligned} \right.$ .	
	}
\end{bt}

%2
\begin{bt}[]
	Trong mặt phẳng toạ độ $Oxy$, cho đường thẳng $(d)\colon y=2bx+1$ và parabol $(P)\colon y=-2{{x}^{2}}$ 
	
	a) Tìm $b$ để để đường thẳng $(d)$ đi qua $B\left( 1;5 \right)$ 
	
	b) Tìm $b$ để đường thẳng $(d)$ cắt parabol $\left( P \right)$ tại hai điểm phân biệt có hoành độ thoả mãn điều kiện ${{x}_{1}}^{2}+{{x}_{2}}^{2}+4{{\left( {{x}_{1}}+x \right)}_{2}}=0$.
	\loigiai{
		a) Để đường thẳng $(d)$ đi qua điểm $B\left( 1,5 \right)$ 
		$\Leftrightarrow 5=2.b.1+1\Leftrightarrow 2b=4\Rightarrow b=2$.\\
		Vậy khi $b=2$ thì đường thẳng $(d)$ đi qua điểm $B\left( 1,5 \right)$.
		\\
		b) Để đường thẳng $(d)$ cắt parabol $(P)$ tại hai điểm phân biệt.\\
		$\Rightarrow $ Phương trình $-2{{x}^{2}}=2bx+1$ có hai nghiệm phân biệt\\
		$\Leftrightarrow 2{{x}^{2}}+2bx+1=0$ có hai nghiệm phân biệt.\\
		$\Rightarrow {{\Delta }^{'}}={{b}^{2}}-2>0\Rightarrow \left| b \right|>\sqrt{2}$.\\
		Vì $2{{x}^{2}}+2bx+1=0\Rightarrow 2{{x}^{2}}=-2bx-1$. \\
		$\Rightarrow\left\{ \begin{aligned}
			& 2{{x}_{1}}^{2}=-2b{{x}_{1}}-1 \\ 
			& 2{{x}_{2}}^{2}=-2b{{x}_{2}}-1 \\ 
		\end{aligned} \right.\Rightarrow 2\left( {{x}_{1}}^{2}+{{x}_{2}}^{2} \right)=-2b\left( {{x}_{1}}+{{x}_{2}} \right)-2$.\\
		Từ $\begin{aligned}[t]
			&{{x}_{1}}^{2}+{{x}_{2}}^{2}+4({{x}_{1}}+{{x}_{2}})=0
			\Leftrightarrow  2{{x}_{1}}^{2}+2{{x}_{2}}^{2}+8\left( {{x}_{1}}+{{x}_{2}} \right)=0\\
			\Leftrightarrow & -2b\left( {{x}_{1}}+{{x}_{2}} \right)-2+8\left( {{x}_{1}}+{{x}_{2}} \right)=0
			\Leftrightarrow  \left( 8-2b \right)\left( {{x}_{1}}+{{x}_{2}} \right)-2=0\\
			\Leftrightarrow & \left( 4-b \right)\left( {{x}_{1}}+{{x}_{2}} \right)-1=0
		\end{aligned}$\\
		Theo định lý Viet  ${{x}_{1}}+{{x}_{2}}=\dfrac{-2b}{2}=-b$ 
		$\Rightarrow \left( 4-b \right)\left( -b \right)-1=0\Leftrightarrow -4b+{{b}^{2}}-1=0$ \\
		$\Leftrightarrow {{b}^{2}}-4b-1=0$ \\
		${{\Delta }^{'}}=4-\left( -1 \right)=5$ \\
		$\Rightarrow {{b}_{1}}=2-\sqrt{5},{{b}_{2}}=2+\sqrt{5}$.\\
		Kết luận $b=2-\sqrt{5},b=2+\sqrt{5}$ thoả mãn điều kiện bài toán.
	}
\end{bt}
%3
\begin{bt}
	Trong mặt phẳng $Oxy$ cho đường thẳng $(d)\colon y=x+n-1$ và parabol $(P)\colon y={{x}^{2}}$.
	
	a) Tìm $n$ để $\left( d \right)$ đi qua điểm $B\left( 0,2 \right)$.
	
	b) Tìm $n$ để đường thẳng $\left( d \right)$ cắt parabol $\left( P \right)$ tại hai điểm phân biệt có hoành độ lần lượt là ${{x}_{1}},{{x}_{2}}$ thoả mãn 
	$4\left( \dfrac{1}{{{x}_{1}}}+\dfrac{1}{{{x}_{2}}} \right)-{{x}_{1}}{{x}_{2}}+3=0$ 
	\loigiai{
		a) Để $\left( d \right)$ đi qua điểm $B\left( 0,2 \right)$ 
		$\Rightarrow 2=0+n-1\Rightarrow n-1=2\Rightarrow n=3$.\\
		Vậy $n=3$ thì $\left( d \right)$ đi qua điểm $B\left( 0,2 \right).$\\ 
		b) Hoành độ giao điểm $\left( d \right)$ và $\left( P \right)$ thoả mãn phương trình
		\\
		${{x}^{2}}=x+\left( n-1 \right)\Leftrightarrow {{x}^{2}}-x-n+1=0$ \\
		$\Delta =1-4\left( -n+1 \right)=1+4n-4=4n-3>0\Rightarrow n>\dfrac{3}{4}$.
		\\
		Để $\left( d \right)$ cắt $\left( P \right)$ tại hai điểm phân biệt có hoành độ ${{x}_{1}},{{x}_{2}}\ne 0$ \\
		$\Rightarrow \left\{ \begin{aligned}
			& \Delta >0 \\ 
			& {{x}_{1}}{{x}_{2}}\ne 0 \\ 
		\end{aligned} \right.\Leftrightarrow \left\{ \begin{aligned}
			& n>\dfrac{3}{4} \\ 
			& {{x}_{1}}{{x}_{2}}=n-1\ne 0 \\ 
		\end{aligned} \right.\Rightarrow \left\{ \begin{aligned}
			& n>\dfrac{3}{4} \\ 
			& n\ne 1. \\ 
		\end{aligned} \right.$\\ 
		Theo định lý Viet $\left\{ \begin{aligned}
			& {{x}_{1}}+{{x}_{2}}=1 \\ 
			& {{x}_{1}}{{x}_{2}}=1-n \\ 
		\end{aligned} \right.$ \\ 
		$\begin{aligned}
			&4\left( \dfrac{1}{{{x}_{1}}}+\dfrac{1}{{{x}_{2}}} \right)-{{x}_{1}}{{x}_{2}}+3=0 \Leftrightarrow 4\left( \dfrac{{{x}_{2}}}{{{x}_{1}}{{x}_{2}}}+\dfrac{{{x}_{1}}}{{{x}_{1}}{{x}_{2}}} \right)-{{x}_{1}}{{x}_{2}}+3=0\\
			\Leftrightarrow 
			&\dfrac{4\left( {{x}_{1}}+{{x}_{2}} \right)}{{{x}_{1}}{{x}_{2}}}-{{x}_{1}}{{x}_{2}}+3=0
			\Leftrightarrow \dfrac{4}{1-n}-\left( 1-n \right)+3=0\\
			\Leftrightarrow 
			&{{n}^{2}}+n-6=0 \Leftrightarrow 
			\left[ \begin{aligned}
				& n=-3 \\ 
				& n=2. \\ 
			\end{aligned}\right.
		\end{aligned}
		$\\ 
		Kết hợp $\left\{ \begin{aligned}
			& n>\dfrac{3}{4} \\ 
			& n\ne 1 \\ 
		\end{aligned} \right.\Rightarrow n=2$ thoả mãn điều kiện bài toán.
	}
\end{bt}
%4
\begin{bt}[]
	Trong mặt phẳng toạ độ $Oxy$ cho đường thẳng $(d)\colon y=mx-3$ tham số $m$ và Parabol $(P)\colon y={{x}^{2}}$ 
	
	a) Tìm $m$ để đường thẳng $(d)$ đi qua điểm $A\left( 1,0 \right)$. 
	
	b) Tìm $m$ để đường thẳng $\left( d \right)$ cắt Parabol $\left( P \right)$ tại hai điểm phân biệt có hoành độ là ${{x}_{1}},{{x}_{2}}$ thoả mãn $\left| {{x}_{1}}-{{x}_{2}} \right|=2$ .
	
	\loigiai{
		a) Để đường thẳng $\left( d \right)$ đi qua $A\left( 1,0 \right)$ $\Rightarrow 0=m.1-3\Rightarrow m-3=0\Rightarrow m=3$.\\
		Vậy khi $m=3$ thì đường thẳng $\left( d \right)$ đi qua điểm $A\left( 1,0 \right)$.
		\\
		b) Để đường thẳng $\left( d \right)$ cắt parabol $\left( P \right)$ tại hai điểm phân biệt\\
		$\Rightarrow $ Phương trình ${{x}^{2}}=mx-3$ có hai nghiệm phân biệt\\
		$\Leftrightarrow {{x}^{2}}-mx+3=0$ \\
		$\Delta ={{m}^{2}}-4.3={{m}^{2}}-12>0\Rightarrow {{m}^{2}}>12\Rightarrow \left| m \right|>2\sqrt{3}$.\\
		Theo công thức tính nghiệm\\
		${{x}_{1}}=\dfrac{-b+\sqrt{\Delta }}{2a},{{x}_{2}}=\dfrac{-b-\sqrt{\Delta }}{2a}$  
		$\left| {{x}_{1}}-{{x}_{2}} \right|=\left| \dfrac{-b+\sqrt{\Delta }}{2a}-\dfrac{-b-\sqrt{\Delta }}{2a} \right|=\dfrac{\sqrt{\Delta }}{\left| a \right|}=2$ \\
		$\Leftrightarrow \dfrac{\sqrt{{{m}^{2}}-12}}{1}=2\Rightarrow \sqrt{{{m}^{2}}-12}=2\Rightarrow {{m}^{2}}-12=4\Rightarrow {{m}^{2}}=16\Rightarrow m=\pm 4$.\\
		So sánh với điều kiện thì $m=\pm 4.$ 
	}
\end{bt}
%5
\begin{bt}[Đề thi thử Trường THCS Cầu Giấy năm học 2017-2018]
	Trong mặt phẳng toạ độ $Oxy$ cho đường thẳng $(d)\colon y=2\left( m-2 \right)x-4m+13$ tham số $m$ và Parabol $(P)\colon y={{x}^{2}}$.
	\\
	a) Với $m=4$, trên cùng một hệ tọa độ $Oxy$, vẽ $\left( P \right)$ và $\left( d \right)$. Xác định tọa độ giao điểm $A,B$.
	 \\
	b) Tìm $m$ để đường thẳng $\left( d \right)$ cắt Parabol $\left( P \right)$ tại hai điểm phân biệt có hoành độ là ${{x}_{1}},{{x}_{2}}$ sao cho biểu thức $S=x_{1}^{2}+x_{2}^{2}+4{{x}_{1}}{{x}_{2}}+2018$ đạt giá trị nhỏ nhất.
	
	\loigiai{
		Phương trình hoành độ giao điểm ${{x}^{2}}-2\left( m-2 \right)x+4m-13=0$.
		\\
		a) $A\left( 1;1 \right);B\left( 3;9 \right)$. Bạn đọc tự vẽ hình.
		\\
		b) Có $S=4{{\left( m-2 \right)}^{2}}+8m-26+2018$.
		Tìm được ${{S}_{\min }}=2004\Leftrightarrow m=1.$ 
	}
\end{bt}
%6
\begin{bt}
	Trong mặt phẳng toạ độ $Oxy$ cho đường thẳng $(d)\colon y=mx+m+1$ tham số $m$ và Parabol $(P)\colon y={{x}^{2}}$ 
	
	a) Với giá trị nào của $m$ thì $\left( d \right)$ tiếp xúc với $\left( P \right)$. Tìm tọa độ tiếp điểm. 
	
	b) Tìm $m$ để đường thẳng $\left( d \right)$ cắt Parabol $\left( P \right)$ tại hai điểm phân biệt nằm khác phía với trục tung, có hoành độ là ${{x}_{1}},{{x}_{2}}$ thoả mãn $2{{x}_{1}}-3{{x}_{2}}=5$ .
	\loigiai{
		Phương trình hoành độ giao điểm của $\left( P \right)$ và $\left( d \right)$ là
		
		\centerline{${{x}^{2}}-mx-m-1=0$ (*); $\Delta ={{\left( m+2 \right)}^{2}}$.}
		\noindent a) $\left( d \right)$ tiếp xúc với $\left( P \right)$ khi và chỉ khi $\Delta =0\Leftrightarrow m=-2.$ \\
		Tìm được tọa độ tiếp điểm $A\left( -1;1 \right)$.\\
		b) Dễ thấy phương trình (*) có nghiệm $x=-1;$ $x=m+1$.
		Điều kiện để đường thẳng $\left( d \right)$ cắt Parabol $\left( P \right)$ tại hai điểm phân biệt nằm khác phía với trục tung là $m+1>0\Leftrightarrow m>-1$.\\ 
		+ Trường hợp 1 $\left\{ \begin{aligned}
			& {{x}_{1}}=-1 \\ 
			& {{x}_{2}}=1+m \\ 
			& 2{{x}_{1}}-3{{x}_{2}}=5 \\ 
		\end{aligned} \right.\Rightarrow m=-\dfrac{10}{3}$ (loại).\\
		+ Trường hợp 2 $\left\{ \begin{aligned}
			& {{x}_{1}}=1+m \\ 
			& {{x}_{2}}=-1 \\ 
			& 2{{x}_{1}}-3{{x}_{2}}=5 \\ 
		\end{aligned} \right.\Rightarrow m=-0$ (thỏa mãn điều kiện).
		
	}
\end{bt}
%7
\begin{bt}[Đề thi thử THCS Nghĩa Tân năm học 2017 - 2018]
	Trong mặt phẳng toạ độ $Oxy$ cho đường thẳng $(d)\colon y=\left( m+2 \right)x-2m$ tham số $m$ và Parabol $(P)\colon y={{x}^{2}}$.
	
	a) Tìm $m$ để đường thẳng $(d)$ cắt Parabol $\left( P \right)$ tại hai điểm phân biệt $A$ và $B$. 
	
	b) Gọi hoành độ của $A$ và $B$  là ${{x}_{1}},{{x}_{2}}$. Tìm $m$ để  $x_{1}^{2}+\left( m+2 \right){{x}_{2}}=12$ .
	
	\loigiai{\noindent
		a) Phương trình hoành độ giao điểm ${{x}^{2}}-\left( m+2 \right)x+2m=0$ có $\Delta ={{\left( m-2 \right)}^{2}}.$ 
		$(d)$ cắt $\left( P \right)$ tại hai điểm phân biệt $A$ và $B$ khi và chỉ khi $m\ne 2.$ 
		\\
		b) Áp dụng Vi-et có $\begin{aligned}[t]
			&\heva{& {{x}_{1}}+{{x}_{2}}=m+2 \\ 	& {{x}_{1}}.{{x}_{2}}=2m }\\
			&\Rightarrow x_{1}^{2}+\left( {{x}_{1}}+{{x}_{2}} \right){{x}_{2}}=12\Leftrightarrow x_{1}^{2}+x_{2}^{2}+{{x}_{1}}{{x}_{2}}-12=0\\
			&\Leftrightarrow {{\left( {{x}_{1}}+{{x}_{2}} \right)}^{2}}-{{x}_{1}}{{x}_{2}}-12=0\Rightarrow {{\left( m+2 \right)}^{2}}-2m-12=0.
		\end{aligned}$ \\
		Giải ra tìm được $m=-4$ (thỏa mãn); $m=2$  (loại).
	}
\end{bt}
%8
\begin{bt}[Đề thi HK.II Quận Đống Đa năm học 2016 - 2017]
	Trong mặt phẳng toạ độ $Oxy$ cho đường thẳng $(d)\colon y=\left( 2m+1 \right)x-2m$ tham số $m$ và parabol $(P)\colon y={{x}^{2}}$. 
	
	a) Xác định tọa độ giao điểm của $\left( d \right)$ và $\left( P \right)$ khi $m=1$.
	
	b) Tìm $m$ để đường thẳng $\left( d \right)$ cắt parabol $\left( P \right)$ tại hai điểm phân biệt có lần lượt hoành độ là ${{x}_{1}},{{x}_{2}}$ thoả mãn ${{y}_{1}}+{{y}_{2}}-{{x}_{1}}{{x}_{2}}=1.$
	
	\loigiai{
		Phương trình hoành độ giao điểm của $\left( d \right)$ và $\left( P \right)$ là
		${{x}^{2}}-\left( 2m+1 \right)x+2m=0$ (*)
		\\
		a) Với $m=1$, phương trình (*) trở thành ${{x}^{2}}-3x+2=0$.\\
		Giải ra ta tìm được $x=1;x=2$. Các tung độ tương ứng là $y=1;y=4$.
		\\
		b) Từ phương trình (*) suy ra $\left( d \right)$ cắt $\left( P \right)$ tại hai điểm phân biệt khi và chỉ khi $\Delta ={{\left( 2m-1 \right)}^{2}}>0\Leftrightarrow m\ne \dfrac{1}{2}.$ \\
		Tung độ $y$ dùng một trong 2 công thức $\left\{ \begin{aligned}
			& y={{x}^{2}} \\ 
			& y=\left( 2m+1 \right)x-2m. \\ 
		\end{aligned} \right.$\\
		Rõ ràng, $y={{x}^{2}}$ dễ chịu hơn.\\
		Ta có ${{y}_{1}}+{{y}_{2}}-{{x}_{1}}{{x}_{2}}=1\Leftrightarrow x_{1}^{2}+x_{2}^{2}-{{x}_{1}}{{x}_{2}}=1\Leftrightarrow {{\left( {{x}_{1}}+{{x}_{2}} \right)}^{2}}-3{{x}_{1}}{{x}_{2}}-1=0$.\\
		Áp dụng Vi-et có ${{\left( 2m+1 \right)}^{2}}-6m-1=0\Leftrightarrow \left[ \begin{aligned}
			& m=0 \\ 
			& m=\dfrac{1}{2} \\ 
		\end{aligned} \right..$ \\
		Đối chiếu với điều kiện chọn $m=0.$ 
		
	}
\end{bt}
%9
\begin{bt}
	Trong mặt phẳng toạ độ $Oxy$ cho đường thẳng $(d)\colon y=2mx-{{m}^{2}}+1$ tham số $m$ và Parabol $(P)\colon y={{x}^{2}}$
	 
	a) Chứng minh rằng đường thẳng $d$ luôn cắt Parabol $\left( P \right)$ tại hai điểm phân biệt với mọi giá trị của $m$. Tìm tọa độ giao điểm của $(d)$ và $\left( P \right)$ khi $m=3.$ 
	 
	b) Tìm $m$ để đường thẳng $d$ cắt Parabol $\left( P \right)$ tại hai điểm $A\left( {{x}_{1}};{{y}_{1}} \right)$ và $B\left( {{x}_{2}};{{y}_{2}} \right)$ có tung độ thỏa mãn ${{y}_{1}}-{{y}_{2}}>4.$ 
	
	\loigiai{\noindent
		a) Phương trình hoành độ ${{x}^{2}}-2mx+{{m}^{2}}-1=0$, có ${{\Delta }^{'}}=1>0$.\\
		$\Rightarrow $ Phương trình luôn có hai nghiệm phân biệt với mọi m.\\
		$\Rightarrow $ Đường thẳng $d$ luôn cắt Parabol $\left( P \right)$ tại hai điểm phân biệt.\\
		Khi $m=3$, tọa độ hai giao điểm là $\left( 2;4 \right)$ và $\left( 4,16 \right)$.
		\\
		b) Tìm được $\left( {{x}_{1}};{{x}_{2}} \right)=\left( 1-m;1+m \right)$ hoặc $\left( {{x}_{1}};{{x}_{2}} \right)=\left( 1+m;1-m \right)$\\
		${{y}_{1}}-{{y}_{2}}=x_{1}^{2}-x_{2}^{2}>4$.
		\\
		+ Trường hợp 1 ${{\left( 1-m \right)}^{2}}-{{\left( 1+m \right)}^{2}}>4\Leftrightarrow m<-1$.
		\\
		+ Trường hợp 2 ${{\left( 1+m \right)}^{2}}-{{\left( 1-m \right)}^{2}}>4\Leftrightarrow m>1$.
		\\
		Kết luận $m<-1$ hoặc $m>1$ .	
	}
\end{bt}
%10
\begin{bt}[Đề thi thử Trường THCS Giảng Võ năm học 2017-2018]
	Trong mặt phẳng toạ độ $Oxy$ cho đường thẳng $(d)\colon y=2x+3$ tham số $m$ và Parabol $(P)\colon y={{x}^{2}}$.\\
	a) Tìm tọa độ giao điểm của $(d)$ và $\left( P \right)$. \\
	b) Gọi $A,B$ là giao điểm của $(d)$ và $\left( P \right)$; $C$ là điểm thuộc $\left( P \right)$ có hoành độ bằng $\dfrac{3}{2}$. Tính diện tích tam giác $ABC.$ 
	\loigiai{
	\noindent	a) Giao điểm của $(d)$ và $\left( P \right)$ có tọa độ $\left( -1;1 \right);\left( 3;9 \right)$.\\
		b) Từ $C$ kẻ đường thẳng song song với $Ox$ cắt $(d)$ tại $M$.\\
		Có $S_{ACM}+ S_{BCM}= S_{ABC}$.
		Từ đây ta có ${{S}_{ABC}}=7{,}5$  (đvdt).	
	}
\end{bt}
%11
\begin{bt}[Đề thi thử THCS Cổ Loa năm học 2018-2019]
	Trong mặt phẳng toạ độ $Oxy$ cho đường thẳng $(d)\colon y=2\left( m-3 \right)x+4$ tham số $m$ và Parabol $(P)\colon y={{x}^{2}}$ 
	
	a) Chứng minh rằng đường thẳng $(d)$ luôn cắt Parabol $\left( P \right)$ tại hai điểm phân biệt $A$ và $B$ với mọi giá trị của $m$.
	
	b) Gọi $I$ là giao điểm của $\left( d \right)$ với trục $Oy$.\\
	Tìm $m$ để $A$ và $B$ đối xứng với nhau qua $I$.
	
	\loigiai{\noindent
		a) Xét phương trình hoành độ ${{x}^{2}}-2\left( m-3 \right)x-4=0$ với $ac=-4<0$.\\
		Phương trình luôn có hai nghiệm phân biệt trái dấu hay $\left( d \right)$ và $\left( P \right)$ luôn cắt nhau tại hai điểm phân biệt nằm về hai phía của trục tung.
		
		\noindent b) $A,B$ đối xứng với nhau qua $I$ tức $I$ là trung điểm của $AB$.
		
		\chenhinh[-0.2]
		{
			\noindent 
			Kẻ các chiều cao $AH$ và $BK$ (hình vẽ).\\
			Áp dụng định lý Ta-let có $\dfrac{AH}{BK}=\dfrac{AI}{IB}=1$\\ 
			hay $\dfrac{\left| {{x}_{A}} \right|}{\left| {{x}_{B}} \right|}=1\Leftrightarrow \left| {{x}_{A}} \right|=\left| {{x}_{B}} \right|$.\\
			Mà theo Vi-et, ${{x}_{A}}.{{x}_{B}}=-4$\\
			$\Rightarrow \left\{ \begin{aligned}
				& {{x}_{A}}=2 \\ 
				& {{x}_{B}}=-2 \\ 
			\end{aligned} \right.$ hoặc $\left\{ \begin{aligned}
				& {{x}_{A}}=-2 \\ 
				& {{x}_{B}}=2. \\ 
			\end{aligned} \right.$ \\
			${{x}_{A}}+{{x}_{B}}=0\Rightarrow 2\left( m-3 \right)=0\Leftrightarrow m=3$.
		}
		{\includegraphics[width=4.2cm]{1.Hinh_anh/CD_1/Bt11.pdf}}		
	}
\end{bt}
%12
\begin{bt}[Đề thi tuyển sinh vào lớp 10 Hà Nội năm học 2019-2020]
	Trong mặt phẳng toạ độ $Oxy$ cho đường thẳng $(d)\colon y=2mx-{{m}^{2}}+1$ tham số $m$ và Parabol $(P)\colon y={{x}^{2}}$.
	
	a) Chứng minh $\left( d \right)$ luôn cắt $\left( P \right)$ tại hai điểm phân biệt.
	
	b) Tìm tất cả giá trị của m để $\left( d \right)$ cắt $\left( P \right)$tại hai điểm phân biệt có hoành độ ${{x}_{1}}$; ${{x}_{2}}$ thỏa mãn $\dfrac{1}{{{x}_{1}}}+\dfrac{1}{{{x}_{2}}}=\dfrac{-2}{{{x}_{1}}{{x}_{2}}}+1$
	
	\dapso{
		a) Bạn đọc tự chứng minh. \quad
		b)  $m=3$.
	
	}
\end{bt}
%13
\begin{bt}[Trích đề thi thử vào lớp 10 Trường THCS Việt Hưng, Long Biên, Hà Nội năm học 2020-2021]
	Trong mặt phẳng tọa độ $Oxy$ cho đường thẳng $(d)\colon y=mx+2$ và Parabol $(P)\colon y=\dfrac{{{x}^{2}}}{2}$.
	
	a) Chứng minh rằng $\left( d \right)$ luôn cắt $\left( P \right)$ tại hai điểm phân biệt.
	
	b) Gọi giao điểm của $\left( d \right)$ với trục tung là $G$. Gọi $H$ và $K$ lần lượt là hình chiếu của $A, B$ trên trục hoành. Tìm $m$ để diện tích tam giác $GHK$ bằng $4$.
	\dapso{
		a) Bạn đọc tự giải.\quad
		b) $m=0$.	}
\end{bt}
%14
\begin{bt}[Trích đề thi thử vào lớp 10 Trường THCS Trung Hòa, Cầu Giấy, Hà Nội năm học 2020 - 2021]
	Trong mặt phẳng tọa độ $Oxy$ cho đường thẳng $(d)\colon y=x-m+3$ và Parabol $(P)\colon y={{x}^{2}}$
	
	a) Tìm $m$ để $\left( d \right)$ luôn cắt $\left( P \right)$ tại hai điểm phân biệt.
	
	b) Tìm các giá trị của m để $\left( d \right)$] cắt $\left( P \right)$ tại hai điểm phân biệt $M\left( {{x}_{1}};{{y}_{1}} \right)$ và $N\left( {{x}_{2}};{{y}_{2}} \right)$ sao cho ${{y}_{1}}+{{y}_{2}}=3\left( {{x}_{1}}+{{x}_{2}} \right)$
	
	\dapso{
		a) $m<\dfrac{13}{4}$.\quad
		b) $m=2$.
		
	}
\end{bt}
%15
\begin{bt}[Trích đề thi thử vào lớp 10 Trường THCS Thượng Thanh, Long Biên, Hà Nội năm học 2020 - 2021]
	Tìm các giá trị của tham số m để đường thẳng $(d)\colon y=\dfrac{3}{2}x+2m-1$ cắt Parabol $(P)\colon y=-\dfrac{1}{2}{{x}^{2}}$ tại điểm khác gốc tọa độ có hoành độ gấp đôi tung độ.
%	\loigiai{
%		
%	}
\end{bt}
%16
\begin{bt}[Trích đề thi thử vào lớp 10 Trường THCS Thạch Bàn, Long Biên, Hà Nội năm học 2020 - 2021]
	Trong mặt phẳng tọa độ $Oxy$ cho đường thẳng $(d)\colon y=mx+1-m$ và Parabol $(P)\colon y={{x}^{2}}$.
	
	a) Xác định tọa độ giao điểm của (d) và (P)  khi $m=-1$.
	
	b) Tìm các giá trị của m để (d)  cắt (P) tại hai điểm phân biệt có hoành độ ${{x}_{1}}$, ${{x}_{2}}$ sao cho $\sqrt{{{x}_{1}}}+\sqrt{{{x}_{2}}}=3$.
	
%	\loigiai{
%		
%	}
\end{bt}
%17
\begin{bt}[Trích đề thi thử vào lớp 10 Trường THCS Phúc Đồng, Long Biên, Hà Nội năm học 2020 - 2021]
	Trong mặt phẳng tọa độ $Oxy$ cho đường thẳng $(d)\colon y=-mx+m+1$ và Parabol $(P)\colon y={{x}^{2}}$.
	
	a) Xác định tọa độ giao điểm của (d) và (P) khi $m=2$.
	
	b) Tìm các giá trị của m để (d)  cắt (P) tại hai điểm phân biệt có hoành độ ${{x}_{1}}$, ${{x}_{2}}$ sao cho $x_{1}^{2}+x_{2}^{2}<2$.
	
	\dapso{
		a) $\left( 1;1 \right)$ và $\left( -3;9 \right)$.\quad
		b) $-2<m<0$.
		
	}
\end{bt}

%18
\hienthidapan
\begin{bt}
	Cho prabol $(P)\colon y=x^2$ và đường thẳng $(d)\colon y=mx-2$.
	
	a) Tìm $m$ để $(d)$ cắt $(P)$ tại hai điểm phân biệt.
	
	b) Khi đó, gọi $C$ là giao điểm của $(d)$	với trục $Ox$. Gọi $A$ và $B$ là giao điểm của $(d)$ với $(P)$ ($A$ nằm giữa $B$ và $C$). Tìm $m$ để $\dfrac{AC}{AB}=\dfrac{1}{3}$.
	\loigiai{
		\noindent
		a) Tự chứng minh.
		\\
		b) Gọi $H,K$ lần lượt là hình chiếu của $A$, $B$ trên $Ox$.\\
		Áp dụng Định lý Thales để tính tỉ số $\dfrac{AC}{BC}=\dfrac{AH}{BK}=\dfrac{|y_A|}{y_B}=\dfrac{1}{4}$.\\
		Từ đó tìm $m=3$ hoặc $m= -3$.
	}
\end{bt}





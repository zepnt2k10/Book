\def\i{\item}
\graphicspath{{../pictures/vande37/}}
\chapter{HÌNH HỌC TRỰC QUAN}
\section{Chu vi và diện tích}
\subsection{Kiến thức}
\subsubsection{Chu vi, diện tích của hình vuông, hình chữ nhật, hình thang}
Ở Tiểu học các em đã học và biết sử dụng các công thức tính chu vi, diện tích của một số hình sau:
\begin{tabular}{c c c}
	Hình vuông     &          Hình chữ nhật&                      Hình thang\\
	$C=4a$& $C=2(a+b)$ & $C=a+b+c+d$\\
	$S=a\cdot a$& $S=a\cdot b$& $S=\frac{1}{2}(a+b)\cdot h$ 
\end{tabular}
\subsubsection{Chu vi, diện tích của hình bình hành, hình thoi}
\begin{tabular}{c c}
	Hình bình hành             &                       Hình thoi\\
	$C=2(a+b)$& $C=4m$\\
	$S=a\cdot h $ & $S=\frac{1}{2}ab$\\
	($a$ là cạnh, $h$ là chiều cao tương ứng) &($a,b$ là độ dài hai đường chéo)
\end{tabular}
\subsection{Thực hành giải toán}
\begin{vd}
	Tính chu vi, diện tích của:
	\begin{enumerate}[a), leftmargin=*]
		\i Hình vuông có cạnh  $3\,cm$. 
		\i Hình chữ nhật có chiều rộng là  $5 \,dm$ , chiều dài là  $60\,cm$. 
		\i Hình thang cân có cạnh bên là  $5 \,cm$ , hai cạnh đáy lần lượt là  $4\,cm$; $10\,cm$ , chiều cao là  $4 \,cm.$ 
		\i Hình bình hành có hai cạnh lần lượt là:  $5\,m$; $8\,m$ ; chiều cao ứng với cạnh đáy  $8\,m$ là $4\,m$. 
		\i Hình thoi có cạnh  $5 \,dm$ . Hai đường chéo có cạnh lần lượt  $6 \,dm$, $8\, dm.$ 
	\end{enumerate}
	\loigiai{
		Sử dụng các công thức chu vi, diện tích các hình.
		\begin{enumerate}[a), leftmargin=*]
			\i Chu vi hình vuông là:  $4.3=12 \,(cm)$.\\
			Diện tích hình vuông là:  $3.3=9 \,(cm^2)$. 
			\i Đổi  $60\, cm=6 \,dm$\\ 
			Chu vi hình chữ nhật là:  $2\cdot(5+6)=22 \,(dm)$.\\
			Diện tích hình chữ nhật là:  $5\cdot 6=30\, dm^2$.\\
			\i Chu vi hình thang cân là:  $4+10+5+5=24$.\\ 
			Diện tích của hình thang cân là:  $\frac{1}{2}(4+10)\cdot4=28 \,(cm^2)$.\\ 
			\i Chu vi hình bình hành:  $2\cdot(5+8)=26\, (m)$.\\ 
			Diện tích của hình bình hành là:  $8\cdot4=32\, (m^2)$.\\ 
			\i Chu vi hình thoi là:  $4\cdot5=20\, (cm)$.\\ 
			Diện tích hình thoi là:  $\frac{1}{2}\cdot6\cdot8=24\, (dm^2)$. 
		\end{enumerate}
	}
\end{vd}
\begin{vd}
	Một mảnh ruộng hình thang có kích thước như hình dưới. Biết năng suất lúa là  $0,8kg/{{m}^{2}}.$ 
	\begin{enumerate}[a), leftmargin=*]
		\i Tính diện tích mảnh ruộng.
		\i Hỏi mảnh ruộng cho sản lượng là bao nhiêu kilogam thóc?
	\end{enumerate}
	\loigiai{
		\begin{enumerate}[a), leftmargin=*]
			\i Diện tích mảnh ruộng
			\[(15+25)\cdot10:2=200 \,(m^2)\] 
			\i Số lượng thóc trên mảnh ruộng là: 
			\[200\cdot 0,8=160 \,(kg)\] 
		\end{enumerate}
	}
\end{vd}
\begin{vd}
	Cho tam giác đều được tạo thành từ 36 tam giác đều nhỏ như hình dưới đây. Mỗi tam giác đều nhỏ có diện tích bằng 3. Tính diện tích của tam giác $ABC$ .
	\loigiai{
		Ta đánh dấu các điểm  $D,E$ như hình vẽ.
		
		Khi đó diện tích tam giác  $ACD$  bằng một nửa diện tích hình bình hành  $ECDA.$ 
		
		Ta thấy hình bình hành  $ECDA$  được tao từ 12 tam giác nhỏ nên diện tích hình bình hành  $ECDA$ bằng diện tích của 12 tam giác nhỏ. Suy ra, diện tích tam giác được tô màu xanh bằng diện tích của 6 tam giác nhỏ.
		
		Tương tự ta có:
		\begin{enumerate}[--, leftmargin=*]
			\i Diện tích tam giác được tô màu hồng bằng diện tích của 2 tam giác nhỏ.
			\i Diện tích tam giác được tô màu cam bằng diện tích của 3 tam giác nhỏ.
			\i Diện tích tam giác  $ABC$  bằng tổng diện tích tam giác được tô màu xanh, hồng và cam. Mỗi tam giác nhỏ có diện tích bằng 3 nên diện tích tam giác  $ABC$  là:
		\end{enumerate}
		\[(6+2+3)\cdot3=33\] 
		Vậy diện tích tam giác  $ABC$  là 33.
	}
\end{vd}
\subsection{Mở rộng kiến thức}
Diện tích hình tam giác. 
Hình tam giác có đáy bằng  $a$  và chiều cao tương ứng bằng  $h$  có diện tích là:  $S=\frac{a\cdot h}{2}.$ 
\subsection{Bài tập tự luyện}
\subsubsection*{Mức độ cơ bản}
\Opensolutionfile{loigiaichung}[loigiaichuong37]
\begin{bt}
	\begin{enumerate}[a), leftmargin=*]
		\i Tính chu vi của hình bình hành có cạnh là  $4\,cm$, $5\,cm$ .
		\i Tính diện tích hình thoi  $ABCD$  có 2 đường chéo $AC=8 \,cm$, $BD=6 \,cm$. 
		\i Tính chu vi và diện tích của hình chữ nhật  $ABCD$  có  $AB=4\,cm$, $BC=6\, cm$. 
		\i Tính cạnh và diện tích hình vuông $IKMN$ biết chu vi hình vuông đó bằng  $16 \,dm$. 
		\i Tính chu vi hình thoi  $MNPQ$ biết  $MQ=7\,cm$. 
		\i Tính độ dài đường chéo hình thoi biết có một đường chéo là  $56\,cm$ và diện tích là  $56 \,dm^2$.
	\end{enumerate}
	\begin{loigiaichuong37}
		\begin{enumerate}[a), leftmargin=*]
			\i Chu vi của hình bình hành là:  $2.(4+5)=18 \,(cm).$ 
			\i Diện tích hình thoi ABCD là:  $\frac{1}{2}.6.8=24\, (cm^2)$ 
			\i Chu vi hình chữ nhật ABCD là:  $2.(4+6)=5 \,(cm).$\\
			Diện tích hình chữ nhật ABCD là:  $4.6=24\,(cm^2).$ 
			\i Cạnh hình vuông IKMN là:  $16:4=4\,(dm).$\\ 
			Diện tích hình vuông IKMN là:  $4.4=16\,(dm^2).$ 
			\i Chu vi hình thoi MNPQ là:  $4.7=28\,(cm).$ 
		\end{enumerate}
	\end{loigiaichuong37}
\end{bt}
\begin{bt}
	Một gia đình dự định mua gạch men loại hình vuông cạnh  $30\, cm$  để lát nền của căn phòng hình chữ nhật có chiều rộng  $3\, m$  chiều dài  $9\, m$ . Tính số viên gạch cần mua để lát căn phòng đó.
	\begin{loigiaichuong37}
		Đổi  $3m=300\,cm,9m=900\,cm.$
		 
		Diện tích nền của căn phòng là:  $300\cdot900=270000\,(cm^2).$
		 
		Diện tích viên gạch lát nền là:  $30\cdot30=900\,(cm^2)$
		 
		Số viên gạch dùng để lát nền của că phòng hình chữ nhật là:  $270000:900=300$  (viên)
	\end{loigiaichuong37}
\end{bt}
\begin{bt}
	Một khu vườn hình chữ nhật có chiều dài  $12 \,m$ , chiều rộng  $8\, m$  như hình dưới, cổng vào có độ rộng bằng  $\frac{1}{4}$ chiều dài, phần còn lại là hàng rào. Hỏi hàng rào của khu vườn dài bao nhiêu mét?
	\begin{loigiaichuong37}
		Độ rộng của cổng vào là:  $12\cdot\frac{1}{4}=3 \,(cm).$
		
		Chu vi mảnh vườn hình chữ nhật là:  $2\cdot(12+8)=10 (m).$ 
		
		Hàng rào của khu vườn đó dài:  $10-3=7 \,(m).$ 
	\end{loigiaichuong37}
\end{bt}
\begin{bt}
	Mặt sàn của một ngôi nhà được thiết kế như hình dưới (đơn vị  $m$ ). Hãy tính diện tích mặt sàn.
	\begin{loigiaichuong37}
		Chiều dài của ngôi nhà là:  $8+6=14\,(m)$ 
		
		Chiều rộng của ngôi nhà là:  $6+2=8\,(m)$ 
		
		Diện tích mặt cắt ngôi nhà là:  $14\cdot8=112 \,(m^2).$ 
	\end{loigiaichuong37}
\end{bt}
\begin{bt}
	Một bồn hoa hình vuông có cạnh $3 \,m$ .Người ta xây thành cho bồn hoa có độ rộng $20\,cm$. Tính diện tích để trồng hoa. 
	\begin{loigiaichuong37}
		Đổi  $3\, m=300 \,cm.$
		 
		Cạnh của mảnh vườn hình vuông để trồng hoa là:  $300-2\cdot20=260\, (cm)$ 
		
		Diện tích mảnh vườn hình vuông để trồng hoa là:  $260\cdot260=67600 \,(cm^2).$ 
	\end{loigiaichuong37}
\end{bt}
\begin{bt}
	Tính diện tích của khu đất có hình dạng dưới đây.
	\begin{loigiaichuong37}
		Diện tích khu đất tam giác $ABC$ là:  $\frac{1}{2}.25.75=937,5\,(m^2).$ 
		
		Diện tích khu đất hình thang $ACDE$ là:  $\frac{1}{2}(75+25)\cdot37=1850 \,(m^2)$
		 
		Diện tích cả khu đất $ABCDE$ là:  $937,5+1850=2787,5 \,(m^2)$ 
	\end{loigiaichuong37}
\end{bt}
\begin{bt}
	Diện tích các hình sẽ thay đổi thế nào trong các câu sau?
	\begin{enumerate}[a), leftmargin=*]
		\i Hình vuông có cạnh tăng lên 2 lần.
		\i Hình chữ nhật có chiều dài tăng lên 3 lần, chiều rộng tăng lên 2 lần.
		\i Hình chữ nhật có chiều dài tăng 4 lần, chiều rộng giảm 2 lần.
	\end{enumerate}
	\begin{loigiaichuong37}
		\begin{enumerate}[a), leftmargin=*]
			\i Diện tích hình vuông sẽ tăng lên 4 lần.
			\i Diện tích hình chữ nhật sẽ tăng lên 6 lần.
			\i Diện tích hình chữ nhật sẽ tăng lên 2 lần.
		\end{enumerate}
	\end{loigiaichuong37}
\end{bt}
\begin{bt}
	Các cạnh của các hình sẽ phải thay đổi thế nào trong các trường hợp sau:
	\begin{enumerate}[a), leftmargin=*]
		\i Hình vuông có diện tích giảm xuống 4 lần.
		\i Hình chữ nhật có diện tích tăng 8 lần và chiều rộng tăng lên 2 lần.
		\i Hình chữ nhật có diện tích tăng 2 lần và chiều dài giảm 2 lần.
	\end{enumerate}
	\begin{loigiaichuong37}
		\begin{enumerate}[a), leftmargin=*]
			\i Cạnh hình vuông sẽ giảm đi 2 lần.
			\i Chiều dài hình chữ nhật tăng lên 4 lần.
			\i Chiều rộng của hình chữ nhật tăng lên 4 lần.
		\end{enumerate}
	\end{loigiaichuong37}
\end{bt}
\begin{bt}
	Cho tam giác đều $ABC$  có diện tích bằng 36 $({{m}^{2}}).$  Các đường thẳng song song lần lượt chia các cạnh của tam giác thành ba phần bằng nhau. Tính diện tích phần màu trắng.
	\begin{loigiaichuong37}
		Ta thấy tam giác đều $ABC$ có thể chia thành 9 hình tam giác nhỏ bằng nhau 
		
		Diện tích của mỗi hình tam giác nhỏ là:  $36:9=4\,(m^2).$ 
		
		Diện tích của phần màu trắng là:  $4\cdot3=12\,(m^2).$ 
	\end{loigiaichuong37}
\end{bt} 
\begin{bt}
	Diện tích phần được tô màu hồng biết chu vi mỗi hình vuông nhỏ trong hình dưới đây bằng  $4\,cm.$ 
	\begin{loigiaichuong37}
		Diện tích của hình vuông to là:  $4\cdot4=16 \,(cm^2).$ 
		
		Mỗi cạnh của hình vuông nhỏ là:  $4:4=1\,(cm).$
		 
		Ta thấy phần được tô màu hồng chính là phần còn lại khi lấy diện tích cả hình vuông trừ 4 hình tam giác vuông màu trắng bằng nhau.
		
		Diện tích các tam giác vuông:  $\frac{1}{2}\cdot3\cdot1=\frac{3}{2}\, (cm^2).$
		 
		Diện tích của phần được tô màu hồng là:  $16-\frac{3}{2}\cdot4=10\, (cm^2).$ 
	\end{loigiaichuong37}
\end{bt}
\begin{bt}
	Cho tam giác $ABC$ có diện tích là  $150\, cm^2$. Biết nếu kéo dài cạnh đáy $BC$ thêm $5 \,cm$  thì diện tích tăng thêm $37,5\, cm^2$ Tính độ dài cạnh đáy $BC$. 
	\begin{loigiaichuong37}
		Do diện tích tăng thêm là  $37,5 \,cm^2$ nên ta suy ra diện tích $ACD$ là  $37,5 cm^2$. Gọi $AH$ là đường cao của tam giác $ACD$
		
		Độ dài $AH$ là:  $37,5\cdot2:5=15\,(cm).$
		 
		Do tam giác $ABC$ và tam giác $ACD$ có chung đường cao $AH$ nên suy ra:
		
		Độ dài $BC$ là:  $150\cdot2:15=20 \,(cm).$ 
	\end{loigiaichuong37}
\end{bt}
\subsubsection*{Mức độ nâng cao}
\begin{bt}
	Hai hình vuông $ABCD$ và $EFGH$ được lồng vào nhau sao cho $AB$ song song với $EF$ như hình vẽ dưới đây:
	
	Phần tô xám có diện tích bằng 1. Hỏi diện tích của hình vuông là bao nhiêu?
	\begin{loigiaichuong37}
		Nối thêm các đường như hình vẽ 
		
		Khi đó ta có:  ${{S}_{ABE}}={{S}_{HCD}}$  và  ${{S}_{AED}}={{S}_{BCF}}$ 
		
		Mà  ${{S}_{ABCD}}={{S}_{ABE}}+{{S}_{BEC}}+{{S}_{DEC}}+{{S}_{AED}}$ 
		Hay  ${{S}_{ABCD}}={{S}_{DCH}}+{{S}_{BEC}}+{{S}_{DEC}}+{{S}_{BCF}}=S$ phần màu xám.
		Vậy  ${{S}_{ABCD}}=1$.
	\end{loigiaichuong37}
\end{bt} 
\begin{bt}
	Một khu vườn hình chữ nhật có chiều dài gấp 3 lần chiều rộng. Nếu tăng chiều rộng thêm  $3 \,m$ thì diện tích khu vườn tăng thêm  $135\, m^2$.  Người ta dùng cọc đóng rào xung quanh vườn đó, cứ  $3 \,m$ đóng 1 cọc. Hỏi đóng hết tất cả bao nhiêu cọc.
	\begin{loigiaichuong37}
		Chiều dài của khu vườn hình chữ nhật là:  $135:5=27\,(m)$
		 
		Chiều rộng khu vườn hình chữ nhật là:  $27:3=9 \,(m)$ 
		
		Chu vi ban đầu của khu vườn đó là:  $2\cdot(9+27)=72\,(m)$ 
		
		Người ta dùng số cọc là:  $72:3=24$ (cọc)
	\end{loigiaichuong37}
\end{bt}
\begin{bt}
	Các hình vuông dưới đây tạo bởi đonạ thẳng  $AB$ dài  $100 \,cm$  và đường gấp khúc $A{{A}_{1}}{{A}_{2}}...{{A}_{10}}B.$  Hỏi tổng chu vi của tất cả hình vuông bằng bao nhiêu?
	\begin{loigiaichuong37}
		Chu vi hình vuông thứ nhất là:  $4\cdot{{A}_{1}}{{A}_{2}}$ 
		
		Chu vi hình vuông thứ hai là:  $4\cdot{{A}_{3}}{{A}_{4}}$ 
		
		Chu vi hình vuông thứ ba là:  $4\cdot{{A}_{5}}{{A}_{6}}$ 
		
		Chu vi hình vuông thứ tư là:  $4\cdot{{A}_{7}}{{A}_{8}}$ 
		
		Chu vi hinh vuông thứ năm là:  $4\cdot{{A}_{9}}{{A}_{10}}$ 
		
		Vậy tổng chu vi của tất cả các hình vuông trên là: 
		$4\cdot({{A}_{1}}{{A}_{2}}+{{A}_{3}}{{A}_{4}}+{{A}_{5}}{{A}_{6}}+{{A}_{7}}{{A}_{8}}+{{A}_{9}}{{A}_{10}})=4\cdot AB=4\cdot 100=400 \,(cm)$ 
	\end{loigiaichuong37}
\end{bt}
\begin{bt}
	Sân bóng đá của một trường học có kích thước như hình vẽ trên. Trường học đó muốn trồng lại cỏ trong khu vực $16\,m\, 50$ (khu $16\,m\,50$ là khu vực được giới hạn bởi đường viền màu trắng có chiều dài $400\, dm$ và chiều rộng là $165\,dm$). Các con hãy giúp câu lạc bộ:
	\begin{enumerate}[a), leftmargin=*]
		\i Tính diện tích mỗi khu vực $16\,m\,50$ của sân bóng trên và số tiền dự kiến để trồng lại cỏ biết rằng giá tiền trồng cỏ mỗi mét vuông là $350000$ đồng.
		\i Câu lạc bộ muốn sơn lại đường viền bên ngoài của sân bóng bằng sơn màu trắng. Phải dùng ít nhất bao nhiêu thùng sơn để kẻ xong đường viền của sân bóng trên biết rằng khi dùng một thùng sơn thì kẻ được một vạch dài $1000 \,dm$.
	\end{enumerate}
	\begin{loigiaichuong37}
		\begin{enumerate}[a), leftmargin=*]
			\i Diện tích khu vực 16m50 là: 
			\[165\cdot400=66000 \,(dm^2)=660 \,(m^2)\] 
			Số tiền dự kiến để trồng lại cỏ là: 
			\[660\cdot35000=231000000 \,\text{(đồng)}\]
			\ Chu vi của sân bóng là: 
			\[2\cdot(1000+700)=3400 \,(dm)\] 
			Để kẻ xong đường viền sân bóng cần ít nhất: 
			\[3400:1000=3,4\approx 4\,  \text{(thùng sơn)}\]
		\end{enumerate}
	\end{loigiaichuong37}
\end{bt}
\begin{bt}
	Một vườn hình thang có đáy nhỏ $20\,m$ , đáy lớn  $40\,m$ .sau khi đáy lớn kéo dài thêm  $15\,m$ thì diện tích hình thang tăng lên  $45 \, m^2$ . Tính diện tích vườn lúc đầu.
	\begin{loigiaichuong37}
		Diện tích phần tăng thêm là diện tích của tam giác có cạnh đáy  $15m$ và chiều cao là chiều cao của hình thang ban đầu.
		
		Chiều cao của hình thang ban đầu là:  $45\cdot2:15=6 \,(m).$
		 
		Diện tích hình thang ban đầu là:  $\frac{1}{2}(20+40)\cdot6=180\, (m^2)$ 
	\end{loigiaichuong37}
\end{bt} 
\begin{bt}
	Cho tam giác $ABC$. Trên cạnh $AB$, lấy điểm $D$ sao cho $AD$ gấp 2 lần $DB$. Trên cạnh $AC$ lấy điểm $E$ sao cho $AE$ gấp 2 lần $EC$. Nối $B$ với $E$, $C$ với $D$, đoạn thẳng $BE$ cắt $CD$ tại $G$. Tính tỉ số  $\frac{S_{GBD}}{S_{GEC}}$.
	\begin{loigiaichuong37}
		Xét tam giác $EBC$ và tam giác $ABC$ có chung đường cao kẻ từ đỉnh $B$, do $AE$ gấp $2$ lần $AC$ nên  $EC=\frac{1}{3}AC.$
		  
		Vậy  ${{S}_{EBC}}=\frac{1}{3}{{S}_{ABC}}$
		    
		Xét tam giác $BDC$ và tam giác $ABC$ có chung đường cao kẻ từ đỉnh $C$, do $AD$ gấp $2$ lần $BD$ nên  $BD=\frac{1}{3}AB.$
		 
		Vậy  ${{S}_{BDC}}=\frac{1}{3}{{S}_{ABC}}$ 
		$\Rightarrow {{S}_{EBC}}={{S}_{BDC}}$.
		 
		Mà  ${{S}_{EBC}}={{S}_{BGC}}+{{S}_{GEC}}$ và  ${{S}_{BCD}}={{S}_{BGC}}+{{S}_{GBD}}$
		 
		Suy ra  ${{S}_{GEC}}={{S}_{GBD}}$  nên  $\frac{{{S}_{GBD}}}{{{S}_{GEC}}}=1$.
	\end{loigiaichuong37}
\end{bt}
\begin{bt}
	Trong hình chữ nhật $ABCD$, các điểm $P,Q,R,S$ lần lượt là trung điểm của các cạnh $AB,BC,CD,AD$. Biết điểm $T$ là trung điểm của cạnh $RS$. Hỏi phần diện tích tô xám chiếm bao nhiêu phần diện tích hình chữ nhật $ABCD$?
	\begin{loigiaichuong37}
		Nối thêm các hình như hình vẽ ta được hình vẽ: 
		
		
		Do $P,Q,R,S$ lần lượt là trung điểm của các cạnh $AB,BC,CD,AD$ nên  ${{S}_{1}}={{S}_{2}}={{S}_{3}}={{S}_{4}}$ 
		
		Mà  ${{S}_{1}}+{{S}_{2}}+{{S}_{3}}+{{S}_{4}}=\frac{{{S}_{DROS}}+{{S}_{RCQO}}+{{S}_{OQPB}}+{{S}_{SOPA}}}{2}=\frac{{{S}_{ABCD}}}{2}$ 
		Suy ra  ${{S}_{RQPS}}=\frac{{{S}_{ABCD}}}{2}$.
		
		Lại có $T$ là trung điểm của cạnh $RS$ nên  $ST=TR=\frac{SR}{2}$ 
		
		Nên  ${{S}_{STP}}=\frac{{{S}_{SPR}}}{2}=\frac{{{S}_{RQPS}}}{4}$  và  ${{S}_{QRT}}=\frac{{{S}_{SRQ}}}{2}=\frac{{{S}_{RQPS}}}{4}$.
		
		Vậy  ${{S}_{PTQ}}={{S}_{RQPS}}-\frac{{{S}_{RQPS}}}{4}-\frac{{{S}_{RQPS}}}{4}=\frac{{{S}_{RQPS}}}{2}=\frac{{{S}_{ABCD}}}{4}$.
	\end{loigiaichuong37}
\end{bt}
\begin{bt}
	Khu vực đỗ xe ô tô của một cửa hàng có hình chữ nhật với chiều dài $17\, m$, chiều rộng $12\, m$. Trong đó, một nửa khu vực dành cho quay đầu xe, hai phần tam giác ở góc để trồng hoa và phần còn lại chia đều cho năm chỗ đỗ ô tô. Tính diện tích đỗ xe dành cho các ô tô.
	\begin{loigiaichuong37}
		Mỗi chỗ để đỗ xe ô tô là một hình bình hành có chiều cao là một nửa chiều rộng của hình chữ nhật.
		
		Chiều cao của hình bình hành là: 
		\[12:2=6 \,(m)\] 
		Diện tích của một chỗ đỗ xe ô tô là: 
		\[3\cdot6=18 \,(m^2)\] 
		Diện tích 5 chỗ đỗ xe ô tô là:
		\[18\cdot5=90 \,m^2\] 
	\end{loigiaichuong37}
\end{bt} 
\begin{bt}
	Tính diện tích hình thoi $BMND$. Biết $ABCD$ là hình vuông và hai đường chéo của hình vuông:  $AC=BD=20 \,dm$. $M$ là điểm chính giữa $AO$; $N$ là điểm chính giữa $OC$. $AC$ và $BD$ cắt nhau tại $O$.
	\begin{loigiaichuong37}
		Xét tam giác $BAM$ và tam giác $BMO$ có chung đường cao từ đỉnh $B$ và $M$ nằm chính giữa $OA$ nên  $AM=MO$  $\Rightarrow {{S}_{BAM}}={{S}_{BMO}}$.
		
		Xét tam giác $BCN$ và tam giác $BNO$ có chung đường cao từ đỉnh $B$ và $N$ là điểm chính giữa $OC$ nên  $ON=NC\Rightarrow {{S}_{BNO}}={{S}_{BCN}}$.
		 
		Tương tự ta cũng có:  ${{S}_{DAM}}={{S}_{DMO}},{{S}_{DCN}}={{S}_{DNO}}$.
		 
		Mà  ${{S}_{BAM}}+{{S}_{BMO}}+{{S}_{BNO}}+{{S}_{BCN}}+{{S}_{DAM}}+{{S}_{DMO}}+{{S}_{DCN}}+{{S}_{DNO}}={{S}_{ABCD}}$ 
		
		$\Rightarrow 2({{S}_{BMO}}+{{S}_{BNO}}+{{S}_{DMO}}+{{S}_{DNO}})={{S}_{ABCD}}$\\ 
		$\Rightarrow 2{{S}_{BMDN}}={{S}_{ABCD}}$ \\ 
		$\Rightarrow {{S}_{BMDN}}=\frac{{{S}_{ABCD}}}{2}$\\ 
		
		Do $ABCD$ là hình vuông nên nó cũng là hình thoi
		
		$\Rightarrow {{S}_{ABCD}}=\frac{1}{2}\cdot20\cdot20=200\,(dm^2)$
		 
		Vậy  ${{S}_{BMDN}}=\frac{1}{2}{{S}_{ABCD}}=100\,(dm^2)$ 
	\end{loigiaichuong37}
\end{bt}
\Closesolutionfile{loigiaichung}

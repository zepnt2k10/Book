\def\i{\item}
\graphicspath{{../pictures/vande36/}}
\chapter{HÌNH HỌC TRỰC QUAN}
\section{Trục đối xứng – Tâm đối xứng}
\subsection{Kiến thức cần nhớ} 
\subsubsection{Hình có trục đối xứng}
\begin{enumerate}[--, leftmargin=*]
	\i Các hình bên đều có chung một tính chất: Có một đường thẳng $d$ chia hình thành 2 phần mà nếu ``gấp" hình theo đường thẳng $d$ thì hai phần đó chồng khít lên nhau.
	\i Những hình như thế gọi là \textbf{\textit{hình có trục đối xứng}} và đường thẳng $d$ là \textbf{\textit{trục đối xứng}} của nó.
\end{enumerate}
\subsubsection{Hình có tâm đối xứng}
\begin{enumerate}[--, leftmargin=*]
	\i Hình tròn, hình chong chóng bốn cánh đều có chung đặc điểm: Mỗi hình có một điểm $O$, mà khi quay hình đó xung quanh điểm $O$ đúng một nửa vòng thì hình thu được ``Chồng khít" với chính nó ở vị trí ban đầu (trước khi quay)
	\i Những hình như thế được gọi là hình có \textbf{\textit{tâm đối xứng}} và điểm $O$ được gọi là \textbf{\textit{tâm đối xứng}} của hình. 
\end{enumerate}
\subsection{Thực hành giải toán}
\begin{vd}
	\begin{enumerate}[a), leftmargin=*]
		\i Những chữ cái nào dưới đây có trục đối xứng, hãy vẽ trục đối xứng cho mỗi hình đó.
		
		\i Những chữ cái nào dưới đây có tâm đối xứng, hãy xác định tâm đối xứng của mỗi hình đó.
	\end{enumerate}
	\loigiai{
	\begin{enumerate}[a), leftmargin=*]
		\i Chữ cái có trục đối xứng: O, V, E, M, A, T, H.  
		\i Chữ cái có tâm đối xứng: O, N.
	\end{enumerate}	
	}
\end{vd}
\begin{vd}
	\begin{enumerate}[a), leftmargin=*]
		\i Vẽ tiếp để được hình có trục đối xứng.  
		
		\i Vẽ tiếp để được hình có tâm $O$ đối xứng 
	\end{enumerate}
	\loigiai{
		\begin{enumerate}[a), leftmargin=*]
			\i
			\i
		\end{enumerate}
	}
\end{vd}
\subsection{Mở rộng kiến thức}
\begin{enumerate}[--, leftmargin=*]
	\i Một hình có thể có một trục đối xứng, hai trục đối xứng, \ldots, vô số trục đối xứng hoặc không có trục đối xứng.
	\i Một có nhiều nhất một tâm đối xứng.
	\i Nếu hình có từ 2 trục đối xứng trở lên thì giao điểm của các trục đối xứng đó là tâm đối xứng của hình đó.
	\i Trục đối xứng của một hình chia hình đó thành hai phần bằng nhau.
\end{enumerate}
\subsection{Bài tập tự luyện}
\subsubsection*{Mức độ cơ bản}
\Opensolutionfile{loigiaichung}[loigiaichuong36]
\begin{bt}
	Chỉ ra các hình có trục đối xứng, hình có tâm đối xứng. Xác định trục đối xứng và tâm đối xứng của các hình.
%	Hình 1                         Hình 2                                   Hình 3        
%	
%	Hình 4                                    Hình 5                            Hình 6
%	
%	Hình 7                                                                   Hình 8
	\begin{loigiaichuong36}
		Các hình có trục đối xứng là: Hình 1, Hình 2, Hình 3, Hình 4, Hình 5, Hình 6, Hình 7.
		
		Các hình có tâm đối xứng là: Hình 1, Hình 3, Hình 5, Hình 6.
	\end{loigiaichuong36}
\end{bt}
\begin{bt}
	Cho các hình ảnh sau. Hãy xác định tâm đối xứng của các hình đó (nếu có)
	\begin{loigiaichuong36}
		Các hình có tâm đối xứng là:
	\end{loigiaichuong36}
\end{bt}
\begin{bt}
	Kể tên ít nhất 3 hình ảnh trong thực tế có
	\begin{enumerate}[a), leftmargin=*]
		\i Trục đối xứng
		\i Tâm đối xứng
	\end{enumerate}
	\begin{loigiaichuong36}
		\begin{enumerate}[a), leftmargin=*]
			\i Các vật dụng có trục đối xứng: Cổng nhà, con diều, bình  hoa, cửa sổ, \ldots
			
			\i Các vật dụng có tâm đối xứng: Mặt của thớt, ạch hoa lát nền, mặt đồng hồ, cỏ 4 lá, lá lốt \ldots
		\end{enumerate}
	\end{loigiaichuong36}
\end{bt}
\begin{bt}
	Vẽ phần đối xứng với phần đã cho theo trục là đường thẳng $d$ cho trước.
	\begin{loigiaichuong36}
		chèn ảnh
	\end{loigiaichuong36}
\end{bt}
\begin{bt}
	Vẽ phần hình còn lại để $O$ là tâm đối xứng của hình tạo thành.
	\begin{loigiaichuong36}
		chèn ảnh
	\end{loigiaichuong36}
\end{bt}
\begin{bt}
	Trong câu ``BOI DUONG KI NANG TU HOC TOAN 6"
	Em hãy chỉ ra:
	\begin{enumerate}[a), leftmargin=*]
		\i Các chữ cái chỉ có tâm đối xứng.
		\i Các chữ cái chỉ có trục đối xứng.
		\i Các chữ cái vừa có tâm, vừa có trục đối xứng.
	\end{enumerate}
	\begin{loigiaichuong36}
		\begin{enumerate}[a), leftmargin=*]
			\i Các chữ chỉ có tâm đối xứng: N, S.
			\i Các chữ chỉ có trục đối xứng: V, E, M, A, T, D.
			\i Các chữ vừa có tâm, vừa có trục đối xứng: O, H, I.
		\end{enumerate}
	\end{loigiaichuong36}
\end{bt}
\begin{bt}
	Bạn Mai gấp đôi tờ giấy để cắt một số hình như hình dưới đây. Hãy vẽ thêm phần còn khuyết của các hình đó.
	\begin{loigiaichuong36}
		chèn ảnh
	\end{loigiaichuong36}
\end{bt}
\begin{bt}
	\begin{enumerate}[a), leftmargin=*]
		\i Trong các biển báo sau, biển báo nào có trục đối xứng?
%		
%		
%		Hình 1                                                                        Hình 2
%		
%		Hình 3                                                                        Hình 4
%		
%		Hình 5
		\i Trong các biển báo sau, biển báo nào có tâm đối xứng?
%		
%		Hình 6                                                                            Hình 7
%		
%		Hình 8  
	\end{enumerate}
	\begin{loigiaichuong36}
		\begin{enumerate}[a), leftmargin=*]
			\i Hình có trục đối xứng là: Hình 1, Hình  4, Hình  5.
			\i Hình có tâm đối xứng là: Hình 6, Hình 8, Hình  9.
		\end{enumerate}
	\end{loigiaichuong36}
\end{bt}                                                                 Hình 9
\begin{bt}
	Mỗi hình sau có tất cả bao nhiêu trục đối xứng? Vẽ các trục đối xứng đó
%	
%	Hình a                                                                               Hình b
%	
%	
%	Hình c                                                            Hình d
	\begin{loigiaichuong36}
		Hình a có 1 trục đối xứng
		
		Hình b có 2 trục đối xứng
		
		Hình c có 2 trục đối xứng
		
		Hình d có 4 trục đối xứng
	\end{loigiaichuong36}
\end{bt}
\begin{bt}
	Các em tự sáng tạo vẽ 3 hình
	\begin{enumerate}[a), leftmargin=*]
		\i có trục đối xứng mà không có tâm đối xứng 
		\i có cả trục đối xứng và tâm đối xứng
	\end{enumerate}
	\begin{loigiaichuong36}
		Các em tự vẽ
	\end{loigiaichuong36}
\end{bt}
\subsubsection*{Mức độ nâng cao}
\begin{bt}
	\begin{enumerate}[a), leftmargin=*]
		\i Từ 4 hình tam giác vuông bằng nhau (hình minh họa phía dưới) em hãy ghép thành ít nhất 5 hình có trục đối xứng.
		
		\i Từ 4 hình vuông bằng nhau (hình minh họa phía dưới) em hãy ghép thành ít nhất 4 hình có tâm đối xứng.
	\end{enumerate}
	\begin{loigiaichuong36}
		\begin{enumerate}[a), leftmargin=*]
			\i Ta có thể ghép thành các hình như sau: 

			\i Các hình có tâm đối xứng từ 4 hình vuông đã cho như sau
			
		\end{enumerate}		
	\end{loigiaichuong36}
\end{bt}
\begin{bt}
	Từ hai mảnh ghép hình chữ L, ta có thể xếp được các hình có trục đối xứng như hình minh họa bên dưới. Hãy chỉ ra cách để có thể xếp được các hình có trục đối xứng từ $n$ miếng ghép trên, với $n$ là số tự nhiên bất kì lớn hơn 2
	\begin{loigiaichuong36}
		Nếu $n$ là số chẵn ta sẽ chia đều hai bên và ghép theo ý tưởng hình ở đề bài
		Nếu $n$ lẻ ta xếp thep ý tưởng sau
	\end{loigiaichuong36}
\end{bt}
\Closesolutionfile{loigiaichung}
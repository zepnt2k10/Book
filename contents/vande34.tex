\def\i{\item}
\graphicspath{{../pictures/vande34/}}
\chapter{HÌNH HỌC TRỰC QUAN}
\section{Một số hình phẳng có dạng đặc biệt}
\section*{Phần 1: Hình tam giác đều -- Hình vuông -- Hình lục giác đều}
\subsection{Kiến thức cần nhớ}
\subsubsection{Hình tam giác đều.}
Tam giác $ABC$ là tam giác đều có:
\begin{enumerate}[--, leftmargin=*]
	\i Ba cạnh bằng nhau: $AB = BC = CA$.  
	\i Ba góc bằng nhau và bằng $60^\circ$: $\widehat{A} = \widehat{B}= \widehat{C} = 60^\circ$.
\end{enumerate}
\subsubsection{Hình vuông} 
Trong hình vuông:
\begin{enumerate}[--, leftmargin=*]
	\i Bốn cạnh bằng nhau: $AB = BC = CD = DA$. 
	\i Bốn góc bằng nhau và bằng $90^\circ$: $\widehat{A} = \widehat{B}= \widehat{C} = \widehat{D} = 90^\circ$.   
	\i Hai đường chéo bằng nhau: $AC = BD$.
\end{enumerate} 
\subsubsection{Hình lục giác đều}
Hình lục giác đều có:
\begin{enumerate}[--, leftmargin=*]
	\i Sáu cạnh bằng nhau: $AB = BC = CD = DE = EF = FA$.  
	\i Sáu góc bằng nhau, mỗi góc bằng $120^\circ$: $\widehat{A} = \widehat{B}= \widehat{C} = \widehat{D} = \widehat{E} = \widehat{F} =  120^\circ$. 
	\i Ba đường chéo chính bằng nhau: $AC = CF = BE$.
\end{enumerate}

\subsection{Thực hành giải toán.}
\begin{vd}
	\begin{enumerate}[a), leftmargin=*]
		\i Trong các hình sau, hình nào là tam giác đều? 
		\i Trong các hình sau, hình nào là hình vuông? 
		\i Trong các hình sau, hình nào là lục giác đều? 
	\end{enumerate}
	\loigiai{
	\begin{enumerate}[a), leftmargin=*]
		\i Hình 1, hình 3.
		\i Hình 1.
		\i Hình 2.
	\end{enumerate}
	}
\end{vd}
\begin{vd}
	Vẽ tam giác đều có cạnh 3 cm.
	\loigiai{
		\textbf{Cách 1: Dùng thước thẳng và ê--ke có góc $60^\circ$} 
		\begin{enumerate}[Bước 1:, leftmargin=*]
			\i Vẽ đoạn thẳng $AB = 3$ cm.
			\i Dùng ê--ke có góc  $60^\circ$, vẽ góc  $BAx = 60^\circ$.
			\i Tiếp tục dùng ê--ke có góc $60^\circ$ vẽ góc $ABy = 60^\circ$.
			\i Tia $Ax$  và $By$ cắt nhau tại $C$ ta được tam giác đều $ABC$ 
		\end{enumerate}
		\textbf{Cách 2: Dùng thước thẳng và compa}
		\begin{enumerate}[Bước 1:, leftmargin=*]
			\i Vẽ đoạn thẳng $AB = 3$ cm.
			\i Lấy $A$ làm tâm. Vẽ một phần đường tròn bán kính $AB$
			\i Lấy $B$ làm tâm dùng compa vẽ $1$ vẽ một phần đường tròn $AB$ (vẽ cùng phía với $AB$ và phần đường tròn ở bước $2$).
			\i Gọi $C$ là giao điểm của 2 phần đường tròn vừa vẽ. Nối các đoạn $AC$, $BC$ ta được tam giác đều $ABC$. 
		\end{enumerate}
	}
\end{vd}
\begin{vd}
	Vẽ hình vuông ABCD có cạnh 4 cm. Kể tên các đường chéo của hình vuông đó
	\loigiai{
		\begin{enumerate}[Bước 1:, leftmargin=*]
			\i Vẽ đoạn thẳng $AB =4$ cm.
			\i Vẽ đường thẳng vuông góc với $AB$ tại $A$. Xác định điểm $D$ trên đường thẳng đó sao cho $AD =4$ cm. 
			\i Vẽ đường thẳng vuông góc với $AB$ tại $B$. Xác định điểm $C$ trên đường thẳng đó sao cho $BC = 4$ cm.
			\i Nối $C$ với $D$ ta được hình vuông $ABCD$
		\end{enumerate}
	Các đường chéo của hình vuông $ABCD$ là $AC$ và $BD$.
	}
\end{vd}
\begin{vd}
	Cho hình vuông kích thước $9\times9$ do 10 đường kẻ ngang và 10 đường kẻ dọc (gọi chung là đường lưới) tạo thành. Tìm tất cả số hình vuông tạo bởi các đường lưới ấy.
	\loigiai{
		\textbf{Tìm lời giải.}
		
		\textit{Các hình vuông cần tìm sẽ có những kích thước: $1\times 1, 2\times2, \ldots, 9\times9$. Ta sẽ tính số hình vuông mỗi kích thước trên.}
		\begin{enumerate}[--, leftmargin=*]
			\i Hình vuông kích thước $1\times 1$: Ta cho các hình vuông kích thước $1\times1$ này di chuyển trong hình vuông lớn, đến các vị trí sao cho đỉnh của hình vuông nằm trên giao điểm của các đường lưới. Mỗi vị trí tương ứng với một hình vuông kich thước $1\times1$. 
			\i Ta thấy hình vuông $1\times1$ di chuyển theo chiều ngang sẽ có $9$ vị trí, theo chiều dọc cũng có 9 vị tri do đó số hình vuông kích thước $1\times 1$ là $9 \times 9 = 81$ (hình)
			\i Số hình vuông kích thước $2\times 2$: Tương tự như trên, hình vuông kích thước $2\times 2$ di chuyển theo chiều ngang sẽ có 8 vị trí, chiều dọc cũng có 8 vị trí nên số hình vuông kích thước $2\times 2$ là $8 \times 8 =64$ (hình).
			\i Tương tự cho các hình vuông kích thước còn lại
			\i Số hình vuông được tạo thành là
			\begin{align*}
				9^2 + 8^2 + 7^2 + 6^2 + 5^2 + 4^2 + 3^2 +2^2 +1^2 = 190 \text{ (hình).}
			\end{align*}
		\end{enumerate}
	}
\end{vd}
\subsection{Mở rộng kiến thức}
Cách vẽ lục giác   đều cạnh  cm bằng compa và thước thẳng.

\textit{Cách 1.}
\begin{enumerate}[Bước 1:, leftmargin=*]
	\i Vẽ hình tròn tâm  $O$ có bán kính $a$ cm.
	\i Trên đường tròn lấy điểm $A$ bất kì. Vẽ đường tròn tâm $A$ bán kính $a$ cm, cắt đường tròn ban đầu tại $2$ điểm $B$  và $F$. 
	\i Đặt thước đi qua $B, O$  cắt đường tròn ban đầu tại $E$\\ 
	Tương tự đặt thước qua $F,O$  cắt đường tròn ban đầu tại $C$;\\
	Đặt thước qua $AO$ cắt đường tròn ban đầu tại $D$.\\
	\i Nối các cạnh $AB$, $BC$, $CD$, $DE$, $EF$, $FA$ ta được tam giác đều $ABCDEF$.
\end{enumerate}
\textit{Cách 2.}
\begin{enumerate}[--, leftmargin=*]
	\i Vẽ lần lượt các tam giác đều $OAB$, $OBC$, $OCD$, $ODE$, $OEF$ có cạnh bằng $a$ cm trên cùng một hình vẽ. Nối $AF$ ta được lục giác đều $ABCDEF$.
\end{enumerate}
\subsection{BÀI TẬP TỰ LUYỆN}
\Opensolutionfile{loigiaichung}[loigiaichuong34]
\subsubsection{Mức độ cơ bản}
\begin{bt}
	\begin{enumerate}[a), leftmargin=*]
		\i Trong các hình sau, hình nào là tam giác đều? 
		\i Trong các hình sau, hình nào là hình vuông? 
		\i Trong các hình sau, hình nào là lục giác đều?
	\end{enumerate}
	\begin{loigiaichuong34}
		\begin{enumerate}[a), leftmargin=*]
			\i Các hình là tam giác đều là: Hình 1, Hình 3.
			\begin{enumerate}[--, leftmargin=*]
				\i Dùng thước kẻ đo độ dài 3 cạnh của tam giác.
				\i Hình 1: Ba cạnh bằng nhau và bằng 4 cm.
				\i Hình 3: Ba cạnh bằng nhau và bằng 4,5 cm.
			\end{enumerate}
			\i Hình vuông là: Hình 1.
			\begin{enumerate}[--, leftmargin=*]
				\i Bốn cạnh bằng nhau.
				\i Bốn góc bằng nhau và bằng $90^\circ$.
				\i Hai đường chéo bằng nhau.
			\end{enumerate}
			\i Hình lục giác đều là: Hình 2.
			\begin{enumerate}[--, leftmargin=*]
				\i Sáu cạnh bằng nhau.
				\i Ba đường chéo chính bằng nhau.
			\end{enumerate}
		\end{enumerate}
	\end{loigiaichuong34}
\end{bt}
\begin{bt}
	Vẽ tam giác đều $OMN$ có cạnh $4$ cm. Chỉ rõ các cạnh, góc, đỉnh.
	\begin{loigiaichuong34}
		\textit{Cách 1.}
		\begin{enumerate}[Bước 1:, leftmargin=*]
			\i Vẽ đoạn thẳng $MN =4$ cm.
			\i Dùng eke có góc $60^\circ$ vẽ $NMx = 60^\circ$ và góc  $MNy = 90^\circ$.
			\i Tia $Mx$ và tia $Ny$ cắt nhau tại $O$.
			Ta được tam giác đều $OMN$ có. 
			\begin{enumerate}[+, leftmargin=*]
				\i Các cạnh:  $OM = ON = MN = 4$ cm.
				\i Các góc:  $ONM, OMN, MON$.
				\i Các đỉnh: $M, N, O$.
			\end{enumerate}
		\end{enumerate}
		\textit{Cách 2.}
		\begin{enumerate}[Bước 1:, leftmargin=*]
			\i Vẽ đoạn thẳng  $MN = 4$ cm.
			\i Lấy $M$ làm tâm. Vẽ một phần đường tròn bán kính  $4$ cm.\\
			 Lấy $N$ làm tâm dùng compa vẽ 1 phần đường tròn $MN$.
			\i Gọi $O$ là giao điểm giữa 2 phần đường tròn vừa vẽ.
			\i Nối các đoạn $OM. ON$ ta được tam giác đều $OMN$.
		\end{enumerate}
	\end{loigiaichuong34}
\end{bt}
\begin{bt}
	Vẽ hình vuông $EFOP$ có cạnh $5$ cm. Chỉ rõ các cạnh, góc, đỉnh.
	\begin{loigiaichuong34}
		\begin{enumerate}[Bước 1:, leftmargin=*]
			\i Vẽ đoạn thẳng $EF = 5$ cm 
			\i Vẽ đường thẳng vuông góc với $EF$  tại $E$. Xác định điểm $P$  trên đường thẳng đó sao cho $EP = 5$ cm. 
			\i Vẽ đường thẳng vuông góc với $EF$  tại $F$.  Xác định điểm $P$  trên đường thẳng đó sao cho $FO = 3$ cm. 
			\i Nối $O,P$  ta được hình vuông  $EFOP$ có:
			\begin{enumerate}[--, leftmargin=*]
				\i Các cạnh: $EF, FO, PO, EP$. 
				\i Các góc: $EFO, FOP, OPE, PEF$. 
				\i Các đỉnh: $E,F,O,P$.
			\end{enumerate} 
		\end{enumerate}
	\end{loigiaichuong34}
\end{bt} 
\begin{bt}
	Vẽ hình lục giác đều $OPNQEF$ cạnh bằng $3$ cm. Chỉ rõ đường chéo chính, đường chéo phụ.
	\begin{loigiaichuong34}
		\textit{Cách 1:}
		\begin{enumerate}[Bước 1:, leftmargin=*]
			\i Vẽ đường tròn tâm $I$  bán kính $3$ cm. 
			\i Trên đường tròn lấy điểm  bất kì. Vẽ đường tròn tâm $O$ bán kính $3$ cm cắt đường tròn tại $2$ điểm $P, F$. 
			\i Đặt thước đi qua $P,I$ cắt đường tròn ban đầu tại $E$.\\  
			Đặt thước đi qua$F,I$  cắt đường tròn ban đầu tại $N$.\\ 
			Đặt thước đi qua $O,I$ cắt đường tròn ban đầu tại $Q$. 
		\end{enumerate}
		\textit{Cách 2.}
		
		Vẽ lần lượt các tam giác đều: $IOP, IPN, INQ, IQE, IEF, IFO$, có các cạnh bằng 3 cm trên cùng 1 hình vẽ. Nối $O,F$ ta được lục giác đều $OPNQEF$.
		 
		Lục giác đều  có: 
		\begin{enumerate}[--, leftmargin=*]
			\i Các đường chéo phụ: $ON, NE, EO, FP, PQ, QF$. 
			\i Các đường chéo chính: $OQ, PE, FN$.
		\end{enumerate}             
	\end{loigiaichuong34}
\end{bt}
\begin{bt}
	Trong mỗi hình vẽ sau có bao nhiêu tam giác đều? 
	\begin{loigiaichuong34}
		Hình a) có 15 tam giác đều
		Hình b) có 6 tam giác đều.
	\end{loigiaichuong34}
\end{bt}
\begin{bt}
	Trong hình vẽ sau có bao nhiêu lục giác đều? bao nhiêu tam giác đều?
	\begin{loigiaichuong34}
		Hình vẽ có 2 lục giác đều và có 8 tam giác đều.
	\end{loigiaichuong34}
\end{bt}
\begin{bt}
	\begin{enumerate}[a), leftmargin=*]
		\i Từ 6 tam giác đều bằng nhau em hãy ghép lại được một lục giác đều.
		\i Từ 4 tam giác đều. Em hãy ghép lại để được một tam giác đều. 
	\end{enumerate}
	\begin{loigiaichuong34}
		\begin{enumerate}[a), leftmargin=*]
			\i Từ 6 tam giác bằng nhau ghé được 1 lục giác đều.
			
			\i Từ 4 tam giác đều ghép thành 1 tam giác đều.

		\end{enumerate}
	\end{loigiaichuong34}
\end{bt}
\begin{bt}
	Cho hình vuông kích thước $5 \times 5$ do 6 đường kẻ ngang và 6 đường kẻ dọc (gọi chung là đường lưới) tạo thành. Tìm tất cả số hình vuông tạo bởi các đường lưới ấy
	\begin{loigiaichuong34}
		\textit{Các hình vuông cần tìm sẽ có những kích thước: $1\times 1, 2\times 2,\ldots, 5\times5$. Ta sẽ tính số hình vuông mỗi kích thước trên.}
		\begin{enumerate}[--, leftmargin=*]
			\i Hình vuông kích thước  $1\times 1$: Ta cho các hình vuông kích thước $1\times 1$  này di chuyển trong hình vuông lớn, đến các vị trí sao cho đỉnh của hình vuông nằm trên giao điểm của các đường lưới. Mỗi vị trí tương ứng với một hình vuông kich thước $ 1\times 1$.
			\i Ta thấy hình vuông di chuyển theo chiều ngang sẽ có 5 vị trí, theo chiều dọc cũng có 5 vị tri do đó số hình vuông kích thước $1\times 1$  là $5 \times 5 = 25$ (hình).
			\i Số hình vuông kích thước  $2 \times 2$: Tương tự như trên, hình vuông kích thước $2 \times 2$  di chuyển theo chiều ngang sẽ có 4 vị trí, chiều dọc cũng có 4 vị trí nên số hình vuông kích thước $2 \times 2$  là $4 \times 4 = 16$  (hình).
			\i Tương tự ta có: 
			\begin{enumerate}[+, leftmargin=*]
				\i Số hình vuông có kích thước $3 \times 3$ là: $3\times 3 = 9$.  
				\i Số hình vuông có kích thước $4 \times 4$ là: $2 \times 2 = 4$. 
				\i Sô hình vuông có kích thước $5 \times 5$ là: $1 \times 1 = 1$. 
			\end{enumerate}
		\end{enumerate}
		Vậy có tất cả số hình vuông là: $5^2 + 4^2 + 3^2 +2^2 + 1^2 = 55$ (hình vuông).
	\end{loigiaichuong34}
\end{bt}
\subsubsection{Mức độ nâng cao}
\begin{bt}
	Dùng 12 que diêm giống nhau để xếp:
	\begin{enumerate}[a), leftmargin=*]
		\i 1 hình vuông
		\i 1 tam giác đều
		\i 1 hình lục giác đều
	\end{enumerate}
	\begin{loigiaichuong34}
		\begin{enumerate}[a), leftmargin=*]
			\i Dùng 12 que diêm để xếp một hình vuông:
			
			\i Dùng 12 que diêm vẽ tam giác đều.
			
			\i Dùng 12 que diêm vẽ lục giác đều.
		\end{enumerate}
	\end{loigiaichuong34}
\end{bt}
\begin{bt}
	Trong hình sau có bao nhiêu tam giác đều?
	\begin{loigiaichuong34}
		Trong hình có 20 tam giác đều.
	\end{loigiaichuong34}
\end{bt}
\begin{bt}
	Trong hình sau có bao nhiêu hình vuông?
	\begin{loigiaichuong34}
		Trong hình có 10 hình vuông.
	\end{loigiaichuong34}
\end{bt}
\begin{bt}
	Từ một tam giác đều, em hãy gấp thành một chiếc hộp kín có 4 mặt là các tam giác đều bằng nhau.
	\begin{loigiaichuong34}
		chèn ảnh
	\end{loigiaichuong34}
\end{bt} 
\begin{bt}
	Em hãy vẽ các hình tam giác đều, lục giác đều và hình vuông vào các ô trống sao cho mỗi hàng ngang, dọc không có hình nào được lặp lại.
	\begin{loigiaichuong34}
		chèn ảnh
	\end{loigiaichuong34}
\end{bt}
\begin{bt}
	Trò chơi: Đi tìm kho báu
	
	Một người xuất phát ở vị trí $A$ để đi tìm kho báu vị trí $B$. Người đó chỉ được đi theo cạnh hoặc đường chéo của hình vuông nhỏ. Trên đường đi có 2 chướng ngại vật như hình vẽ. Biết thời gian đi hết 1 cạnh là 2 giây, thời gian đi hết đường chéo là 3 giây. Em hãy tìm con đường nhanh nhất để đến được vị trí kho báu. Thời gian đó là bao nhiêu giây? 
	\begin{loigiaichuong34}
		Quãng đường nhanh nhất đến kho báu mất: 
		\[3 \times 3 + 3\times 2 = 13 \text{ (giây).}\]
	\end{loigiaichuong34}
\end{bt}
\begin{bt}
	Cho hình chữ nhật kich thước $4 \times 5$ được tạo thành từ 5 đường kẻ ngang và 6 đường kẻ dọc (đường lưới). Hỏi có bao nhiêu hình vuông được tạo thành bởi các đường lưới ấy?
	\begin{loigiaichuong34}
		\begin{enumerate}[--, leftmargin=*]
			\i Số hình vuông kích thước $1\times 1$  là: $4 \times 5 = 20$. 
			\i Số hình vuông kích thước  $2\times 2$: hình vuông kích thước $2\times 2$  di chuyển theo chiều ngang sẽ có $4$ vị trí, chiều dọc có $3$ vị trí nên số hình vuông kích thước $2\times 2$  là $3\times 4 = 12$  (hình).
			\i Số hình vuông có kích thước  $3\times 3$: hình vuông kích thước $3\times 3$  di chuyển theo chiều ngang sẽ có 3 vị trí, chiều dọc có 2 vị trí nên số hình vuông kích thước  $3\times 3$ là $2\times 3 = 6$ (hình)
			\i Số hình vuông có kích thước $4 \times 4$ là: $1\times 2 = 2$. 
		\end{enumerate}
		Vậy có tất cả số hình vuông là: 40 (hình vuông)
	\end{loigiaichuong34}
\end{bt}
\begin{bt}
	Cho một hình vuông gồm $9 \times 9 = 81$ ô kẻ vuông tạo thành, trong đó có 9 ô kẻ vuông bị đục bỏ (phần màu xám). Hỏi có tất cả bao nhiêu hình vuông tạo bởi các đường lưới trong hình?
	\begin{loigiaichuong34}
		Số hình vuông được tạo thành là (không kể phần màu xám)
		\[{9^2} + {8^2} + {7^2} + {6^2} + {5^2} + {4^2} + {3^2} + {2^2} + {1^2} = 285 \text{ (hình).}\]
		Bây giờ ta đếm số hình vuông có một phần cạnh bị bỏ đi (tức một phần cạnh nằm trong phần màu xám), gọi chung là ``hình vuông xấu".
		\begin{enumerate}[--, leftmargin=*]
			\i Có $3\times3$  hình vuông $1\times 1$  ``xấu".
			\i Có $4\times4$  hình vuông $2\times 2$  ``xấu".
			\i Số hình vuông $3\times 3$  xấu là: $5\times 5 - 1\times 1 = 24$. (các đỉnh trên cùng bên phải của chúng được đánh dấu trong hình dưới bởi các chấm nhỏ màu đen trong hình).
		\end{enumerate}
		Tương tự ta có: 
		\begin{enumerate}[--, leftmargin=*]
			\i Số hình vuông $4\times 4$  ``xấu" là: $5 \times 6 - 2 \times 2 = 26$.  
			\i Số hình vuông $5 \times 5$  ``xấu" là: $5 \times 5 - 3\times 3 = 16$. 
			\i Có 4 hình vuông $6 \times 6$  ``xấu"
			
			\i Các hình vuông $7\times7; 8\times 8; 9\times 9$ thì không có phần cạnh nào nằm ở phần màu xám nên số hình vuông ``xấu" là 0.
		\end{enumerate}
		Vì vậy số hình vuông được tạo thành là: $285 - \left( {9 + 16 + 24 + 26 + 16 + 4} \right) = 190$  (hình vuông).
	\end{loigiaichuong34}
\end{bt}
\Closesolutionfile{loigiaichung}
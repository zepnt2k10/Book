\def\i{\item}
\subsection{Thực hành giải toán}
\begin{vd}
	Cho tập hợp $P=\{x \in \mathbb{N} \mid 4 \le x <12\}$ và tập hợp $Q$ là tập các số tự nhiên chẵn có một chữ số.
	\begin{enumerate}[a)]
		\i Trong các số: $0$; $2$; $6$; $9$; $12$ số nào thuộc tập hợp $P$? Số nào không thuộc tập hợp $P$? Dùng kí hiệu để trả lời.
		\i Viết tập hợp $Q$ bằng cách liệt kê phần tử.
		\i Chỉ ra những phần tử thuộc cả hai tập hợp $P$ và $Q$.
		\i Gọi $M$ là tập hợp các số lẻ thuộc tập $P$. Hãy viết tập hợp $M$ bằng 2 cách.
	\end{enumerate}
	\loigiai{
		\begin{enumerate}[a)]
			\i $0 \notin P$, \quad $2 \notin P$, \quad $6 \in P$, \quad $9 \in P$, \quad $12 \notin P$.
			\i $Q=\{0;\, 2;\, 4;\, 6;\, 8\}$.
			\i Gọi $R$ là tập hợp gồm những phần tử thuộc cả hai tập hợp $P$ và $Q$. Ta có: $R=\{4;\, 6;\, 8\}$.
			\i Cách 1: Liệt kê phần tử: $M=\{5;\, 7;\, 9;\, 11\}$.\\
			Cách 2: Chỉ ra tính chất đặc trưng: $M=\{x \in \mathbb{N} \mid x \text{ là số tự nhiên lẻ và } 5 \le x\le 11\}$.
	\end{enumerate}}
\end{vd}
Gọi $M$ là tập hợp các số lẻ thuộc tập $P$. Hãy viết tập hợp $M$ bằng 2 cách.
\begin{itemize}[$\circ$]
	\i Cách 1: {\it Liệt kê tất cả các phần tử của tập hợp.}		Ví dụ: $P=\{0;\, 2;\, 4;\, 6;\, 8\}$.
	\i Cách 2: {\it Nêu dấu hiệu đặc trưng của các phần tử trong tập hợp đó.}		Ví dụ: $P=\{n \mid n \text{ là số tự nhiên nhỏ hơn } 10\}$.
\end{itemize}

\begin{enumerate}[$\circ$]
	\i Cách 1: {\it Liệt kê tất cả các phần tử của tập hợp.}		Ví dụ: $P=\{0;\, 2;\, 4;\, 6;\, 8\}$.
	\i Cách 2: {\it Nêu dấu hiệu đặc trưng của các phần tử trong tập hợp đó.}		Ví dụ: $P=\{n \mid n \text{ là số tự nhiên nhỏ hơn } 10\}$.
\end{enumerate}

\begin{enumerate}[--,leftmargin=*]
	\i Trong các số: $0$; $2$; $6$; $9$; $12$ số nào thuộc tập hợp $P$? Số nào không thuộc tập hợp $P$? Dùng kí hiệu để trả lời.
	\i Viết tập hợp $Q$ bằng cách liệt kê phần tử.
	\i Chỉ ra những phần tử thuộc cả hai tập hợp $P$ và $Q$.
	\i Gọi $M$ là tập hợp các số lẻ thuộc tập $P$. Hãy viết tập hợp $M$ bằng 2 cách.
	\begin{itemize}[+,leftmargin=*)]
		\i Trong các số: $0$; $2$; $6$; $9$; $12$ số nào thuộc tập hợp $P$? Số nào không thuộc tập hợp $P$? Dùng kí hiệu để trả lời.
		\i Viết tập hợp $Q$ bằng cách liệt kê phần tử.
		\i Chỉ ra những phần tử thuộc cả hai tập hợp $P$ và $Q$.
		\i Gọi $M$ là tập hợp các số lẻ thuộc tập $P$. Hãy viết tập hợp $M$ bằng 2 cách.
	\end{itemize}
\end{enumerate}
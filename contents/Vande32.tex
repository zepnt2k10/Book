\def\i{\item}
\graphicspath{{../pictures/vande31/}}
\chapter{LÀM QUEN VỚI TOÁN KINH TẾ}
\section{LÃI ĐƠN -- LÃI KÉP}
\begin{center}
	\textit{Bác Dũng tiết kiệm được 50 triệu đồng. Bác dự định cất vào két sắt trong vòng 2 năm. Em có ủng hộ việc làm đó của bác Sơn không? Vì sao?}
\end{center}
\textbf{\begin{center}
		BẢNG THUẬT NGỮ
\end{center}}
\begin{tabular}{|p{0.2\textwidth}|p{0.745\textwidth}|}
	\hline
	TÊN THUẬT NGỮ &	DIỄN GIẢI\\
	\hline
	Tiền lãi&	Là khoản tiền chênh lệch (lớn hơn), thu được từ một động sản xuất, kinh doanh.\\
	\hline
	Lãi suất&	Là tỉ số phần trăm của tiền lãi với tiền gốc\\
	\hline
	Lãi đơn&	Lãi suất được tính dựa trên số tiền gốc ban đầu trong một khoảng thời gian nhất định.
	Ví dụ, ta gửi tiết kiệm ở ngân hàng 100 triệu đồng với lãi suất 7\%/ năm thì số tiền lãi sau một năm tính theo phương pháp lãi đơn sẽ là 7 triệu đồng.\\
	\hline
	Lãi kép&	Lãi suất của được tính dựa trên số tiền gốc ban đầu và số tiền lãi thu được của thời kì trước đó.
	Ví dụ, ta gửi tiết kiệm ở ngân 100 triệu đồng với lãi suất kép 5\%/năm. Sau 1 năm thì tiền lãi là 5 triệu đồng. Số tiền 5 triệu đồng này được cộng vào tiền gốc thành 105 triệu đồng để tính lãi của năm tiếp theo.
	Cách nói khác: Lãi chồng lãi, lãi mẹ đẻ lãi con\\
	\hline
\end{tabular}
\begin{vd}
	Sau Tết Nguyên Đán, Hoa đưa cho mẹ 2000000 đồng tiền mừng tuổi. Mẹ Hoa mang ra ngân hàng gửi tiết kiệm theo phương thức lãi đơn với lãi suất 6\%/năm. Hỏi:
	\begin{enumerate}[a),leftmargin=*]
		\i Sau 2 năm số tiền mẹ Hoa nhận được là bao nhiêu?
		\i Sau ít nhất bao lâu thì mẹ Hoa rút được cả vốn lẫn lãi là 2360000 đồng?
	\end{enumerate}
	\loigiai{
		\begin{enumerate}[a),leftmargin=*]
			\i Sau 1 năm, tiền lãi của mẹ Hoa là
			\[6\%\times 2000000= 120000 \text{ (đồng).}\]
			Sau 2 năm, tiền lãi của mẹ Hoa là
			\[2.1 200 000= 240000 \text{ (đồng).}\] 
			Sau 2 năm, tiền lãi của mẹ Hoa là
			\[2000000+ 240000=2240000 \text{ (đồng).}\] 
			\i Tiền lãi mẹ Hoa nhận được là
			\[2360000 -2000000 = 360000 \text{ (đồng).}\]
			Cứ sau 1 năm, tiền lãi được cộng thêm 120 000 đồng. Do đó, để được số tiền lãi là 360 000 đồng thì mẹ Hoa sẽ mất
			\[ 360000 : 120000 = 3 \text{ (năm).}\]
		\end{enumerate}
		\begin{mku}
			\textbf{Tổng quát}
			\begin{enumerate}[--,leftmargin=*]
				\i Giả sử số tiền gốc là $T_0$, lãi suất đơn là $a\%/$ năm.
				\i Số tiền lãi thu được sau $n$ năm là $n\times a\%\times T_0.$
				\i Số tiền cả gốc lẫn lãi thu được sau $n$ năm là 
				\[n\times a\%\times T_0 + T_0 = T_0 \left(n \times a\% + 1\right).\]
			\end{enumerate}
		\end{mku}
	}
\end{vd}
\begin{vd}
	Để có đủ tiền xây nhà, bố em đã làm hợp đồng vay vốn từ ngân hàng với số tiền 100 triệu đồng với lãi suất đơn 1\%/tháng và chọn hình thức thanh toán cho ngân hàng cả vốn lẫn lãi sau 24 tháng kể từ ngày ký hợp đồng. Vậy khi kết thúc hợp đồng, bố em phải chi trả cho ngân hàng với số tiền là bao nhiêu? 
	\loigiai{
		Đổi 1\%/tháng = 12\%/năm
		
		Số tiền bố chi trả cho ngân hàng là
		\[100000000 \times \left(2\times 12\% +1\right) = 124000000 \text{ (đồng).}\] 
	}
\end{vd}
\begin{vd}
	Chú Việt gửi vào ngân hàng 1 tỉ đồng với lãi kép 5\%/năm. Tính số tiền cả gốc lẫn lãi chú Việt nhận được sau khi gửi ngân hàng 3 năm.
	\loigiai{
		Giả sử tiền gốc ban đầu của chú Việt là $T_0$.
		
		Sau năm thứ nhất, tiền cả gốc lẫn lãi của chú Việt là 105\%. $T_0$
		
		Sau năm thứ nhất, tiền cả gốc lẫn lãi của chú Việt là $T_1 = T_0 + 5\%T_0 = 1,05 T_0$
		
		Sau năm thứ 2, tiền cả gốc lẫn lãi của chú Việt là $T_2 = 1,05T_1 = 1,05\cdot 1,05\cdot T_0 = 1,05^2T_0$.
		 
		Sau năm thứ 3, tiền cả gốc lẫn lãi của chú Việt là $T_3 = 1,05\cdot1,05\cdot 1,05\cdot T_0 = 1,05^3T_0$.
		Vậy sau khi gửi 3 năm. Chú Việt nhận được
		\[1000000000\cdot1,05^3 = 1157625000 \text{ (đồng)}\] 
		\begin{mku}
			\textbf{Tổng quát}
			\begin{enumerate}[--,leftmargin=*]
 				\i Giả sử số tiền gốc là $T_0$, lãi suất kép là $m\%$/năm.
				\i Số tiền cả gốc lẫn lãi thu được sau $n$ năm là $\left(1 + m \%\right)^n \times T_0$
			\end{enumerate}
		\end{mku}
	}
\end{vd}
\begin{vd}
	Bác Hưng có 2 tỉ đồng tiền nhàn rỗi. Có 2 phương án cho bác Hưng
	\begin{enumerate}[--,leftmargin=*]
		\i Phương án 1: Gửi ở ngân hàng A với lãi suất đơn 3\% trên nửa năm.
		\i Phương án 2: Gửi ở ngân hàng B với lãi suất kép 5,5\%/năm.
	\end{enumerate}
	Hỏi phương án nào có lợi hơn nếu thời gian gửi của bác Hưng là:
	\begin{enumerate}[a),leftmargin=*]
		\i Sau 1 năm.
		\i Sau 10 năm.
	\end{enumerate}
	\loigiai{
		Đổi: 3\%/ nửa năm = 6\%/ năm.
		
		Công thức tính tiền cả gốc lẫn lãi theo phương án 1: $T_0\left(n \times 6\% +1\right)$.
		
		Công thức tính tiền cả gốc lẫn lãi theo phương án 2: $\left(1,055\right)^nT_0$.
		\begin{enumerate}[a),leftmargin=*]
			\i Sau 1 năm, phương án 1 thu được
			\[2\left( 1\times 6\%+1 \right)=2,12 \text{ (tỉ đồng).}\] 
			Sau 1 năm, phương án 2 thu được
			\[1,055^1\times 2=2,11 \text{ (tỉ đồng).}\] 
			Nếu gửi sau 1 năm thì phương án 1 có lợi hơn.
			\i Sau 10 năm, phương án 1 thu được
			\[2\left( 10\times 6\%+1 \right)=3,2 (tỉ đồng).\] 
			Sau 4 năm, phương án 2 thu được
			\[1,05^{10}\times 2 \approx 3,42 \text{ (tỉ đồng).}\] 
			Nếu gửi sau 10 năm thì phương án 2 có lợi hơn.
		\end{enumerate}
	}
\end{vd}
\subsection{BÀI TẬP TỰ LUYỆN}
\Opensolutionfile{loigiaichung}[loigiaichuong32]
\begin{bt}
	Em hãy trả lời tình huống đặt ra ở đầu bài.
	\begin{loigiaichuong32}
		Gợi ý: Bác Sơn có thể đem tiền đến gửi ngân hàng để đồng tiền “sinh sôi”.
	\end{loigiaichuong32}
\end{bt}
\begin{bt}
	Cô Thanh có 100 triệu đồng tiền nhàn rỗi. Cô Thanh cho cô Dung vay để mua nhà ở với lãi suất đơn 1\%/ tháng.
	\begin{enumerate}[a),leftmargin=*]
		\i Sau 2 năm, cô Dung phải trả cô Thanh bao nhiêu tiền?
		\i Cô Dung muốn trả hết nợ trước khi số tiền vay lên đến 150 triệu đồng. Hỏi thời điểm muộn nhất cô Dung cần phải thanh toán là khi nào?
	\end{enumerate}
	\begin{loigiaichuong32}
		Đáp số
		\begin{enumerate}[a),leftmargin=*]
			\i 124 triệu.
			\i Trước khi sang tháng thứ 51.
		\end{enumerate}
	\end{loigiaichuong32}
\end{bt}
\begin{bt}
	Chị Thanh gửi tiền vào ngân hàng theo phương thức lãi đơn. Để sau 2,5 năm chị Thanh rút được cả vốn lẫn lãi số tiền là 10892000 đồng với lãi suất $\frac{5}{3}$\% một quý thì chị ấy  phải gửi tiết kiệm số tiền là bao nhiêu?
	\begin{loigiaichuong32}
		Đáp số: 9336000 đồng.
	\end{loigiaichuong32}
\end{bt}
\begin{bt}
	Anh Sơn mua một chiếc xe máy 50 triệu đồng. Hình thức trả tiền như sau:
	\begin{enumerate}[--,leftmargin=*]
		\i Trả trước 20 triệu đồng
		\i Số tiền còn lại trả trong vòng 6 tháng, chia đều cho từng tháng với lãi suất 1\% số tiền còn nợ. Em hãy tính xem mỗi tháng anh Sơn phải trả cho cửa hàng xe máy bao nhiêu tiền?
	\end{enumerate}
	\begin{loigiaichuong32}
		Hướng dẫn giải
		\begin{center}
			\begin{tabular}{|c|c|c|c|}
			\hline
			Tháng&Tiền gốc	&Tiền lãi&	Tiền phải trả\\
			\hline
			1&	5000000 đồng&	300000 đồng&	5300000 đồng\\
			\hline
			2&	5000000 đồng&	250000 đồng&	5250000 đồng\\
			\hline
			3&	5000000 đồng&	200000 đồng&	5200000 đồng\\
			\hline
			4&	5000000 đồng&	100000 đồng&	5100000 đồng\\
			\hline
			6&	5000000 đồng&	50000 đồng&	5050000 đồng\\
			\hline
		\end{tabular}
		\end{center}
	\end{loigiaichuong32}
\end{bt} 
\Closesolutionfile{loigiaichung}


\def\i{\item}
\graphicspath{{../pictures/vande31/}}
\chapter{LÀM QUEN VỚI TOÁN KINH TẾ}
\section{DOANH THU -- CHI PHÍ --  LỢI NHUẬN}
\begin{center}
	\textit{Hoa bán đồ ăn vặt trên mạng. Tháng vừa qua, tiền bán hàng Hoa thu được 2 triệu đồng, tiền nhập hàng Hoa bỏ ra 1,2 triệu đồng. Vậy mà vẫn không thấy tiền đâu. Thật kì lạ!!!}
\end{center}
\begin{vd}
	Một bác nông dân quyết định đi buôn bò. Bác mua một con bò giá 18 triệu đồng. Sau đó, bác bán 20 triệu đồng. Bác lại mua con bò khác 19 triệu đồng rồi bán đi với giá 22 triệu đồng.
	\begin{enumerate}[a),leftmargin=*]
		\i Hỏi sau 2 thương vụ này, bác nông dân kiếm được bao nhiêu tiền chênh lệch?
		\i Giả sử bác nông dân phải bỏ ra 3\% số tiền bán bò cho người thiệu khách đến mua (tiền hoa hồng). Hỏi bác Sơn còn lãi được bao nhiêu?
	\end{enumerate}
	\loigiai{
		\begin{enumerate}[a),leftmargin=*]
			\i Tổng số tiền bác nông dân bán 2 con bò là
			\[20+22=42 \text{ (triệu đồng).}\] 
			Tổng số tiền bác nông dân mua vào 2 con bò là
			\[18+19=37 \text{ (triệu đồng).}\] 
			Bác nông dân kiếm được số tiền chênh lệch là
			\[42-37=5 \text{ (triệu đồng).}\] 
			\i Số tiền bác nông dân chi cho người giới thiệu là
			\[\frac{3.42}{100}=1,26 \text{ (triệu đồng).}\] 
			Số tiền lãi còn lại của bác Sơn là
			\[5-1,26=3,74 \text{ (triệu đồng).}\] 
			\begin{mku}
				\textbf{Trong ví dụ trên:}
				\begin{enumerate}[--,leftmargin=*]
					\i Tổng số tiền bác nông dân có được khi bán 2 con bò gọi là \textbf{doanh thu}.
					\i Tổng số tiền bác nông dân bỏ ra mua 2 con bò về gọi là \textbf{giá vốn}.
					\i Số tiền chênh lệch giữa \textbf{doanh thu} và \textbf{giá vốn} (là 5 triệu đồng) gọi là \textbf{lợi nhuận gộp}.
					\i Số tiền bác nông dân chi ra cho người giới thiệu khách hàng gọi là \textbf{chi phí}.
					\i Số tiền còn lại sau khi đã trừ hết chi phí được gọi là \textbf{lợi nhuận ròng}.
				\end{enumerate}
				
			\end{mku}
		\end{enumerate}
	}
\end{vd}
\begin{vd}
	Bác Hưng nuôi một đàn lợn trong 90 ngày, được 3 tấn và bán với giá 60 000 đồng/kg. Biết bác Hưng mua giống hết 80 triệu đồng, tiền thức ăn bình quân hết 500 000 đồng/ngày cho cả đàn. Hỏi:
	\begin{enumerate}[a),leftmargin=*]
		\i Doanh thu bán lợn của bác Hưng là bao nhiêu? Giá vốn của đàn lợn là bao nhiêu?
		\i Lợi nhuận gộp bác Hưng thu được từ việc bán lợn là bao nhiêu?
		\i Theo em, để nuôi đàn lợn này, bác Hưng đã bỏ ra thêm những khoản chi phí nào? Giả sử những chi phí ấy bằng 10\% doanh thu. Hãy tính lợi nhuận ròng bác Hưng thu được.
		\i Giả sử bác Hưng không bán ngay mà muốn đợi thêm 10 ngày nữa để giá lợn lên đến 61000 đồng/kg. Em có đồng ý với phương án trên của bác Hưng không? Vì sao? (biết rằng chi phí như ở ý c) vẫn là 10\% lợi nhuận).
	\end{enumerate}
	\loigiai{
		\begin{enumerate}[a),leftmargin=*]
			\i Doanh thu bán lợn của bác Hưng là
			\[3000\times 60000=180000000 \text{ (đồng).}\] 
			Tiền thức ăn cho đàn lợn của bác Hưng là
			\[500000\times 90=45000000 \text{ (đồng).}\] 
			Giá vốn của đàn lợn trên là
			\[80000000+45000000=135000000 \text{ (đồng).}\] 
			\i Lợi nhuận gộp bác Hưng thu được từ việc bán lợn là
			\[180000000-135000000=45000000 \text{ (triệu đồng)}\] 
			\i Những chi phí bác Hưng còn phải bỏ ra: tiền điện nước, tiền thuốc tiêm phòng, tiền thuê nhân công (nếu có), tiền khấu hao chuồng trại, tiền vận chuyển (nếu có)…
			Giả sử những chi phí ấy bằng 10\% doanh thu. Thì  lợi nhuận ròng bác Hưng thu được là
			\[45000000-10\%.180000000=27000000 \text{ (đồng).}\]
			\i Khi giá lợn lên 61 000 đồng/kg thì doanh thu thêm là
			\[3000\times \left( 61000-60000 \right)=3000000\]
			Chi phí thức ăn trong vòng 10 ngày là
			\[500000\times 10=5000000 \text{ (đồng).}\] 
			Do vậy, Bác Hưng không nên nuôi thêm đàn lợn này.
		\end{enumerate}
	}
\end{vd}
\begin{vd}
	Chị Vân Anh có một cửa hàng kinh doanh trà sữa. Mỗi cốc trà sữa chị làm có giá vốn là 25.000 đồng/cốc và bán cho khách hàng với giá 15 000 đồng/cốc. Chị thuê cửa hàng hết 8 000 000 đồng/ tháng. Mỗi tháng, chị mất thêm 1 500 000 đồng tiền chi phí điện nước, 1 000 000 đồng tiền thuế và 10 000 000 đồng tiền lương cho nhân viên. Hỏi:
	\begin{enumerate}[a),leftmargin=*]
		\i Lợi nhuận một cốc trà sữa của chị Vân Anh là bao nhiêu?
		\i Tổng chi phí để cửa hàng của chị Vân Anh hoạt động trên một tháng là bao nhiêu?
		\i Chị Vân Anh cần phải bán tối thiểu bao nhiêu cốc trà sữa một ngày để không bị lỗ?
		\i Để kiếm được lợi nhuận 15.000.000 đồng/tháng thì trung bình một ngày chị Vân Anh phải bán được bao nhiêu cốc trà sữa?
	\end{enumerate}
	\loigiai{
		\begin{enumerate}[a),leftmargin=*]
			\i Lợi nhuận gộp trên một cốc trà sữa của chị Vân Anh là
			\[25000-15000=10000 \text{ (đồng/cốc)}\] 
			\i Chi phí hoạt động 1 tháng của cửa hàng là
			\[1500000+1000000+10000000=12500000 \text{ (đông).}\] 
			\i Với số tiền chi phí là 12 500 000 đồng/tháng và lợi nhuận mỗi cốc trà sửa là 10 000 đồng/cốc thì để hòa vốn, chị Vân Anh cần bán
			\[12500000:10000=1250 \text{ (cốc/tháng)}\] 
			Vậy mỗi ngày, chị Vân Anh cần bán
			\[1250:30\approx 42 \text{ (cốc/tháng).}\] 
			\i Để kiếm được lợi nhuận 15000000 đồng/tháng thì một ngày chị Vân Anh cần bán tối thiểu
			\[42+\left( 15000000:10000 \right):30\approx 192 \text{ (cốc).}\] 
		\end{enumerate}
		\begin{mku}
			\textbf{Trong ví dụ trên:}
			\begin{enumerate}[--,leftmargin=*]
				\i Mỗi ngày chị Vân Anh phải bán khoảng 42 cốc trà sữa thì mới hòa vốn. Ta gọi con số 42 cốc/ngày là điểm hòa vốn của cửa hàng.
				\i Doanh thu từ 42 cốc trà sữa này là $25000.42=1050000$ là doanh thu hòa vốn trong ngày. 
			\end{enumerate}
		\end{mku}
	}
\end{vd}
\subsection{BÀI TẬP TỰ LUYỆN}
\Opensolutionfile{loigiaichung}[loigiaichuong33]
\begin{bt}
	Trong tình huống đặt ra ở đầu bài, theo em có những nguyên nhân nào khiến Hoa không có tiền dù việc việc bán hàng có lãi?
	\begin{loigiaichuong33}
		Có thể có những khả năng sau đây
		\begin{enumerate}[--,leftmargin=*]
			\i Hoa làm rơi tiền.
			\i Hoa sử dụng tiền không có kế hoạch, nhầm lẫn…
			\i Hoa chi phí cho việc bán hàng quá nhiều. Lợi nhuận Hoa có là 800.000 đồng mới chỉ là lợi nhuận gộp chưa tính các chi phí như tiền điện thoại, tiền cước vận chuyển, tiền chi hoa hồng cho bạn bè giới thiệu (nếu có)
		\end{enumerate}
	\end{loigiaichuong33}
\end{bt}
\begin{bt}
	Một công ty phát hành sách có lợi nhuận trung bình là 25000 đồng/cuốn. Tiền thuê địa điểm là 10 triệu đồng/tháng, trả lương nhân viên hết 35 triệu đồng/ tháng, tiền chi phí khác là 5 triệu đồng/ tháng.
	\begin{enumerate}[a),leftmargin=*]
		\i Tính điểm hòa vốn theo tháng.
		\i Để có lợi nhuận 20 triệu đồng/tháng thì công ty cần phải bán bao nhiêu cuốn sách?
	\end{enumerate}
	\begin{loigiaichuong33}
		\begin{enumerate}[a),leftmargin=*]
			\i 2000 cuốn/tháng.
			\i 2800 cuốn/tháng
		\end{enumerate}
	\end{loigiaichuong33}
\end{bt}
\begin{bt}
	Trường THCS Mạc Đĩnh Chi tổ chức hội chợ gây quỹ từ thiện. Lớp 6A dự định xây dựng một gian hàng kinh doanh đồ lưu niệm. Biết rằng mỗi sản phẩm có giá nhập 3000 đồng, giá bán 6000 đồng, tiền in tờ rơi quảng cáo hết 70000 đồng, tiền trang trí gian hàng hết 140000 đồng, các chi phí còn lại do hội phụ huynh lớp tài trợ.
	\begin{enumerate}[a),leftmargin=*]
		\i Tính điểm hòa vốn và doanh thu hòa vốn của dự án
		\i Để có số tiền lãi 510000 đồng ùng hộ quỹ từ thiện thì lớp 6A cần phải bán bao nhiêu sản phẩm?
	\end{enumerate}
	\begin{loigiaichuong33}
		\begin{enumerate}[a),leftmargin=*]
			\i 70 sản phẩm và 420000 đồng
			\i 240 sản phẩm.
		\end{enumerate}
	\end{loigiaichuong33}
\end{bt}
\Closesolutionfile{loigiaichung}
\section{Đọc thêm: CHI PHÍ CƠ HỘI}
Mẹ cho em 25.000 đồng để săn sáng. Nếu ăn xôi thì hết 15.000 đồng/bát, nếu ăn phở thì hết 25.000 đồng/bát. em sẽ chọn phương án nào?

\textit{Cuộc sống đôi khi đặt chúng ta ở tình thế phải lựa chọn một trong nhiều phương án.}
\begin{center}
	“Bâng khuâng đứng giữa hai dòng nước\\
Chọn một dòng hay để nước trôi?”
\end{center}
Trong kinh doanh, \textbf{\textit{Chi phí cơ hội}} được định nghĩa như phần thu nhập mất đi do đã không lựa chọn một cơ hội đầu tư khác.
\begin{center}
	Chi phí cơ hội = lợi nhuận của lựa chọn hấp dẫn nhất - lợi nhuận của lựa chọn hiện tại
\end{center}
\begin{vd}
	Nhà bác Lan có một căn nhà không sử dụng. Bác Lan có 2 lựa chọn:
	\begin{enumerate}[1.,leftmargin=*]
		\i Cho công ty A thuê làm văn phòng với giá 8 triệu đồng/ tháng.
		\i Cho hàng xóm thuê để mở quán Internet với giá 10 triệu đồng/tháng.
	\end{enumerate}
	Ta thấy, căn nhà của bác Lan chỉ có một, không thể vừa cho công ty A thuê lại vừa cho bác hàng xóm thuê. Phương án tốt nhất là 2: cho bác hàng xóm thuê.
	Nếu chọn phương án 1 thì chi phí cơ hội (phần thu nhập bị mất đi) là $10-8=2$ (triệu đồng).
	Nếu chọn phương án 2 thì chi phí cơ hội bằng 0, tức là ta không bị bỏ lỡ cơ hội
\end{vd}
\begin{vd}
	Nếu ngày hôm đó em có một buổi hẹn đi đá bóng với bạn bè nhưng em lại từ chối để ở nhà hoàn thành bài tập về nhà môn toán. Giá trị em nhận được ở đây là kiến thức tốt cho bản thân, không bị phê bình còn giá trị em mất đi là đã bỏ lỡ một trận bóng đá hay, bỏ lỡ thời gian vui chơi với bạn bè. Vậy chi phí cơ hội em đánh đổi ở đây là hi sinh thời gian vui chơi, đá bóng với bạn bè để nhận lại được những kiến thức có trong bài tập, hoàn thành nhiệm vụ mà giáo viên đã giao.
\end{vd}
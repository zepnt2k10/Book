\usepackage[most,many]{tcolorbox} 
\tcbuselibrary{skins}
\usetikzlibrary{shadings}
\usetikzlibrary{shadows}
\usepackage{varwidth}
\usepackage{collect}
\definecollection{btcol}
\definecollection{vdcol}
\definecollection{excol}
%%%==========Lệnh lời giải==========%%%
\newcommand{\loigiai}[1]{}
%%%==========Môi trường chèn hình==========%%%
\newdimen\widthimmini
\newdimen\heightimmini
\newdimen\textimagewidth
\newbox\imbox
\newcommand{\immini}[3][-12pt]{
		\setbox\imbox=\vbox{\hbox{#3}}
		\widthimmini=\wd\imbox
		\heightimmini=\ht\imbox	
		\textimagewidth=\dimexpr\linewidth - 1.075\widthimmini\relax
\par\vspace*{\dimexpr#1\relax}\noindent{\ignorespaces\begin{minipage}[t]{\textimagewidth}
	\vspace*{0pt}\ignorespaces
	\begingroup
	#2
	\endgroup
	\end{minipage}~
	\begin{minipage}[t]{\widthimmini}
	\vspace*{-1pt}
\begin{center}
	\begingroup
		#3
	\endgroup
\end{center}
\end{minipage}}}

\DeclareSymbolFont{yhlargesymbols}{OMX}{yhex}{m}{n}
\DeclareMathAccent{\wideparen}{\mathord}{yhlargesymbols}{"F3}

\newlength\boxwidth
\newbool{btracnghiem}
\newbool{blbaitap}
\newbool{blvd}
%%%=======lệnh cho biết môi trường loại gì
%%\newcommand{\loaivd}{\global\setbool{blvd}{true}\global\setbool{bltracnghiem}{false}
%%	\global\setbool{blbaitap}{false}}
%%\newcommand{\loaibt}{\global\setbool{blbaitap}{true}\global\setbool{blvd}{false}
%%	\global\setbool{bltracnghiem}{false}}
%%\newcommand{\loaitn}{\global\setbool{bltracnghiem}{true}\global\setbool{blvd}{false}
%%	\global\setbool{blbaitap}{false}}


\definecolor{nennam}{RGB}{238,238,238}
\definecolor{maukhung}{RGB}{255,246,143}
\def\boxhead{
    \settowidth\boxwidth{Ví dụ 99}
\begin{tcolorbox}[breakable,top=0.5cm,bottom=-0.1cm,enhanced,before skip=10pt,after skip=5mm,
%	after upper={\par\hfill\textcolor{green!40!black}%
%		{\itshape Nguồn}},
    boxsep=2mm,attach boxed title to top left={xshift=5mm,yshift=-\tcboxedtitleheight},boxrule=0.75pt,
    colback=nennam!25,rounded corners=southeast,arc=3.5pt,arc is curved,drop fuzzy shadow,left=5pt,right=5pt,
	title={\makebox[1.15\boxwidth][c]{}},
	 boxed title style={interior empty,frame code={
	 	       \filldraw[fill=nennam!80,line width=0.7pt] ([xshift=5mm]frame.north east) coordinate (B) arc (180:0:1mm)
	 	         ([xshift=3mm]frame.north west) coordinate (A)   arc(0:180:1mm);
	 	        \filldraw[fill=nennam!25,line width=0.65pt]
	 	        	 	       ([shift={(-.115,.1)}]A)--([shift={(.1cm,.1)}]B)[rounded corners=1mm]--([shift={(0,0)}]B)--([shift={(-0.1cm,-15pt)}]B)
	 	        	 	       --([shift={(0.1cm,-15pt)}]A)
	 	        	 	       --([shift={(0,0)}]A)[rounded corners=1mm]--([shift={(-.115,.1)}]A);
	 	       \path ([shift={(0pt,-6pt)}]A)--([shift={(0,-6pt)}]B) 
	 	      node[midway]{\refstepcounter{vd}\bfseries\fontfamily{pag}\selectfont Ví dụ \thevd};  
	  }}	 	      
      ]}
 \newcommand{\boxend}{\end{tcolorbox}}
%%%%==========Môi trường ví dụ==========%%%
\newcounter{vd}[section]

\newcommand{\ghichuvd}{}
\BeforeBeginEnvironment{vd}{\global\setbool{blvd}{true}
	\global\setbool{blbaitap}{false}\global\setbool{bltracnghiem}{false}}
\newenvironment{vd}[1][]{\Pluudapan%điều khiển lưu đáp
	\boxhead \ifthenelse{\equal{#1}{}}{\gdef\ghichuvd{}}{\gdef\ghichuvd{\par\hfill{(\textit{#1})}}}}
{\ghichuvd\boxend}
\newcommand{\khongindapanvd}{\renewcommand{\boxend}{\end{tcolorbox}}
\renewcommand{\loigiai}[1]{}
}

%\BeforeBeginEnvironment{vd}{\global\setbool{blvd}{true} 
%	\global\setbool{blbaitap}{false}\global\setbool{bltracnghiem}{false}}		
%\newenvironment{vd}[1][]{\Pluudapan%điều khiển lưu đáp 
%	\boxhead \ifthenelse{\equal{#1}{}}{}{{(\textit{#1})}}\unskip}
%	{\boxend}
%\newcommand{\khongindapanvd}{\renewcommand{\boxend}{\end{tcolorbox}}
%	\renewcommand{\loigiai}[1]{}
%}
\newcommand{\indapanvd}{
	\renewcommand{\loigiai}[1]{
\renewcommand{\boxend}{\end{tcolorbox}\vspace*{-15pt}\begin{center}
	{\fontfamily{qag}\bfseries\selectfont Lời giải}\\[-19pt]
	\end{center}\vspace*{-15pt}
	\hspace*{-0.2cm}
	##1}}}
%%%=================moi truong bai tap=================%%%
\newcommand{\namebt}{Bài tập }
\def\boxheadbt{\begin{tcolorbox}[enhanced,breakable,
	before skip=2mm,after skip=9pt,
    left=1mm,right=1mm,top=2mm,bottom=1mm,
	colback=gray!10,boxrule=0pt,
	arc=3mm,rounded corners,
	colbacklower=white, frame hidden]}
\newcommand{\boxendbt}{\end{tcolorbox}}
\makeatletter
\newcounter{bt}
\AtBeginEnvironment{bt}{\global\setbool{blvd}{false}
	\global\setbool{blbaitap}{true}\global\setbool{bltracnghiem}{false}}
\newenvironment{bt}[1][]{
	\refstepcounter{bt}\Pluudapan%điều khiển lưu đáp 
	\ifthenelse{\equal{#1}{}}{\boxheadbt{{\bfseries \namebt \thebt. }}}
	{\boxheadbt{{\bfseries\namebt \thebt\ (#1). }}}}
	{\boxendbt} 
\makeatother
\newcommand{\khongindapanbaitap}{\renewcommand{\boxendbt}{\end{tcolorbox}}
	\renewcommand{\loigiai}[1]{
			\begin{Answer}
			\begingroup
			##1
			\endgroup
			\end{Answer}
	}
}
\newcommand{\indapanbaitap}{
	\renewcommand{\loigiai}[1]{
	\renewcommand{\boxendbt}{\end{tcolorbox}\vspace*{0pt}
%	\begin{center}
	{\bfseries\selectfont\noindent \hdg}\\[0pt]
%	\end{center}
	##1
	}
	}
}
%%%============Hướng dẫn giải Bài tập về nhà==================%%%
\newlength{\gbt}
\settowidth{\gbt}{Huong dan giai bai tap}
\newcommand{\giaibaitapvenha}{ \setcounter{bt}{0}
	\par
	
	\medskip\noindent
	\tikz[overlay,remember picture,color=brown!50!black]{
		\node[anchor=west,inner sep=0pt](A){\includegraphics[height=20pt]{Khaibao/Key.pdf}};
		\draw[thick] (A.south east) coordinate (B)--+(1.2\gbt,0) arc (-90:90:12pt) --+(-1.3\gbt,0) coordinate (C);
		\path ($ (B)!0.5!(C) $) node[anchor = west]{{\bfseries\fontfamily{qag}\selectfont\; 
				Hướng dẫn giải bài tập}};
%		\addcontentsline{toc}{section}{\bf Hướng dẫn giải bài tập}  %%không thêm vào mục lục
	}	
	\bigskip}


%%%============Bài tập về nhà==================%%%
\newlength{\daibt}
\settowidth{\daibt}{Bai tap ve nha a}
\newcommand{\baitapvenha}{\setcounter{bt}{0}
\par

\medskip\noindent
\tikz[overlay,remember picture]{
\node[anchor=west,inner sep=0pt](A){\includegraphics[height=24pt]{Khaibao/Nha.pdf}};
\draw[thick] (A.south east) coordinate (B)--+(1.2\daibt,0) arc (-90:90:12pt) --+(-1.3\daibt,0) coordinate (C);
\path ($ (B)!0.5!(C) $) node[anchor = west]{{\bfseries\fontfamily{qag}\selectfont\; 
Bài tập luyện tập}};
%\addcontentsline{toc}{section}{\bf Bài tập luyện tập}  %%không thêm vào mục lục
}	
\bigskip}
%%%============Môi trường đóng khung============%%%
\RequirePackage[framemethod=default]{mdframed}
\newmdenv[skipabove=7pt,
skipbelow=7pt,
rightline=false,
leftline=true,
topline=false,
bottomline=false,
backgroundcolor=black!5,
innerleftmargin=5pt,
innerrightmargin=5pt,
innertopmargin=5pt,
leftmargin=0cm,
rightmargin=0cm,
linewidth=3pt,
innerbottommargin=5pt]{cBox}

\newmdenv[skipabove=7pt,
skipbelow=7pt,
rightline=false,
leftline=true,
topline=false,
bottomline=false,
linecolor=orange,
backgroundcolor=nennam,
innerleftmargin=5pt,
innerrightmargin=5pt,
innertopmargin=5pt,
leftmargin=0cm,
rightmargin=0cm,
linewidth=3pt,
]{bBox}

\newmdenv[skipabove=7pt,
skipbelow=7pt,
rightline=false,
leftline=true,
topline=false,
bottomline=true,
backgroundcolor=nennam,
innerleftmargin=5pt,
innerrightmargin=5pt,
innertopmargin=5pt,
leftmargin=0cm,
rightmargin=0cm,
linewidth=2pt,
]{dnBox}
%%%=============Moi truong dn==================%%%
\newcounter{sodn}[section]
\newenvironment{dn}{\begin{dnBox}\refstepcounter{sodn}
{\fontfamily{qag}\selectfontĐịnh nghĩa \arabic{chapter}.\arabic{section}.\thesodn}.\;
}
{\end{dnBox}}

%%%==============Moi truong dang====================%%%
\definecolor{xanh}{HTML}{04AA6D}
\definecolor{xanhdatroi}{HTML}{2196F3}
\definecolor{cam}{HTML}{ff9800}
\definecolor{do}{HTML}{f44336}

\newcounter{sodang}[section]
\makeatletter
\def\calcLength(#1,#2)#3{%
\pgfpointdiff{\pgfpointanchor{#1}{center}}%
             {\pgfpointanchor{#2}{center}}%
\pgf@xa=\pgf@x%
\pgf@ya=\pgf@y%
\FPeval\@temp@a{\pgfmath@tonumber{\pgf@xa}}%
\FPeval\@temp@b{\pgfmath@tonumber{\pgf@ya}}%
\FPeval\@temp@sum{(\@temp@a*\@temp@a+\@temp@b*\@temp@b)}%
\FProot{\FPMathLen}{\@temp@sum}{2}%
\FPround\FPMathLen\FPMathLen5\relax
\global\expandafter\edef\csname #3\endcsname{\FPMathLen}
}
\makeatother

\newmdenv[skipabove=7pt,
skipbelow=7pt,
rightline=false,
leftline=true,
topline=false,
bottomline=false,
linecolor=xanhdatroi,
backgroundcolor=nennam,
innerleftmargin=5pt,
innerrightmargin=5pt,
innertopmargin=5pt,
leftmargin=0cm,
rightmargin=0cm,
linewidth=2pt,
]{dangBox}

\newenvironment{dang}[1][]{\begin{dangBox}\refstepcounter{sodang}\setcounter{vd}{0}
\begin{tikzpicture}[remember picture,overlay]
\path (-6pt,7pt) node[fill=xanhdatroi,inner sep=6pt,text width=0.975\linewidth,anchor=north west] (AT)
	{{\bfseries\fontfamily{pag}\selectfont\color{white}Dạng~\arabic{section}.\thesodang. #1}};
\path (AT.north east) coordinate (A)
		(AT.south east) coordinate (B)
		($ (A)!0.5!(B) $) coordinate (C);
\calcLength(A,C){mylen}
\draw[xanhdatroi,ultra thick,fill=white] ([xshift=0.5*\mylen]C) 
coordinate (TC) let  \p1 = ($ (C) - (A) $),  \n1 = {veclen(\x1,\y1)} 
 in circle (1.25*\n1);
 \path (TC) let  \p1 = ($ ([xshift=-12pt]C) - (A) $),  \n1 = {veclen(\x1,\y1)} 
  in  node{\includegraphics[width=\n1]{Khaibao/pencil}};
\end{tikzpicture}

\par\bigskip\noindent
}
{\end{dangBox}}

%%================Moi truong phuong phap===================%%%
\newmdenv[skipabove=7pt,
skipbelow=7pt,
rightline=false,
leftline=true,
topline=false,
bottomline=false,
linecolor=cam,
backgroundcolor=nennam,
innerleftmargin=5pt,
innerrightmargin=5pt,
innertopmargin=5pt,
leftmargin=0cm,
rightmargin=0cm,
linewidth=2pt,
]{ppBox}

\newenvironment{pp}[1][]{\begin{ppBox}
\begin{tikzpicture}[remember picture,overlay]
\path (-6pt,7pt) node[fill=cam,inner sep=6pt,text width=0.975\linewidth,anchor=north west] (AT)
	{{\bfseries\fontfamily{pag}\selectfont\color{white}Phương pháp #1}};
\path (AT.north east) coordinate (A)
		(AT.south east) coordinate (B)
		($ (A)!0.5!(B) $) coordinate (C);
\calcLength(A,C){mylen}
\draw[cam,ultra thick,fill=white] ([xshift=0.5*\mylen]C) 
coordinate (TC) let  \p1 = ($ (C) - (A) $),  \n1 = {veclen(\x1,\y1)} 
 in circle (1.25*\n1);
 \path (TC) let  \p1 = ($ ([xshift=-12pt]C) - (A) $),  \n1 = {veclen(\x1,\y1)} 
  in  node{\includegraphics[width=\n1]{Khaibao/Key}};
\end{tikzpicture}

\par\bigskip\noindent
}
{\end{ppBox}}

%%================Moi truong phan tich===================%%%
\newenvironment{pt}
{\begin{TCBox}{\fontfamily{qag}
			\bfseries\selectfont Phân tích:\\ }}
	{\end{TCBox}}

%%%=============Moi truong hq==================%%%
\newcounter{sohq}[section]
\newenvironment{hq}{\begin{cBox}\refstepcounter{sohq}
{\fontfamily{qag}\selectfontHệ quả \arabic{chapter}.\arabic{section}.\thesohq}.\;
}{\end{cBox}}

%%%=============Moi truong bd==================%%%
\newcounter{sobd}[section]
\newenvironment{bd}{\begin{cBox}\refstepcounter{sobd}
		{\fontfamily{qag}\selectfont Bổ đề 
			\arabic{section}.\thesobd}.\;
	}{\end{cBox}}

%%%=============Moi truong tc==================%%%
\newmdenv[skipabove=5pt,
skipbelow=5pt,
rightline=false,
leftline=true,
topline=false,
bottomline=false,
backgroundcolor=nennam,
innerleftmargin=5pt,
innerrightmargin=5pt,
innertopmargin=5pt,
leftmargin=0cm,
rightmargin=0cm,
linewidth=3pt,
innerbottommargin=5pt
]{TCBox}
\newcounter{sotc}[section]
\newenvironment{tc}
{\begin{TCBox}{\fontfamily{qag}\refstepcounter{sotc}
\bfseries\selectfont Tính chất \arabic{section}.\thesotc.}}
{\end{TCBox}}
%%%=============Moi truong chứng minh==================%%%
\newenvironment{proof}{\noindent{ \bf\textit{ Chứng minh}}\\}{}
%%%=============Moi truong nx==================%%%
\newenvironment{nx}
{\begin{TCBox}{\fontfamily{qag}
\bfseries\selectfont Nhận xét: }}
{\end{TCBox}}
%%%=============Moi truong lời giải==================%%%
\newenvironment{dapan}{\noindent{ \bf{ \hdg}}\\}{}
%%%=============Moi truong đề thi==================%%%
\newenvironment{dethi}{\refstepcounter{bt}\noindent{ \bf{ Câu \thebt. }}}{}
%%%=============Moi truong dl==================%%%
\newmdenv[skipabove=7pt,
skipbelow=7pt,
rightline=false,
leftline=true,
topline=false,
bottomline=false,
backgroundcolor=nennam,
innerleftmargin=5pt,
innerrightmargin=5pt,
innertopmargin=5pt,
leftmargin=0cm,
rightmargin=0cm,
linewidth=3pt,
innerbottommargin=5pt
]{DLBox}

\newcounter{sodl}[section]
\newenvironment{dl}
{\begin{DLBox}{\fontfamily{qag}\refstepcounter{sodl}
\bfseries\selectfont Định lý \arabic{section}.\thesodl.}}
{\end{DLBox}}
%%%=============Moi truong cy==================%%%
\newmdenv[skipabove=7pt,
skipbelow=7pt,
rightline=false,
leftline=true,
topline=false,
bottomline=false,
backgroundcolor=black!5!white,
innerleftmargin=5pt,
innerrightmargin=5pt,
innertopmargin=5pt,
leftmargin=0cm,
rightmargin=0cm,
linewidth=3pt,
innerbottommargin=5pt
]{NoteBox}

\newenvironment{cy}
{\begin{NoteBox}{\fontfamily{qag}\refstepcounter{sodl}
\bfseries\selectfont Chú ý}. }
{\end{NoteBox}}

\newenvironment{note}
{\begin{NoteBox}{\fontfamily{qag}\refstepcounter{sodl}
\color{red}\bfseries\selectfont Chú ý}.}
{\end{NoteBox}}

%%%======================Box cho tiểu mục lục=================
\newmdenv[skipabove=7pt,
skipbelow=7pt,
rightline=false,
leftline=false,
topline=true,
bottomline=true,
backgroundcolor=nennam,
innerleftmargin=1pt,
innerrightmargin=1pt,
innertopmargin=5pt,
leftmargin=0cm,
rightmargin=0cm,
linewidth=2pt,
]{bhBox}
 
\newmdenv[skipabove=7pt,
skipbelow=7pt,
rightline=false,
leftline=true,
topline=false,
bottomline=false,
backgroundcolor=nennam,
innerleftmargin=5pt,
innerrightmargin=5pt,
innertopmargin=5pt,
leftmargin=12cm,
rightmargin=12cm,
linewidth=3pt,
]{bmBox}
%%%======================================================%%%
%%%====================Phần trắc nghiệm======================%%%
\usepackage{mathrsfs}  
\usepackage{ifthen}
\usepackage{etoolbox}
\usepackage{intcalc}%Tính mod
\usepackage{forloop}
%\usepackage{catchfile}%đưa file vào văn bản ngay lập tức
\newcounter{ex}[section]%số câu hỏi
\newcounter{dapan}[ex] %Số thứ tự a, b, c, d để đưa vào đáp án
\newcounter{numTrue}[ex]%Số thứ tự đáp án của câu
\newcommand{\dotEX}{.}%dấu kết thúc mỗi phương án
\newbool{bltracnghiem}%Kiểm tra có phải trắc nghiệm không
%%%==========khoanh đáp án===========%%%
\newcommand\excircled[1]{\tikz[baseline=(char.base)]{\node[shape=circle,draw,line width=0.65pt,
inner sep=1pt,fill=cyan!15,
minimum size =13pt,font=\fontfamily{pag}\fontsize{9.5pt}{10pt}\selectfont] (char){#1};}}
\newcommand\excircledda[1]{\tikz[baseline=(char.base)]{\node[shape=circle,draw,line width=0.65pt,
inner sep=1pt,fill=orange!15,
minimum size =13pt,font=\fontfamily{pag}\fontsize{10pt}{10pt}\bfseries\selectfont] (char){\color{orange}#1};}}
\newcommand\excircledxanh[1]{\tikz[baseline=(char.base)]{\node[shape=circle,draw,line width=0.65pt,
inner sep=1pt,
minimum size =13pt,font=\fontfamily{pag}\fontsize{10pt}{10pt}\bfseries\selectfont] (char){#1};}}
%%%==========môi trường ex==========%%%
%\AtBeginEnvironment{ex}{\vspace*{-1.25\parskip}}
\newcommand{\exhead}[1][]{\par\noindent\refstepcounter{ex}
%Thông tin đầu câu trắc nghiệm
%{\fontfamily{pag}\fontsize{11pt}{13pt}\selectfont\bfseries Câu \arabic{chapter}.\arabic{section}.\arabic{ex}.}\
{\fontfamily{pag}\fontsize{11pt}{12pt}\selectfont\bfseries Câu \arabic{ex}.\ignorespaces
\ifthenelse{\equal{#1}{}}{}{{(\textit{#1})}}}
\global\setbool{bltracnghiem}{true}
}%khai báo của môi trường ex
\newcommand{\exend}{%Kết thúc môi trường trắc nghiệm
\renewcommand{\loigiai}[1]{\begin{Answer}
	{\noindent{\small\usefont{T5}{HLButlong}{m}{n} \dauloigiaitracnghiem}
	##1\par\noindent
	Chọn đáp án \excircledda{\Alph{numTrue}}
  	}
 \end{Answer}
}
\renewcommand{\exend}{\shipoutAnswer}}%Kết thúc môi trường trắc nghiệm
%\newcommand{\indapan}{\shipoutAnswer}
\newcommand{\dapansaucauhoi}{
\renewcommand{\loigiai}[1]{\begin{Answer}
	{\noindent{\small\usefont{T5}{HLButlong}{m}{n} \dauloigiaitracnghiem}
	##1\par\noindent
	Chọn đáp án \excircledda{\Alph{numTrue}}
  	}
 \end{Answer}
}
\renewcommand{\exend}{\shipoutAnswer}
\renewcommand{\exbend}{\medskip\shipoutAnswer}}
\newcommand{\khongindapansaucauhoi}{
\renewcommand{\loigiai}[1]{\begin{Answer}
	{\noindent{\small\usefont{T5}{HLButlong}{m}{n} \dauloigiaitracnghiem}
	##1\par\noindent
	Chọn đáp án \excircledda{\Alph{numTrue}}
  	}
 \end{Answer}
}
\renewcommand{\exend}{}}
\newcommand{\khongluuloigiai}{\renewcommand{\loigiai}{}}
%%%================Môi trường ex================%%%
\AtBeginEnvironment{ex}{\global\setbool{blvd}{false}
	\global\setbool{blbaitap}{false}\global\setbool{bltracnghiem}{true}}
\newenvironment{ex}[1][]{\Pluudapan%điều khiển lưu đáp 
\refstepcounter{ex}%%%===phần tiêu đầu câu
\par\noindent{\selectfont\bfseries  Câu \arabic{ex}\ignorespaces}\ifthenelse{\equal{#1}{}}{\,}{\,{ (\textit{#1}).}}\unskip%%%===phần ghi chú câu
\unskip}{\exend}
%%%%%%%Tự động chọn \motcot, \haicot, \boncot
     	\newlength\widthcha
        \newlength\widthchb
        \newlength\widthchc
        \newlength\widthchd
        \newlength\widthch
        \newlength\tabmaxwidth
        \newlength\fourthtabwidth
        \newlength\halftabwidth
        \newlength\fifthtabwidth
        \newlength\thirdtabwidth
\setlength\fourthtabwidth{\dimexpr0.225\linewidth-0.25\tabex-30pt\relax}
\setlength\halftabwidth{\dimexpr 0.475\linewidth-0.5\tabex-20pt\relax}
\newcommand{\choice}[4]{
\settowidth\widthch{#1}
\settowidth\widthchb{#2}
\ifdim\widthch<\widthchb\relax\setlength{\widthch}{\widthchb}\fi%
\settowidth\widthchc{#3}%
\ifdim\widthch<\widthchc\relax\setlength{\widthch}{\widthchc}\fi%
\settowidth\widthchd{#4}%
\ifdim\widthch<\widthchd\relax\setlength{\widthch}{\widthchd}\fi%
\ifdim\widthch<\fourthtabwidth
     \boncot{#1}{#2}{#3}{#4}
      \else\ifdim\widthch<\halftabwidth
          \haicot{#1}{#2}{#3}{#4}
        \else
          \motcot{#1}{#2}{#3}{#4}
      \fi
\fi}

\newcommand{\boncot}[4]{
	\setlength{\topsep}{0pt}%
   \setlength{\partopsep}{0pt}%
  \begin{tabbing}
  \hspace*{\tabex}\=
  \hspace*{\dimexpr0.25\linewidth-0.25\tabex\relax}
  \=\hspace*{\dimexpr0.25\linewidth-0.25\tabex\relax}
  \=\hspace*{\dimexpr0.25\linewidth-0.25\tabex\relax}
  \=\kill
  \phantom{a}\>
  \refstepcounter{dapan}\excircled{A} #1\dotEX
  \>
  \refstepcounter{dapan}\excircled{B} #2\dotEX
  \>
 \refstepcounter{dapan}\excircled{C} #3\dotEX
  \>
 \refstepcounter{dapan}\excircled{D} #4\dotEX\vspace*{-1.25\parskip}
\end{tabbing}
%\end{center}
}
   
\newcommand{\haicot}[4]{
 \setlength{\topsep}{0pt}%
 \setlength{\partopsep}{0pt}%
 \begin{tabbing}
 \hspace*{\tabex}\= \hspace*{\dimexpr0.5\linewidth-0.5\tabex+6pt\relax}\=\kill
  \phantom{a}\>\refstepcounter{dapan}\excircled{A} #1\dotEX\>
   \refstepcounter{dapan}\excircled{B} #2\dotEX\\ 
  \phantom{a}\>\refstepcounter{dapan}\excircled{C} #3\dotEX\>
    \refstepcounter{dapan}\excircled{D} #4\dotEX\vspace*{-1.25\parskip}
\end{tabbing}
 }
\newcommand{\motcot}[4]{
 \par\noindent
\hspace*{\tabex}\refstepcounter{dapan}\excircled{A} #1\dotEX\\
\hspace*{\tabex}\refstepcounter{dapan}\excircled{B} #2\dotEX\\ 
\hspace*{\tabex}\refstepcounter{dapan}\excircled{C} #3\dotEX\\
\hspace*{\tabex}\refstepcounter{dapan}\excircled{D} #4\dotEX
  }

%%%===========in đáp án==============%%%
\def\alist{}
\def\True{\ignorespaces\setcounter{numTrue}{\thedapan}{\ifnum\value{numTrue}=1 \listgadd{\alist}{A}\fi
	\ifnum\value{numTrue}=2 \listgadd{\alist}{B}\fi
	\ifnum\value{numTrue}=3 \listgadd{\alist}{C}\fi
	\ifnum\value{numTrue}=4 \listgadd{\alist}{D}\fi}} 
\renewcommand{\loigiai}[1]{\begin{Answer}
\ifbool{bltracnghiem}
	{\noindent{\small\usefont{T5}{HLButlong}{m}{n} \dauloigiaitracnghiem}
	#1\par\noindent
	Chọn đáp án \excircledda{\Alph{numTrue}}
  	}
  	{
  	\begingroup
  	#1
  	\endgroup
  	}
 \end{Answer}
}
%%%===============In đáp án ra bảng=================%%%
%\newcounter{itemcount}
%\renewcommand*{\do}[1]
% {\refstepcounter{itemcount} #1}
%\dolistloop{\alist}

%%%=================môi trường task================%%%
%\usepackage{tasks}
%\settasks{
%  item-indent =\dimexpr \parindent+\parindent, 
%  before-skip = -1.5pt , % undo paragraph skip
%  after-skip =-2pt , % undo paragraph skip
%  after-item-skip =-2pt% undo paragraph skip
%}
%\AtBeginEnvironment{tasks}
%	{\mbox{}\vspace*{\dimexpr -\baselineskip+3mm}}
%	
%%=================Bảng đáp án=================%%
\newcounter{itemcount}
\def\bangdapan{
\begin{center}
	\begin{tcolorbox}[enhanced,hbox,
			left=8mm,right=8mm,boxrule=0.4pt,
			colback=white,
			%drop fuzzy midday shadow=black!50!yellow,
			drop lifted shadow=black!50!yellow,arc is angular,
			before=\par\vspace*{-1mm},after=\par\bigskip]
			{\bfseries\fontfamily{qag}\selectfont ĐÁP ÁN BÀI \arabic{section}} 
		\end{tcolorbox}
		\vspace*{-0.5cm}
\end{center}

\setcounter{itemcount}{0}
 \renewcommand*{\do}[1]{\refstepcounter{itemcount} 
 {\makebox[22pt][l]{\ifthenelse{\value{itemcount}<10}
 {\phantom{.}\theitemcount}{\theitemcount}.}~{\fontfamily{qag}\selectfont##1}}\ignorespaces
 \ifthenelse{\equal{\value{itemcount}}{\value{ex}}}{.}{,}
 \ifthenelse{\equal{\intcalcMod{\value{itemcount}}{10}}{0}}
   {\newline}{}
 }
\noindent\dolistloop{\alist}}
%%=================Bảng đáp án dạng hộp=================%%
\newcounter{itemcounth}
\def\bangdapanhop{
\begin{center}
	\begin{tcolorbox}[enhanced,hbox,
			left=8mm,right=8mm,boxrule=0.4pt,
			colback=white,
			%drop fuzzy midday shadow=black!50!yellow,
			drop lifted shadow=black!50!yellow,arc is angular,
			before=\par\vspace*{-1mm},after=\par\bigskip]
			{\bfseries\fontfamily{qag}\selectfont ĐÁP ÁN BÀI \arabic{section}} 
		\end{tcolorbox}
		\vspace*{-0.5cm}
\end{center}

\setlength{\fboxrule}{1pt}
\setcounter{itemcounth}{0}
 \renewcommand*{\do}[1]{\refstepcounter{itemcounth} 
 \fcolorbox{black}{cyan!10}{\makebox[1cm][c]{
 {\theitemcounth.}\ignorespaces{\fontfamily{qag}\bfseries\selectfont\color{violet}##1 }}}
 \ifthenelse{\equal{\intcalcMod{\value{itemcounth}}{10}}{0}}
 {\par\noindent\ignorespaces}{}
 }
\noindent\dolistloop{\alist}}
%
%\begin{center}
%{\fontfamily{qag}\selectfont Đáp án dạng danh sách, dùng đưa vào các ứng dụng chấm điểm}
%\end{center}
%\renewcommand*{\do}[1]
% {\refstepcounter{itemcount} 
% \ifthenelse{\equal{\value{itemcount}}{\value{ex}}}
% {#1.}
% {#1,}
% }
% \setcounter{itemcount}{0}
%\noindent\dolistloop{\alist}

%%%=====================khung trac nghiem=====================%%%
\newtcolorbox{vdbox}[2][]{
	enhanced,
	before skip=2mm,after skip=2mm,
	top=2pt,bottom=5pt,left=6pt,right=3pt,
	colback=nencl,colbacktitle=nencl,
	colframe=khungcl,
	boxrule=0.2mm, %width=0.95\linewidth,
	attach boxed title to top left={xshift=1cm,yshift*=1mm-\tcboxedtitleheight},
	varwidth boxed title*=-3cm,
	boxed title style={
		frame code={
			\draw[color=khungcl,line width=0.2mm,fill=black!30]
			([yshift=-1mm,xshift=-1mm]frame.north west)
			arc[start angle=0,end angle=180,radius=1mm]
			([yshift=-1mm,xshift=1mm]frame.north east)
			arc[start angle=180,end angle=0,radius=1mm];
			\draw[color=khungcl,fill=nencl,line width=0.2mm]
			([xshift=-2mm]frame.north west) -- ([xshift=2mm]frame.north east)
			[rounded corners=1mm]--
			([xshift=1mm,yshift=-1mm]frame.north east) -- (frame.south east) --
			(frame.south west) -- ([xshift=-1mm,yshift=-1mm]frame.north west)
			[sharp corners]-- cycle;
		},
		interior engine=empty,
	},
	fonttitle=\fontfamily{pag}\fontsize{11pt}{11pt}\bfseries\selectfont,
	title={#2},#1
}
%%%=====================Het khung trac nghiem=====================%%%
%%%==============Moi truong vdex===============%%%
\newcommand{\exbend}{\end{vdbox}}
\AtBeginEnvironment{exb}{\global\setbool{bltracnghiem}{true}\global\setbool{blvd}{false}
\global\setbool{blbaitap}{false}}
\newenvironment{exb}[1][]{\refstepcounter{ex}\Pluudapan%điều khiển lưu đáp 
\ifthenelse{\equal{#1}{}}{\begin{vdbox}{\tieudeboxvd\ \theex}}
{\begin{vdbox}{\tieudeboxvd\ \theex\ {\mdseries({\small\textit{#1}})}}}
}
{\exbend}
%%%==============Hết moi truong vdex===============%%%

 
%%%============Dòng kẻ 22/7=============%%%
\newcommand{\exbdongke}{}
\newcommand{\sodongke}[1][5]{
\ifnum #1>0
	\renewcommand{\exbdongke}{
		\begin{center}
		{\fontfamily{qag}\selectfont\bfseries Bài Làm}\vspace*{-\baselineskip}
		\end{center}
		\foreach \k in {1,...,#1}
			{\par\vspace*{2pt}\noindent\makebox[\linewidth]{\dotfill}}
	}
\else
	\renewcommand{\exbdongke}{}
\fi
}
%%%============Het dòng kẻ 22/7=============%%%
%%%============Tạo dòng kẻ 22/7=============%%%
\newcommand{\taodongke}[1][5]{
	\sodongke[#1]	
	\global\setbool{blluudapan}{true}
}
\newcommand{\vohieuhoadongke}{\renewcommand{\taodongke}[1][]{}}
%%%============Het tạo dòng kẻ 22/7=============%%%
\newcommand{\hdg}{Hướng dẫn giải}
%%%============Lenh hien thi dap an=============%%%
\newcommand{\hienthidapan}{\global\setbool{blluudapan}{false}
	\renewcommand{\exbdongke}{}
	\renewcommand{\loigiai}[1]{\ifbool{blvd}{
		\renewcommand{\boxend}{\end{tcolorbox}
			\vspace*{-\baselineskip}
			\begin{center}
			{\bfseries\selectfont Lời giải} 
			\end{center}
			\vspace*{-0.8\baselineskip}
			\begingroup
			\noindent##1
			\endgroup}
		 }{\ifbool{blbaitap}{
		 		\renewcommand{\boxendbt}{\end{tcolorbox}
		 			\vspace*{-\baselineskip}
		 			\begin{center}
		 			{\it\bfseries\selectfont\noindent \hdg\\
		 			} 
		 			\end{center}
		 			\vspace*{-0.8\baselineskip}
		 			\begingroup
		 			\noindent##1
		 			\endgroup}
		 	 }{\ifbool{bltracnghiem}{
		 	 		 		\renewcommand{\exend}{\par\noindent{\small\usefont{T5}{HLButlong}{m}{n} \dauloigiaitracnghiem}
		 	 		 				##1\par\noindent
		 	 		 				Chọn đáp án \excircledda{\Alph{numTrue}}
		 	 		 			}
		 	 		 	  \renewcommand{\exbend}{\end{vdbox}\par\noindent{\small\usefont{T5}{HLButlong}{m}{n} \dauloigiaitracnghiem}
		 	 		 	  		 	 	##1\par\noindent
		 	 		 	  		 	 	Chọn đáp án \excircledda{\Alph{numTrue}}
		 	 		 	  		 	}
		 	 }{}}		 
		 }
	}
}
%%%============Hết lenh hien thi dap an=============%%%

%%%============Lenh luu dap an=============%%%
\newbool{blluudapan}
\newcommand{\luudapan}{\sodongke[0]
	\global\setbool{blluudapan}{true}
}
\makeatletter
\newcommand{\loigiai@aux}[2]{a}
\newcommand{\Pluudapan}{ 
%%%==================xử lý câu đầu tiên
\ifbool{blluudapan}{
\ifbool{blvd}{ %truong hop vi 
		\renewcommand{\loigiai}[1]{%
		  \@bsphack\expandafter\loigiai@aux\expandafter{\expanded{\thevd}}{##1}\@esphack
		}
		\renewcommand{\loigiai@aux}[2]{%
		  \begin{collect}{vdcol}{\par\noindent{\fontfamily{qag}\bfseries
		  	\small\selectfont  VD ##1}.\ignorespaces}{}{}{} 
			##2%
		  \end{collect}
		}
	\renewcommand{\boxend}{\end{tcolorbox}\exbdongke}
	}{\ifbool{blbaitap}{ % truong hop bai tap
			\renewcommand{\loigiai}[1]{%
			  \@bsphack\expandafter\loigiai@aux\expandafter{\expanded{\thebt}}{##1}\@esphack
			}
			\renewcommand{\loigiai@aux}[2]{%
			  \begin{collect}{btcol}{\par\noindent{\fontfamily{qag}\bfseries
			  	\small\selectfont  BT ##1}.\ignorespaces}{}{}{} 
				##2%
			  \end{collect}
			}
		\renewcommand{\boxendbt}{\end{tcolorbox}\vspace*{3pt}\exbdongke}
		}{\ifbool{bltracnghiem}{ % truong hop trac nghiem
					\renewcommand{\loigiai}[1]{%
					  \@bsphack\expandafter\loigiai@aux\expandafter{\expanded{\theex.\Alph{numTrue}}}{##1}\@esphack
					}
					\renewcommand{\loigiai@aux}[2]{%
					  \begin{collect}{excol}{\par\noindent{\fontfamily{qag}\bfseries
					  	\small\selectfont  TN ##1}.\ignorespaces}{}{}{} 
						##2
					  \end{collect}
					}
				\renewcommand{\exend}{\exbdongke} 
			    \renewcommand{\exbend}{\end{vdbox}\exbdongke} 
				}{}
		}
	}
}{}%ket thuc dieu kien luu an
}
\makeatother

%%%============Hết lenh luu dap an=============%%%

%%%============Lenh in dap an vi du=============%%%
\newcommand{\dapanvidu}{
\begin{center}
%{\usefont{T5}{HLButlong}{m}{n} ĐÁP ÁN VÍ DỤ}
{\bfseries\fontfamily{ugq}\selectfont ĐÁP ÁN VÍ DỤ}
\end{center}
\vspace*{-\baselineskip}
\includecollection{vdcol}}
%%%============Hết lenh in dap an vi du=============%%%

%%%============Lenh in dap an bai tap=============%%%
\newcommand{\dapanbaitap}{
\begin{center}
%{\usefont{T5}{HLButlong}{m}{n} ĐÁP ÁN VÍ DỤ}
{\bfseries\fontfamily{ugq}\selectfont ĐÁP ÁN BÀI TẬP}
\end{center}
\vspace*{-\baselineskip}
\includecollection{btcol}}
%%%============Hết lenh in dap an bai tap=============%%%

%%%============Lenh in dap an bai tap=============%%%
\newcommand{\dapantracnghiem}{
\begin{center}
%{\usefont{T5}{HLButlong}{m}{n} ĐÁP ÁN VÍ DỤ}
{\bfseries\fontfamily{ugq}\selectfont ĐÁP ÁN TRẮC NGHIỆM}
\end{center}
\vspace*{-\baselineskip}
\includecollection{excol}}
%%%============Hết lenh in dap an bai tap=============%%%

%%%============Dat noi dung cau hoi=============%%%
\newtcolorbox{cauhoi}[1][]{enhanced,breakable,
	before skip=1mm,after skip=1mm,
	boxrule=0pt,left=11mm,right=0mm,top=3mm,bottom=1mm,
	arc=0mm,colback=white,fontupper={\fontfamily{pag}\selectfont\fontsize{13pt}{13pt}},
	underlay={%
		\path  node at ([shift={(5mm,2pt)}]interior.west){\includegraphics[height=25pt]{Khaibao/Logo.pdf}};
	},
	#1}	
%%%============Het dat noi dung cau hoi=============%%%


%%%============Khai báo lệnh viết tắt TIKZ=============%%%

\def\tructam(#1,#2,#3)(#4){
	%Trực tâm tam giác
	\path
	($(#1)!(#3)!(#2)$) coordinate (zzzz1)
	($(#1)!(#2)!(#3)$) coordinate (zzzz2)
	(intersection of #3--zzzz1 and #2--zzzz2) coordinate (#4)
	;
}
\def\trongtam(#1,#2,#3)(#4){
	%Trọng tâm tam giác
	\path
	($(#1)!0.5!(#2)$) coordinate (zzzz1)
	($(#1)!0.5!(#3)$) coordinate (zzzz2)
	(intersection of #3--zzzz1 and #2--zzzz2) coordinate (#4)
	;
}
\def\noitiep(#1,#2,#3)(#4){
	%Tầm đường tròn nội tiếp
	\path
	($(#1)!1mm!(#2)$) coordinate (a1)
	($(#1)!1mm!(#3)$) coordinate (a2)
	($(a1)!0.5!(a2)$) coordinate (at)
	($(#2)!1mm!(#1)$) coordinate (b1)
	($(#2)!1mm!(#3)$) coordinate (b2)
	($(b1)!0.5!(b2)$) coordinate (bt)
	(intersection of #1--at and #2--bt) coordinate (#4)
	;
}
\def\ngoaitiep(#1,#2,#3)(#4){
	%Tầm đường tròn ngoại tiếp
	\path
	($(#1)!0.5!(#2)$) coordinate (A1)
	($(A1)!1mm!90:(#2)$) coordinate (d1)
	($(#1)!0.5!(#3)$) coordinate (A2)
	($(A2)!1mm!90:(#1)$) coordinate (d2)
	(intersection of A1--d1 and A2--d2) coordinate (#4)
	;
}
\def\dtron(#1,#2){
	% Đường tròn tâm A bán kính AB
	\path[name path =#1#2] (#1) let\p1=($(#1)-(#2)$) in circle ({veclen(\x1,\y1)});
	\draw (#1) let\p1=($(#1)-(#2)$) in circle ({veclen(\x1,\y1)});
}
%Chân đường phân giác
\def\chanpg(#1,#2,#3)(#4){
	\path let
	\p1=($(#2)-(#1)$),\p2=($(#2)-(#3)$),
	\n1={veclen(\x1,\y1)},\n2={veclen(\x2,\y2)}
	in ($(#1)!scalar((\n1)/(\n1+\n2))!(#3)$) coordinate(#4);
}
%Tâm đường tròn bàng tiếp
\def\bangtiep(#1,#2,#3)(#4){
	\path
	(#2)--(#1)--([turn]0:1) coordinate (p1t)
	(#2)--(#3)--([turn]0:1) coordinate (p3t);
	\chanpg(#3,#1,p1t)(p1tt)
	\chanpg(#1,#3,p3t)(p3tt)
	\coordinate (#4) at (intersection of #1--p1tt and #3--p3tt);
}
% Điểm đẳng giác của 4 qua tam giác 123 tên là 5
\def\diemdanggiac(#1,#2,#3,#4)(#5){
	\chanpg(#1,#2,#3)(pg2)
	\chanpg(#1,#3,#2)(pg3)
	\path 	($(#2)!(#4)!(pg2)$)	coordinate(pg22)
	($(#4)!2!(pg22)$)	coordinate(dx2)
	($(#3)!(#4)!(pg3)$)	coordinate(pg33)
	($(#4)!2!(pg33)$)	coordinate(dx3)
	(intersection of #2--dx2 and #3--dx3) coordinate (#5) 
	;
}
\def\dxtruc(#1,#2,#3)(#4){
	\path
	($(#2)!(#1)!(#3)$)coordinate(hinhchieu1)
	($(#1)!2!(hinhchieu1)$)coordinate(#4)
	;
}
%% ------ Tọa độ điểm 
\def\toado(#1){
	\path 
	(4,0) coordinate (xi)
	(0,4) coordinate (yi)
	(0,0) coordinate (O)
	($(O)!(#1)!(xi)$) coordinate (x#1)
	($(O)!(#1)!(yi)$) coordinate (y#1);
	\draw[dashed] (x#1)--(#1)--(y#1);
}
%%%============Kết thúc TIKZ=============%%%












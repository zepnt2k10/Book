


\newpage


\begin{center}\color{blue}\usefont{T5}{uag}{db}{sc}
	\MakeUppercase{\LARGE  Lời nói đầu}
\end{center}

\setcounter{page}{3}
{
\linespread{1}
Trong đề thi học kì, thi tuyển sinh vào lớp 10 môn Toán trên cả nước, luôn có một số câu hỏi khó nhằm phân loại thí sinh dự thi. Để làm được những câu này, các em cần có một nền tảng kiến thức tương đối tốt và đã được trau đồi qua quá trình ôn tập. Cuốn sách này sẽ giúp các em ôn luyện kiến thức, củng cố kĩ năng để dành được kết quả cao nhất trong các kì thi học kì, thi tuyển sinh vào lớp 10 sắp tới. Cuốn sách gồm 4 chủ đề
\begin{description}
	\item[Chủ đề 1: Đồ thị và các bài toán liên quan] \,\\	
	Chúng tôi tập trung vào ý cuối mỗi bài đồ thị. Đây không phải nội dung quá khó tuy nhiên những năm gần đây, nhiều thí sinh mất điểm một cách đáng tiếc, đặc biệt các bài toán đồ thị có liên quan đến hình học. Chủ đề này sẽ giúp các em nắm sâu hơn được một số dạng toán cơ bản về đồ thị đồng thời chú trọng khâu trình bày lời giải hợp lý, logic.
	
	\item[Chủ đề 2: Phương trình chứa căn thức]\,\\	
	Liệt kê đầy đủ các phương pháp giải phương trình chứa căn thức như bình phương 2 vế, nhân liên hợp, đưa về tích, đặt ẩn phụ và phương pháp đánh giá cùng với hệ thống bài tập có chọn lọc, phù hợp với phạm vi kiến thức THCS. Hi vọng qua chủ đề này sẽ giúp các em giải nhanh một số dạng phương trình chứa căn thức thường gặp.
	
	\item[Chủ đề 3: Hình học] \,\\	
	Chúng tôi cung cấp cho bạn đọc một số bổ đề, định lý thường thường sử dụng trong quá trình giải quyết các ý khó bài hình học. Từ đây, việc quy bài toán lạ về dạng quen thuộc trở nên dễ dàng hơn.
	
	\item[Chủ đề 4: Bất đẳng thức và cực trị]\,\\	
	Cung cấp một số hướng tiếp cận mới cho các bài bất đẳng thức và tìm giá trị lớn nhất/ nhỏ nhất của biểu thức. Đây là những công cụ mạnh giúp giải quyết các bài toán bất đẳng thức và cực trị. Chúng tôi chỉ chắt lọc những cách tiếp cận mới chứ không có ý định “phủ sóng” toàn bộ kiến thức về bất đẳng thức và cực trị trong chủ đề này.
\end{description}

Để cuốn sách được thành công, không thể thiếu được những góp ý của bạn đọc. Chúng tôi rất mong nhận được những ý kiến đánh giá, góp ý để những lần in tiếp theo, cuốn sách hoàn hảo hơn.

Mọi chi tiết, xin bạn đọc liên hệ qua hòm thư sachtoan24h@gmail.com.

\hfill Xin chân thành cám ơn!

\hfill Nhóm tác giả


}


















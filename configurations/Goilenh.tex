\usepackage{picture} %Công cụ tạo header dọc trang (quay header)
\usepackage{cite}
\usepackage{ifthen}
\usepackage{amsmath,amssymb,pb-diagram}
\usepackage{comment}
\usepackage{multirow}
\usepackage{indentfirst}
\usepackage[width=19.20cm, height=26.90cm, left=2.20cm, right=2.20cm, top=2.20cm, bottom=2.20cm]{geometry}
\setlength{\parindent}{0cm}% Thụt lề
\setlength{\parskip}{2.2pt} % khoảng cách giữa 2 đoạn văn bản
\linespread{1.2}  % dãn dòng
\usepackage[svgnames,table]{xcolor}
\usepackage[all]{xy}
\usepackage{bbding}
\usepackage{stmaryrd}
\usepackage{mathptmx}
\usepackage{float}
\usepackage{multicol}
\usepackage{tabularx,booktabs}
\newcolumntype{Y}{>{\centering\arraybackslash}X}
\newcolumntype{Z}{>{\raggedright\arraybackslash}X}
\def\tieudeduoi{Nắm chắc kiến thức và kĩ năng toán 6}
\def\i{\item}
%%%===================Qrcode==============================%%%
%\usepackage{tikz}
%\usepackage{qrcode}				%Qrcode
%\def\maQR#1#2{
%	\tikz\path
%	node[above,draw=black,line width=2pt,rounded corners=1mm]{\qrcode[height=1in]{#2}}
%	node[below,rounded corners=1mm]{\href{#2}{#1}}
%	;
%}
%%%===========================================================%%%
%%%%%%%%%%%%%%%%%%%%%%%%%%%%%%%%%%%%%%%%%%
\usepackage[shortlabels]{enumitem}
\setenumerate{itemsep=1pt,topsep=1pt}
\setlist[enumerate,1]{itemsep=0pt,topsep=0pt,label=\protect\circled{\arabic*}}
\setlist[enumerate,2]{itemsep=0pt,topsep=0pt,label=\protect\circledqd{\small\alph*},wide=0.5cm,
leftmargin=0.5cm}
\setlist[enumerate,3]{itemsep=0pt,topsep=0pt,label=\protect\circledh{\small\roman*},wide=1cm,leftmargin=1cm}
\setlist[itemize,1]{itemsep=0pt,topsep=0pt,label=\protect{\small\starletdot}}
\setlist[itemize,2]{itemsep=0pt,topsep=0pt,label=\protect{\small\rhombusdot},wide=0.35cm,
leftmargin=0.35cm}
\setlist[itemize,3]{itemsep=0pt,topsep=0pt,label=\protect{\small\faAngellist},wide=0.5cm,leftmargin=.5cm}
%%%%%%%%%%%%%%%%%%%%%%%%%%%%%%%%%%%%%%%%%%%%%%%%
\usepackage{fancyhdr}
\renewcommand{\headrulewidth}{0pt}
%\renewcommand\sectionmark[1]{%
%	\markright{Bài \thesection.\ #1}}
%\renewcommand{\chaptermark}[1]{\markboth{Chuyên đề \thechapter.\ #1}{}}
%%%%%%%%%%%%%%%%%%%%%%%%%%%%%%%%%%%%%%%%%
\usepackage{graphicx}
\usepackage[labelsep=period]{caption}
\usepackage{subcaption}
%%%%%%%%%%%%%%%%%%%%%%%%%%%%%%%%%%%%%%%%%%%

%%%%%%%%%%%%%%%%%%%%%%%%%%%%%%%%%%%%%%%%%%
\newcommand\vt[1]{\overrightarrow{#1}}
\newcommand\sm[1]{{\displaystyle\sum\limits_{i=1}^{#1}}}
\newcommand\sk[1]{{\displaystyle\sum_{i=0}^{#1}}}
%%%%%%%%%%%%%%%%%%%%%%%%%%%%%%%%%%%%%%%%%
%Theorem
\usepackage{pgf,tikz,pgfplots}
\usetikzlibrary{arrows}
\usetikzlibrary[patterns]
\usetikzlibrary{shapes.geometric}
%%%%%%%%%%%%%%%%%%%%%%%%%%%%%%%%%%%%%%%%%%%%%
%%  Điều chỉnh chữ trong tiêu đề
\setcounter{secnumdepth}{5}
\numberwithin{equation}{section}
\numberwithin{figure}{section}
\renewcommand{\thepart}{\Roman{part}.}
\renewcommand{\thechapter}{\arabic{chapter}}
\renewcommand{\thesection}{{\usefont{T5}{uag}{db}{sc}vấn đề \arabic{section}.}}
\renewcommand{\thesubsection}{\Alph{subsection}.}
\renewcommand{\thesubsubsection}{\arabic{subsubsection}.} \renewcommand{\theequation}{\arabic{chapter}.\arabic{section}.\arabic{equation}}
\renewcommand{\thefigure}{\arabic{chapter}.\arabic{section}.\arabic{figure}}
%%%%%%%%%%%%%%%%%%%%%%%%%%%%%%%%%%%%%%%%%%%%%%%%%%%%
%%%%%%%%%%%%%%%%%%%%%%%%%%%%%%%%%%%%%%%%%%%%%%%%%%%%%%%%
%\usepackage[answerdelayed]{exercise}
%\renewcommand{\QuestionNB}{(\alph{Question})}
%\renewcommand{\AnswerHeader}{\vspace*{-16pt}}
%\setlength{\ExerciseSkipBefore}{0.1\baselineskip}
%\setlength{\ExerciseSkipAfter}{0pt}
%\setlength{\QuestionBefore}{1pt}
%\font\nsieu=vnr10 at 1pt
%%%%%%%%%%%%%%%%%%%%%%%%%%%%%%%%%
\def\R{\mathbb{R}}
\def\N{\mathbb{N}}
%%%%%%%%%%%%%%%%%%%%%%%%%%%%%%%%%%%%%%%%%%%%%%%%
\usepackage{wrapfig}
\usepackage{yhmath}%\wideparen
%%%%%%%%%%%%%%%%%%%%%%%%%%%%%%%%%%%%%%%%%%%%%%%%%%%%%%%%%
%\usepackage{contour} %Lặp lại từ, dùng làm đậm small cap
\usepackage{multido}
%%%%%%%%%%%%%%%%%%%%%%%%%%%%%%%%%%%%%%%%%%
%%%%%%%%%%%%%%%%%%%%%%%%%%%%%%%%%%%%%%%%%%%%
%%%Môi trường liệt kê
\usepackage{oplotsymbl} % Biểu tượng
\usepackage{fontawesome} %Biểu tượng
\newcommand\circled[1]{\tikz[baseline=(char.base)]{
     \node[shape=circle,minimum size =15pt,
     inner sep=1.25pt] (char) 
            {\fontfamily{qhv}\bfseries\small\selectfont#1};}}
\newcommand\circledqd[1]{\tikz[baseline=(char.base)]{\node[shape=circle,fill=cyan!10,draw,inner sep=1pt,
outer sep=0pt,minimum size =15pt,font=\fontsize{12}{12}\selectfont] (char){#1};}}
\newcommand\circledh[1]{\tikz[baseline=(char.base)]{\node[shape=circle,draw,minimum size =15pt, 
inner sep=1.25pt,font=\fontsize{12}{12}\selectfont] (char){#1};}}
\newcommand\diamondnumh[1]{\tikz[baseline=(char.base)]
	\node[draw,double,diamond, text width=9pt,align=center,inner sep=1pt,font=\fontsize{12}{12}\selectfont\bfseries] (char){#1};}
\usetikzlibrary{intersections,shadings,calc,angles,quotes}
\usepackage{tkz-euclide}
\usepackage{tkz-tab}
%\definecolor{xanhnhat}{RGB}{224,255,255}
%%%%%%%%%%%%%%%================================%%%%%%%%%%%%%
%%%Dóng khung
\newcommand\khungct[1]
{\noindent
\begin{tikzpicture}%
	\tikzstyle{khung} = [rectangle, inner sep=8pt,double]%
	\node [khung] {%
		$
		\; #1.\;
		$
	};%
	\end{tikzpicture}}
\newcommand\khung[1]
{\noindent\begin{tikzpicture}%
	\tikzstyle{khung} = [rectangle, inner sep=6pt,thick]%
	\node [khung]{%
		$
		\; #1\;
		$
	};%
	\end{tikzpicture}}
%%%%====================================================%%%%
%%%===================Hệ hoặc, hệ và===================%%%
\newcommand{\hoac}[1]{
\left[\begin{aligned}#1\end{aligned}\right. }
\newcommand{\heva}[1]{
\left\{\begin{aligned}#1\end{aligned}\right.}
\usepackage{longtable}
%%%================Viết tắt lệnh=========================%%%%
\usepackage{pgfornament}